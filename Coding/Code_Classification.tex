
%---------------------------------------------------------------------------------------------------------------------------------------%

\noindent \textbf{ Classification of Codes}
Classification of codes is best illustrated by an example. Consider Table 10-1 where a source of
size 4 has been encoded in binary codes with symbol 0 and 1.\\ \bigskip
% Table 10-I Binary Codes
\begin{tabular}{c c c c c c c}
X& Code l& Code 2& Code 3 &Code 4& Code 5& Code 6\\
a& 01& 01 &1 &10 &01 &01\\
b& 01& 01 &1 &10 &01 &01\\
c &00 &10& 00& 110& 011 &001\\
d &11& 11& 11& 111 &0111 &0001\\
\end{tabular}


%---------------------------------------------------------------------------------------------------------------------------------------%

\noindent \textbf{Categorisation of Codes}
\begin{enumerate}
\item Fixed Length Codes
\item Variable Length Codes
\item Distinct Codes
\item Prefix-Free Codes
\item Uniquely decodable codes
\item Instantaneous Codes
\item Optimal Codes
\end{enumerate}

%---------------------------------------------------------------------------------------------------------------------------------------%

\begin{itemize}
\item[1.] Fixed-Length Codes: A fixed-length code is one whose code word length is fixed. Code l and code Z of Table 10»l are
fixed-length codes with length 2.
\item[2.] Variable-Length Codes: A variable-length code is one whose code word length is not Exed, All codes of Table l0-l except
codes 1 and 2 are variable-length codes.
\item[3.] Distinct Codes:
A code is distinct if each code word is distinguishable from other code words. All codes of Table
10-1 except code 1 arc distinct codes—notice the codes for xl and xi.
\item[4.] Prefix-Free Codes:
A code in which no code word can be formed by adding code symbols to another code word is
called a prefix-free code. Thus, in a prefix-free code no code word is a prefix of another. Codes 2, 4,
and 6 of Table 10-l are prefix-free codes.
\end{itemize}


%-----------------------------------------------------------------------------------------------------------------------------------------------------------%

\textbf{5. Uniquely Decodable Codes}
A distinct code is uniquely devudable if the original source sequence can be reconstructed perfectly
from the encoded binary sequence. Note that code 3 of Table 10-1 is not a uniquely decodable code.
For example, the binary sequence 1001 may correspond to the source sequences xzxgxz orxzxlxlxz.
\\ \bigskip
A sufticient condition to ensure that a code is uniquely decodable is that no code word is a prefix of
another. Thus, the prefix-free codes 2, 4, and 6 are uniquely decodable codes. Note that the pretix—t`ree
condition is not a necessary condition for unique decodability. \i\bigskip For example, code 5 of Table l0-l does
not satisfy the prefix-free condition, and yet it is uniquely decodable since the bit 0 indicates the
beginning of each code word of the code.

%-----------------------------------------------------------------------------------------------------------------------------------------------------------%

\textbf{6. Instantaneous Codes}
A uniquely decodable code is cmlled an inrtanruneous code if the end of any code word is
recognizable without examining subsequent code symbols. The instantaneous codes have the property
previously mentioned that no code word is a pretix of another code word. For this reason, prefix-free
codes are sometimes called instantaneous codes.
\textbf{7. Optimal Codes}
A code is said to be optimal if it is instantaneous and has mini.mu.rn average length L for a given
source with a given probability assignment for the source symbols.
