
\documentclass[a4paper,12pt]{article}
%%%%%%%%%%%%%%%%%%%%%%%%%%%%%%%%%%%%%%%%%%%%%%%%%%%%%%%%%%%%%%%%%%%%%%%%%%%%%%%%%%%%%%%%%%%%%%%%%%%%%%%%%%%%%%%%%%%%%%%%%%%%%%%%%%%%%%%%%%%%%%%%%%%%%%%%%%%%%%%%%%%%%%%%%%%%%%%%%%%%%%%%%%%%%%%%%%%%%%%%%%%%%%%%%%%%%%%%%%%%%%%%%%%%%%%%%%%%%%%%%%%%%%%%%%%%
\usepackage{eurosym}
\usepackage{vmargin}
\usepackage{amsmath}
\usepackage{graphics}
\usepackage{epsfig}
\usepackage{subfigure}
\usepackage{enumerate}
\usepackage{fancyhdr}
\usepackage{framed}

\setcounter{MaxMatrixCols}{10}
%TCIDATA{OutputFilter=LATEX.DLL}
%TCIDATA{Version=5.00.0.2570}
%TCIDATA{<META NAME="SaveForMode"CONTENT="1">}
%TCIDATA{LastRevised=Wednesday, February 23, 201113:24:34}
%TCIDATA{<META NAME="GraphicsSave" CONTENT="32">}
%TCIDATA{Language=American English}

\pagestyle{fancy}
\setmarginsrb{20mm}{0mm}{20mm}{25mm}{12mm}{11mm}{0mm}{11mm}
\lhead{StatsResource} \rhead{Worked Examples} \chead{Probability} %\input{tcilatex}

\begin{document}

\large 
\subsection*{Probability: Worked Example }
An electronics assembly subcontractor receives resistors from two suppliers: Deltatech provides
70\% of the subcontractors's resistors while another company, Echelon, supplies the remainder.
\\
\medskip
1\% of the resistors provided by Deltatech fail the quality control test, while 2\% of the
resistors from Echelon also fail the quality control test.

\begin{enumerate}[(a)]
\item What is the probability that a resistor will fail the quality control test?
\item What is the probability that a resistor that fails the quality control test was supplied by Echelon?
\end{enumerate}

\subsection*{Solution}
Firstly, let's assign names to each event.
\begin{itemize}
\item $D$ : a randomly chosen resistor comes from Deltatech.
\item $E$ : a randomly chosen resistor comes from Echelon.
\item $F$ : a randomly chosen resistor fails the quality control test.
\item $P$ : a randomly chosen resistor passes the quality control test.
\end{itemize}
\bigskip

\noindent We are given (or can deduce) the following probabilities:
\begin{itemize}
\item $P(D) = 0.70$,
\item $P(E) = 0.30$.
\end{itemize}
\medskip


\noindent We are given two more important pieces of information:
\begin{itemize}
\item The probability that a randomly chosen resistor fails the quality control test, given that it comes from Deltatech: $P(F|D) = 0.01 $.
\item The probability that a randomly chosen resistor fails the quality control test, given that it comes from Echelon: $P(F|E) = 0.02$.
\end{itemize}


\noindent The first question asks us to compute the probability that a randomly chosen resistor fails the quality control test. i.e. $P(F)$.\\
\bigskip
\noindent All resistors come from either Deltatech or Echelon. So, using the \textbf{\emph{law of total probability}}, we can express $P(F)$ as follows:

\[ P(F)  = P(F \cap D) + P(F \cap E) \]


\noindent Using the \textbf{\emph{multiplication rule}}  i.e. $P(A \cap B) = P(A|B) \times P(B)$, we can re-express the formula as follows

\[ P(F)  = P(F|D) \times P(D) + P(F|E) \times P(E) \]

\noindent We have all the necessary probabilities to solve this.

\[ P(F)  = 0.01 \times 0.70 + 0.02 \times 0.30   = 0.007 + 0.006  = 0.013\]


%%%%%%%%%%%%%%%%%%%%%%%%%%%%%%%%%%%%%%%%%

\medskip

\begin{itemize}
\item
The second question asks us to compute probability that a resistor that fails the quality control test was supplied by Echelon.
\item In other words; of the resistors that did fail the quality test only, what is the probability that a randomly selected resistor was supplied by Echelon?
\item We can express this mathematically as $P(E|F)$.
\item We can use \textbf{\emph{Bayes' theorem}} to compute the answer.
\end{itemize}



\noindent \textbf{Bayes Theorem}
Recall Bayes' theorem
\[ P(A|B) = \frac{P(B|A)\times P(A)}{P(B)} \]
\bigskip

\[ P(E|F) = \frac{P(F|E)\times P(E)}{P(F)}  =  \frac{0.02 \times 0.30}{0.013} = 0.46\]

%%%%%%%%%%%%%%%%%%%%%%%%%%%%%%%%%%%%%%%%%5
\newpage
BLANKS
\end{document}
