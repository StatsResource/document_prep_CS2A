\item Let $P = $ ``the individual drinks Pepsi'' and $C = $ ``the individual drink Coca Coca''.\\ Furthermore $\Pr(P) = 0.72$, $\Pr(C) = 0.24$ and $\Pr(P \cap C) = 0.12$.\\[-0.2cm]

\begin{enumerate}[(i)]
\item Calculate the probability that an individual drinks both brands? 
\item Calculate the probability that an individual drinks neither brand? 
\item Are $P$ and $C$ mutually exclusive? 
%\item Are $W$ and $M$ independent?
\end{enumerate}

\begin{framed}


\begin{itemize}
\item Given $\Pr(P) = 0.72$, $\Pr(C) = 0.24$
\item Person drinks both Pepsi and Coca Cola $\Pr(P \cap C) = 0.12$
\item Person drinks either Pepsi or Coca Cola  or Both
\[\Pr(P \cap C) = \Pr(P) +  \Pr(C)   - \Pr(P \cap C) \]
\[\Pr(P \cap C) = 0.72 +  0.24  - 0.12 = 0.84\]
\item Remark: subtract the "P and C" component to prevent double-coutning
\item Remark: People who dont drink either 0.16.
\item Question: Are $P$ and $C$ mutually exclusive?
\\ No - you can drink both (according to the maths here)
\end{itemize}
\end{framed}

\item \textbf{Joint Probability Tables}
% PMS Autumn 2009 Question 8

Find $E[X|Y=2]$

\begin{tabular}{ccccc}
& X=0  & X=1  & X=2  &              \\ \hline
Y=1 & 0.15 & 0.2  & 0.25 & P(Y=1) = 0.6 \\ \hline
Y=2 & 0.05 & 0.15 & 0.20 & P(Y=2) = 0.4 \\ \hline
& P(X=0) = 0.2  & P(X=1) = 0.35  & P(X=2)=0.45  &              \\ \hline
\end{tabular}
%%%%%%%%%%%%%%%%%%%%%%%%%%%%%%%%%%%%%%%%%%%%%%%%%%%%%%%%%%%%%%%%%%%%%%%%%%%%%%%%%%%%


\begin{framed}
\textbf{Solution}
\[   \frac{(0 \times 0.05) + (1 \times 0.15)+(2 \times 0.2) }{0.4}  = \frac{0.55}{0.4}  \] 

$E[X|Y=2] = 1.375$
\end{framed}


\item \textbf{Sampling}
A lot contains 13 items of which 4 are defective. Three items are drawn at random from the lot one after the other. Find the probability $p$ that all three are non-defective.



\item \textbf{Probability of Two Dice Rolls}
A pair of dice is thrown. Let X denote the minimum of the two numbers which occur.
Find the distributions and expected value of X.

\item The following contingency table illustrates the number of 200 students in different
departments according to gender.

\begin{center}
\begin{tabular}{|c|c|c|c|c|}
\hline
% after \\: \hline or \cline{col1-col2} \cline{col3-col4} ...
& Physics & Biology & Chemistry & Total \\\hline
Males & 30 & 20 & 50 & 100 \\  \hline
Females & 20 & 50 & 30 & 100 \\ \hline
Total & 50 & 70 & 80 & 200 \\
\hline
\end{tabular}
\end{center}

\begin{itemize}
\item[a.] (1 mark) What is the probability that a randomly chosen person from the sample is a
Chemistry student?
\item[b.] (1 mark) What is the probability that a randomly chosen person from the sample is both female and studying Biology?
\item[c.] (1 mark) Given that the student is female, what is the probability that she is an
Biology student?
\item[d.] (1 mark) Given that a student studies Biology, what is the probability that the student is female?
\end{itemize}
