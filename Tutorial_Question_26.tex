
\subsection*{Test 2. F Test for equality of variance}
\begin{itemize}
	\item In this procedure, we determine whether or not two \textit{\textbf{populations}} have the same variance.
	\item The assumption of equal variance of two populations underpins several inference procedures. This assumption is tested by comparing the variance of samples taken from both populations.
	\item The null and alternative hypotheses are as follows:
	\[ H_0: \sigma^2_1 = \sigma^2_2 \]
	\[ H_1: \sigma^2_1 \neq \sigma^2_2 \]
\end{itemize}
\begin{framed}
	\begin{verbatim}
	> var.test(X,Y)
	
	F test to compare two variances
	
	data:  X and Y
	F = 2.5122, num df = 9, denom df = 9, p-value = 0.1862
	alternative hypothesis: true ratio of variances
	is not equal to 1
	95 percent confidence interval:
	0.6239986 10.1141624
	sample estimates:
	ratio of variances 
	2.512215 
	\end{verbatim}
\end{framed}

\subsection*{Test 3. Shapiro Wilk's Test for Normality}
\begin{itemize}
	\item We will often be required to determine whether or not a data set is normally distributed.
	This assumption underpins many statistical models.
	\item The null hypothesis is that the data set is normally distributed.
	\item The alternative hypothesis is that the data set is not normally distributed.
	\item One procedure for testing these hypotheses is the Shapiro-Wilk test, implemented in \texttt{R} using the command \texttt{shapiro.test()}.
\end{itemize}
\begin{framed}
	\begin{verbatim}
	> shapiro.test(X)
	
	Shapiro-Wilk normality test
	
	data:  X
	W = 0.9849, p-value = 0.1012
	
	
	\end{verbatim}
\end{framed}

\end{document}	