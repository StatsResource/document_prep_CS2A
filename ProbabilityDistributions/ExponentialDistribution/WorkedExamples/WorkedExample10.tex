
\documentclass[a4paper,12pt]{article}
%%%%%%%%%%%%%%%%%%%%%%%%%%%%%%%%%%%%%%%%%%%%%%%%%%%%%%%%%%%%%%%%%%%%%%%%%%%%%%%%%%%%%%%%%%%%%%%%%%%%%%%%%%%%%%%%%%%%%%%%%%%%%%%%%%%%%%%%%%%%%%%%%%%%%%%%%%%%%%%%%%%%%%%%%%%%%%%%%%%%%%%%%%%%%%%%%%%%%%%%%%%%%%%%%%%%%%%%%%%%%%%%%%%%%%%%%%%%%%%%%%%%%%%%%%%%
\usepackage{eurosym}
\usepackage{vmargin}
\usepackage{amsmath}
\usepackage{graphics}
\usepackage{epsfig}
\usepackage{subfigure}
\usepackage{enumerate}
\usepackage{fancyhdr}
\usepackage{framed}

\setcounter{MaxMatrixCols}{10}
%TCIDATA{OutputFilter=LATEX.DLL}
%TCIDATA{Version=5.00.0.2570}
%TCIDATA{<META NAME="SaveForMode"CONTENT="1">}
%TCIDATA{LastRevised=Wednesday, February 23, 201113:24:34}
%TCIDATA{<META NAME="GraphicsSave" CONTENT="32">}
%TCIDATA{Language=American English}

\pagestyle{fancy}
\setmarginsrb{20mm}{0mm}{20mm}{25mm}{12mm}{11mm}{0mm}{11mm}
\lhead{StatsResource} \rhead{Worked Examples} \chead{Exponential Distribution} %\input{tcilatex}

\begin{document}
\large 

\section*{Exponential Distribution Example: Laptop Lifetimes}

The average lifespan of a laptop is 5 years. You may assume that
the lifespan of computers follows an exponential probability
distribution. 
\begin{enumerate}[(a)]
\item  What is the
probability that the lifespan of the laptop will be at least 6
years? \item 

What is the probability that the lifespan of the laptop will not
exceed 4 years? \item  What is the probability of the
lifespan being between 5 years and 6 years?
\end{enumerate}
\medskip 
\subsection*{Remark}

Here we are told the exponential mean $\mu$, which is related to the rate parameter as follows:
\[ \mu = \frac{1}{\lambda}\]

\subsection*{Solution}
Suppose the lifetime of a PC is exponentially distributed with
mean $\mu =5$. We should be told the average lifetime $\mu$.

\begin{framed}
\[
P( X \geq k) = e^{{-k \over \mu}}
\]
\[
P( X \leq k) = 1 - e^{{-k \over \mu}}
\]
\end{framed}
\large 
\begin{itemize}
\item[(a)] $P(X \geq 6) = e^{-6/5} =  e^{-1.2} = 0.3012$
\item[(b)] $P(X \leq 4) =  1 - e^{-0.8} = 0.5506$
\item[(c)] $P(5 \leq X \leq 6) = 1- [ P( X \leq 5) +  P( X \geq 6)]$ \\ \smallskip
\begin{eqnarray*} 
P( X \leq 5)  &=& 1 - e^{-5/5} \\
&=& 1 - e^{-1} \\
&=& 0.6321 \\
\end{eqnarray*}
\\ \smallskip
\begin{eqnarray*} 
P(5 \leq X \leq 6) &=& 1 - (0.6321 + 0.3012) \\ 
&=& 0.0667\\
&=& 6.67 \%
\end{eqnarray*}
\end{itemize}	
%%	\item[(c)] Alternative approach to (b)\\$P(5 \leq X \leq 10)$ \\ = $P( X \geq 5) - P( X \geq 10)$ \\
%%	= $e^{-0.5} - e^{-1}$
%%	=0.6065 - 0.3678\\
%%	= 0.2386 = 23.86 $\%$
%%\end{itemize}

\newpage
BLANK
\end{document}
