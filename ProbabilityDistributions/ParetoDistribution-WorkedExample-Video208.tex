

\documentclass[a4paper,12pt]{article}
%%%%%%%%%%%%%%%%%%%%%%%%%%%%%%%%%%%%%%%%%%%%%%%%%%%%%%%%%%%%%%%%%%%%%%%%%%%%%%%%%%%%%%%%%%%%%%%%%%%%%%%%%%%%%%%%%%%%%%%%%%%%%%%%%%%%%%%%%%%%%%%%%%%%%%%%%%%%%%%%%%%%%%%%%%%%%%%%%%%%%%%%%%%%%%%%%%%%%%%%%%%%%%%%%%%%%%%%%%%%%%%%%%%%%%%%%%%%%%%%%%%%%%%%%%%%
\usepackage{eurosym}
\usepackage{vmargin}
\usepackage{amsmath}
\usepackage{graphics}
\usepackage{epsfig}
\usepackage{multicol}
\usepackage{subfigure}
\usepackage{enumerate}
\usepackage{fancyhdr}
\usepackage{framed}

\setcounter{MaxMatrixCols}{10}
%TCIDATA{OutputFilter=LATEX.DLL}
%TCIDATA{Version=5.00.0.2570}
%TCIDATA{<META NAME="SaveForMode"CONTENT="1">}
%TCIDATA{LastRevised=Wednesday, February 23, 201113:24:34}
%TCIDATA{<META NAME="GraphicsSave" CONTENT="32">}
%TCIDATA{Language=American English}

\pagestyle{fancy}
\setmarginsrb{20mm}{0mm}{20mm}{25mm}{12mm}{11mm}{0mm}{11mm}
\lhead{StatsResource} \rhead{The Pareto Distribution} \chead{Probability Distributions} %\input{tcilatex}

\begin{document}
\large 

\section*{The Pareto Distribution Worked Example}

Suppose the distribution of monthly salaries of full-time workers in the UK has
a Pareto distribution with minimum monthly salary $x_m = 1000$ and concentration
factor $\alpha = 3$. \\
\medskip

\noindent \textbf{Exercises}
\begin{enumerate}[(a)]
\item Calculate the mean monthly salary of UK full-time workers.
\item Calculate the probability that a UK full-time worker earns more than 2000 per month.
\item Calculate the median monthly salary of UK full-time workers.
\end{enumerate}

\medskip
\noindent \textbf{Part (a)}\\
\noindent The expected value of a random variable following a Pareto distribution is
\[E(X)= \begin{cases} \infty & \mbox{if }\alpha\le 1, \\ \frac{\alpha x_\mathrm{m}}{\alpha-1} & \mbox{if }\alpha>1. \end{cases}
\]

\medskip
Because \textbf{$\alpha$} = $3$, we will use this
{

\[
E(X)= \frac{\alpha x_\mathrm{m}}{\alpha-1}   
\]
}
Recall that $X_m$ = 1000.

\newpage 
\noindent \textbf{Part (b)}\\
\begin{framed}
\noindent The cumulative distribution function of a Pareto random variable with parameters $\alpha$ and $x_m$ is
\[
F_X(x) = \begin{cases}
1-\left(\frac{x_\mathrm{m}}{x}\right)^\alpha & \mbox{for } x \ge x_\mathrm{m}, \\
0 & \mbox{for }x < x_\mathrm{m}.
\end{cases}
\]
\end{framed}

\noindent Using values for this example:
\[
F_X(x) = \begin{cases}
1-\left(\frac{1000}{x}\right)^3 & \mbox{for } x \ge 1000, \\
0 & \mbox{for }x < 1000.
\end{cases}
\]
\[
P(X \geq x) = \begin{cases}
\left(\frac{1000}{x}\right)^3 & \mbox{for } x \ge 1000, \\
1 & \mbox{for }x < 1000.
\end{cases}
\]


\noindent Calculate the probability that a UK full-time worker earns \textit{\textbf{more than}} 2000 per month.

\[P(X \geq 2,000) =  \left(\frac{1000}{2000}\right)^3 = (0.5)^3 = 0.125 \]

\newpage 
\noindent \textbf{Part (c)}\\

\noindent Calculate the median monthly salary of UK full-time workers.

\[ \mbox{Median}: F_X(x) = 0.50\]

{

\[
F_X(x) = \begin{cases}
1-\left(\frac{1000}{x}\right)^3 & \mbox{for } x \ge 1000, \\
0 & \mbox{for }x < 1000.
\end{cases}
\]
}




\[ F_X(x) = 0.5 \qquad \rightarrow \qquad 1-\left(\frac{1000}{x}\right)^3 = 0.50\]
\[ \left(\frac{1000}{x}\right)^3 = 0.50\]

\noindent Get the cube root of both sides
\[ \left(\frac{1000}{x}\right) = \sqrt[3]{0.5} = 0.7937 \]



\[\frac{1000}{0.7937} = 1259.92\]




\end{document}
%----------------------------------------- %
