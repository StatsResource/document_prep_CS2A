%%%%%%%%%%%%%%%%%%%%%%%%%%%%%%%%%%%%%%%%%%%%%%%%%%%%%%%%%%%%%%%%%%%%%%%%%%%%%%%%%%%%%%%%%%%
% HEADER
\documentclass[a4paper,12pt]{article}
\usepackage{eurosym}
\usepackage{vmargin}
\usepackage{amsmath}
\usepackage{graphics}
\usepackage{epsfig}
\usepackage{enumerate}
\usepackage{multicol}
\usepackage{subfigure}
\usepackage{fancyhdr}
\usepackage{listings}
\usepackage{framed}
\usepackage{graphicx}
\usepackage{amsmath}
\usepackage{chngpage}
\usepackage{vmargin}
\setmargins{2.0cm}{2.5cm}{16 cm}{22cm}{0.5cm}{0cm}{1cm}{1cm}
\renewcommand{\baselinestretch}{1.3}
\setcounter{MaxMatrixCols}{10}

\begin{document}
%%%%%%%%%%%%%%%%%%%%%%%%%%%%%%%%%%%%%%%%%%%%%%%%%%%%%%%%%%%%%%%%%%%%%%%%%%%%%%%%%%%%%%%%%%%

3 At the start of the week, Alex can choose salad, pizza or sushi for lunch. On Monday
Alex starts with a uniformly random choice, but in the following days makes the
choices following a discrete time Markov chain with transition matrix:
𝑃 􀵌 􀵭
0.3 0.2 0.5
0.1 0.3 0.6
0.3 0.2 0.5
􀵱
where the 1st, 2nd and 3rd rows and columns correspond to salad, pizza and sushi,
respectively.

\begin{enumerate}[(a)]
\item Derive the probability that Alex has pizza on Monday and salad on
Wednesday. 
\item Calculate the probability that Alex has sushi on Wednesday and Friday. [6]
\end{enumerate}
%%%%%%%%%%%%%%%%%%%%%%%%%%%%%%%%%%%%%%%%%%%%%%%%%%%%%%%%%%%%%%%%%%%%%%%%%%%%%%%%%%%%%%%%%%%%%%%%%%%

\newpage

Q3
(i)
Let X_1, X_2 …X_5 be the choice of meal in the days 1-Monday, 2-Tuesday, through to … 5-Friday.
P(X_1=Pizza, X_3=Salad)= P(X_3=Salad|X_1=Pizza)* P(X_1=Pizza) 
=P^2(2,1) * 1/3 
=0.24*1/3 
=0.08 

% This question was poorly answered and in terms of average marks was the question that attracted lowest marks as a percentage of those available.
%In part \item candidates who scored well set out a clear structure that aligned with the question’s request to “demonstrate” the equality given.
For both parts \item (a) and (b) the marks could be obtained either by starting with LHS of the equations in the question and substituting in the CDF of the uniform / pareto to obtain the RHS or by starting with the RHS and applying the values of A,B,C to obtain the uniform / pareto CDFs. In either case, clarity of structure to the answer is key to demonstrating understanding of the loss distributions in question.
Part \item was particularly poorly answered and candidates are reminded of the importance of revising the whole subject syllabus including Extreme Value.
\newline
Where P^2=􀵭0.30.20.50.10.30.60.30.20.5􀵱∗􀵭0.30.20.50.10.30.60.30.20.5􀵱 =􀵭0.260.220.520.240.230.530.260.220.52􀵱

% (Full marks also available where the candidate uses the relevant direct probability calculations rather than using the P^2 matrix)

%%%%%%%%%%%%%%%%%%%%%%%%%%%%%%%%%%%%%%%%%%%%%%%%%%%%%%
(ii)
P(X_3=Sushi, X_5=Sushi)= P(X_1=Pizza , X_3=Sushi,X_5=Sushi)+ P(X_1=Salad, X_3=Sushi,X_5=Sushi) + P(X_1=X_3 = X_5 =Sushi) 
For the first term:
P(X_1=Pizza , X_3=Sushi,X_5=Sushi)= 
P(X_1=Pizza) P( X_3=Sushi,X_5=Sushi| X_1=Pizza) = 
P(X_1=Pizza) P( X_3=Sushi | X_1=Pizza)* P( X_5=Sushi | X_3=Sushi)= 
1/3 * P^2(2,3) * P^2(3,3) = 1/3*0.53*0.52 
=0.09187 
Similarly the second term is
P(X_1=Salad , X_3=Sushi,X_5=Sushi)= 
1/3* P^2(1,3)* P^2(3,3)=1/3*0.52*0.52 
=0.09013 
and the final term is
P(X_1 = X_3 = X5 = Sushi) = 1/3*052*0.52 = 0.09013 
And so the final figure is 0.09187+0.09013+0.09013 = 0.272 


\end{document}
