%%%%%%%%%%%%%%%%%%%%%%%%%%%%%%%%%%%%%%%%%%%%%%%%%%%%%%%%%%%%%%%%%%%%%%%%%%%%%%%%%%%%%%%%%%%
% HEADER
\documentclass[a4paper,12pt]{article}
\usepackage{eurosym}
\usepackage{vmargin}
\usepackage{amsmath}
\usepackage{graphics}
\usepackage{epsfig}
\usepackage{enumerate}
\usepackage{multicol}
\usepackage{subfigure}
\usepackage{fancyhdr}
\usepackage{listings}
\usepackage{framed}
\usepackage{graphicx}
\usepackage{amsmath}
\usepackage{chngpage}
\usepackage{vmargin}
\setmargins{2.0cm}{2.5cm}{16 cm}{22cm}{0.5cm}{0cm}{1cm}{1cm}
\renewcommand{\baselinestretch}{1.3}
\setcounter{MaxMatrixCols}{10}

\begin{document}
%%%%%%%%%%%%%%%%%%%%%%%%%%%%%%%%%%%%%%%%%%%%%%%%%%%%%%%%%%%%%%%%%%%%%%%%%%%%%%%%%%%%%%%%%%%


CS2A S2023–5
7 A survival study followed twelve patients, for a maximum of 10 days each, following
a major surgical operation. From previous similar studies, around one-third of patients
survived 10 days. The condition of all patients (i.e. whether a patient was alive or
dead) was monitored daily. The results are set out below where S(t) is the Kaplan–
Meier estimate of the survival function:
Time since operation (days) S(t)
0 ≤ t < 2 1
2 ≤ t < 4 0.9
4 ≤ t < 5 0.7
5 ≤ t < 7 0.56
7 ≤ t < 10 0.373

\begin{enumerate}[(a)]
\item Calculate the number of deaths and the number of patients who were censored,
stating the times of all deaths and censoring events. 
An expert analyst has voiced concerns about the accuracy of the data, thinking there is
likely to have been an error.
\item Briefly explain the likely source of the expert’s concerns. 
\item State ways in which this study could be improved. 
\end{enumerate}

%%%%%%%%%%%%%%%%%%%%%%%%%%%%%%%%%%%

Q7
(i)
First calculate 1- \lambda at each event
1-\lambda =0.9 (trivial)
1-\lambda =0.7/0.9 = 0.7778
1-\lambda =0.56/0.7 = 0.8
1-\lambda =0.373/0.56 = 0.66607 but accept rounding to 2/3 hereafter

Part \item was poorly answered whereas part \item was reasonably well answered, meaning that the shape of marks given to many candidates was quite unusual for this question.
The main issue with part \item was the tendency of many candidates to use a ‘sum of normal distributions’ rather than a compound distribution approach. Where this was done answers were either incorrect or unnecessarily approximate. The necessary equations for compound Poisson distributions are all found in the Core Reading and the examiners would expect candidates to be familiar with these.
\newline
Thus lambda:
Time since operation (days)
S(t)
1-lambda
lambda
0≤t<2
1
2≤t<4
0.900
0.900
0.100
4≤t<5
0.700
0.778
0.222
5≤t<7
0.56
0.800
0.200
7≤t<10
0.373
0.667
0.333

\lambda = 0.1
The only combination is d=1, n=10 
\lambda = 0.222
The only combination is d=2, n=9 
\lambda = 0.2000
The only combination is d=1, n=5 
\lambda = 0.33333
The only combination, given there must be less than 4 lives at this point is d=1, n=3

Need to account censored events:
Must be 2 censoring events before time = 2 as n=10 at that point 
Must be 2 censoring events at time = 4 to fall from n=9 to n=5 
Must be 1 censoring event at time = 5 or 6 to fall from n=5 to n=3 
So in summary
Time (days)
deaths
Censoring events
<2
0
2
2
1
0
4
2
2
5
1
1 or
1
6
0
7
1
0
(ii)
Unlikely for there to have been random censoring 
Very sick patient in bed all the time; it’s difficult to see how censoring could have occurred, but data indicates this 
(iii)
Collect more data 
Have more frequent observations eg hourly 
Extend the period of observation 
Patients dying in last 2 intervals implies significant lost important data 
Reduce heterogeneity and obtain several different S(t) estimates 
Separate for male/female, existing health conditions, other sensible classes 
\newline
[Marks available 3, maximum 2]
[Total 11]
