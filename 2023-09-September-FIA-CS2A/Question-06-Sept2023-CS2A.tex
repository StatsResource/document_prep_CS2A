%%%%%%%%%%%%%%%%%%%%%%%%%%%%%%%%%%%%%%%%%%%%%%%%%%%%%%%%%%%%%%%%%%%%%%%%%%%%%%%%%%%%%%%%%%%
% HEADER
\documentclass[a4paper,12pt]{article}
\usepackage{eurosym}
\usepackage{vmargin}
\usepackage{amsmath}
\usepackage{graphics}
\usepackage{epsfig}
\usepackage{enumerate}
\usepackage{multicol}
\usepackage{subfigure}
\usepackage{fancyhdr}
\usepackage{listings}
\usepackage{framed}
\usepackage{graphicx}
\usepackage{amsmath}
\usepackage{chngpage}
\usepackage{vmargin}
\setmargins{2.0cm}{2.5cm}{16 cm}{22cm}{0.5cm}{0cm}{1cm}{1cm}
\renewcommand{\baselinestretch}{1.3}
\setcounter{MaxMatrixCols}{10}

\begin{document}
%%%%%%%%%%%%%%%%%%%%%%%%%%%%%%%%%%%%%%%%%%%%%%%%%%%%%%%%%%%%%%%%%%%%%%%%%%%%%%%%%%%%%%%%%%%


6 A bank branch is responsible for cash replenishments of two cash machines (or
ATMs) near its premises. It is known that the number of customers withdrawing from
the first and the second cash machine follow Poisson distributions at the rate of 20
and 50 per day, respectively.
Withdrawals of cash per customer from the two cash machines follow normal
distributions with mean $1,500 and $1,000 per customer and standard deviation $300
and $200 per customer, respectively.
Assuming that the total cash withdrawals from each of the cash machines
independently follow compound Poisson distributions:
\item Determine the probability that combined cash withdrawal per customer from
both the cash machines is less than $1,400 per customer. [6]
\item Calculate the mean and standard deviation of the combined cash withdrawal
per day from both the cash machines. 
\end{enumerate}
%%%%%%%%%%%%%%%%%%%%%%%%%%%%%%%%%%%%%%%%%%%%%%%%%%%%%%%%%%%%%%%%%%%%%%%%%%%%%%%%%%%%%%%%%%%%%%%%%%%

\newpage

(i)
X1 and X2 be the random variables representing cash withdrawal per customer from the two ATM vestibules respectively 
‘X’ be the random variable representing combined cash withdrawal per customer 
Let lambda1 and lambda2 be the parameters of the Compound Poisson distributions 
As per page 13 of Unit 2 of the core reading, sum of 2 compound Poisson distributions follows a Compound Poisson distribution with
Parameter lambda_c= lambda1 + lambda2 
And the cdf of the combined withdrawn amount per customer is;
P (X < 1400) = 1/ lambda_c *{lambda * P (X1 < 1400) + lambda2 * P ( X2 < 1400) }

= 1/70{20* phi((1400-1500)/300) +50* phi((1400-1000)/200) } 
=0.80359 
(ii)
Let S1 and S2 be the random variables representing the 2 compound distributions from the two ATM vestibules respectively 
Let A= S1+ S2 
Exp(A) = lambda_c * EXP(X) 
Variance(A) = lambda_c * (Var(X) + (EXP(X))^2) 
MGF (X) = EXP(e^tx) = 1/ lambda_c *{lambda1 * MGF (X1) + lambda2 * MGF (X2) } 
Deriving once and taking t= 0 gives EXP(X) = 1/ lambda_c *{lambda1 * EXP (X1) + lambda2 * EXP (X2) } 
This question was reasonably well answered. Following recent examination sessions where Time Series questions in paper A have been very poorly answered, the examiners were pleased to see improvement in this area.
Parts \item and \item were well answered and are straightforward applications of the Core Reading on ARIMA() models.
Part \item was less well answered and as with comments on earlier questions, a major differentiator between stronger and weaker answers was the clarity of structure in the solution set out.
\newline
Deriving TWICE and taking t= 0 = EXP(X^2) = 1/ lambda_c *{lambda1 * EXP (X1^2) + lambda2 * EXP (X2^2) } 
Therefore;
Exp(A) = lambda_c * EXP(X) = 20* 1500+50*1000=$80000 
Variance(A) = lambda_c * EXP(X^2)
= 20* ( 1500^2+ 300^2) +50*(1000^2+ 200^2)
= $ 98,800,000 
S.D (A) = $ 9939.82 
%%%%%%%%%%%%%%%%%%%%%%%%%%%%%%%%%%%%%%%%%%%%%%%%%%%%%%%%%%%%%%%%%%%%%5
\newpage

[Total 14]
Alternative solution
using the formula on p.16 of the Core Reading
E[A] = E[S1 + S2]
= E[S1] + E[S2]
= E[X] * E[N] + E[Y] * E[M]
= 1500*20 + 1000*50
= 80,000
Var[A] = Var[S1+S2] = Var[S1] + Var[S2]
= 20* E[X^2] + 50* E[Y^2]
= 20 * (1500^2 + 300^2) + 50 * (1000^2 + 200^2)
= 98,800,000
so standard deviation = sqrt(98800000) = 9939.82
