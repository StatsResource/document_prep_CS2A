%%%%%%%%%%%%%%%%%%%%%%%%%%%%%%%%%%%%%%%%%%%%%%%%%%%%%%%%%%%%%%%%%%%%%%%%%%%%%%%%%%%%%%%%%%%
% HEADER
\documentclass[a4paper,12pt]{article}
\usepackage{eurosym}
\usepackage{vmargin}
\usepackage{amsmath}
\usepackage{graphics}
\usepackage{epsfig}
\usepackage{enumerate}
\usepackage{multicol}
\usepackage{subfigure}
\usepackage{fancyhdr}
\usepackage{listings}
\usepackage{framed}
\usepackage{graphicx}
\usepackage{amsmath}
\usepackage{chngpage}
\usepackage{vmargin}
\setmargins{2.0cm}{2.5cm}{16 cm}{22cm}{0.5cm}{0cm}{1cm}{1cm}
\renewcommand{\baselinestretch}{1.3}
\setcounter{MaxMatrixCols}{10}

\begin{document}
%%%%%%%%%%%%%%%%%%%%%%%%%%%%%%%%%%%%%%%%%%%%%%%%%%%%%%%%%%%%%%%%%%%%%%%%%%%%%%%%%%%%%%%%%%%


2 Let Xn be a sequence of independent and identically distributed random variables.

\begin{enumerate}[(a)]
\item Demonstrate that the distribution function of the Xn, F, satisfies:
􀵫𝐹􁈺\beta􀯡𝑥 􀵅 α􀯡􁈻􀵯􀯡 􀵌 􁉆1 􀵆
1
𝑛
􀵬1 􀵅
𝐶􁈺𝑥 􀵆 𝐴􁈻
𝐵
􀵰
􀬿􀬵/􀮼
􁉇
􀯡
over the range of possible values of the Xn, in the following cases:
(a) Xn is uniformly distributed on [0,1], αn = 1 − 1/n, \betan = 1/n, A = 0, B = 1
and C = −1.
(b) Xn has the two-parameter Pareto distribution with parameters δ and \lambda,
αn = \lambda(n1/δ − 1), \betan = n1/δ, A = 0, B = \lambda/δ and C = 1/δ.
[6]
\item Explain the significance of your results in part \item by considering the limiting
behaviour of the right-hand side as n → ∞. 

\end{enumerate}
%%%%%%%%%%%%%%%%%%%%%%%%%%%%%%%%%%%%%%%%%%%%%%%%%%%%%%%%%%%%%%%%%%%%%%%%%%%%%%%%%%%%%%%%%%%%%%%%%%%

\newpage


Q2
(i)(a)
(F(beta_n * x + alpha_n)) ^ n
= (beta_n * x + alpha_n) ^ n 
= (x / n + 1 – 1 / n) ^ n 
= (1 – 1 / n * (1 – x)) ^ n 
= (1 – 1 / n * (1 + C * (x – A) / B) ^ -(1 / C)) ^ n 
(i)(b)
(F(beta_n * x + alpha_n)) ^ n
= (1 – (lambda / (lambda + beta_n * x + alpha_n)) ^ delta) ^ n 
= (1 – ((lambda + beta_n * x + alpha_n) / lambda) ^ -delta) ^ n 
= (1 – ((lambda * n ^ (1 / delta) + n ^ (1 / delta) * x) / lambda) ^ -delta) ^ n 
= (1 – 1 / n * (1 + x / lambda) ^ -delta) ^ n 
= (1 – 1 / n * (1 + C * delta * x / lambda) ^ -(1 / C)) ^ n 
= (1 – 1 / n * (1 + C * (x – A) / B) ^ -(1 / C)) ^ n 
(For both (a) and (b) the candidate can obtain marks either by starting with LHS of the equations in the question and substituting in the CDF of the uniform / pareto to obtain the RHS or they can start with the RHS and apply the values of A,B,C to obtain the uniform / pareto CDFs. In either case at least two lines of working are required beyond the initial substitution to “demonstrate” as the question asks)
(ii)
The LHS is the probability that the block maximum of n observations is less than or equal to beta_n * x + alpha_n 
This question was generally well-answered.
There are a number of correct routes to solutions for parts \item and \item and all are able to receive full credit. 
The Gompertz calculations are all quite straightforward. Most candidates completed these in a spreadsheet and then pasted them into their answer document with the addition of appropriate explanations and assumptions.
Where candidates were not awarded many marks,, the most common reason was not reading the question carefully.
\newline

since the block maximum is less than or equal to beta_n * x + alpha_n if and only if all the observations are less than or equal to beta_n * x + alpha_n 
As n tends to infinity, the RHS tends to exp(-1 / n * (1 + C * (x – A) / B) ^ -(1 / C)) 
which is the distribution function of the generalised extreme value (GEV) distribution 
We have therefore shown that the distribution of a linear function of the block maximum approaches a GEV distribution as n tends to infinity 
we can use (1+x/n)^n tends to exp(x) as n tends to infinity 
In part (i)(a), the value of C is negative 
which indicates that the GEV distribution is of Weibull type 
This is as expected since the uniform distribution has a finite upper limit 
In part (i)(b), the value of C is positive 
which indicates that the GEV distribution is of Fréchet type 
This is as expected since the Pareto distribution has a heavy tail 
[Marks available 9, maximum 8]
(½ mark available for other generalised sensible comments, and 1 mark available for other GEV related comments)

\end{document}