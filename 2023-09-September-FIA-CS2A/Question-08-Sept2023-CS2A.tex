%%%%%%%%%%%%%%%%%%%%%%%%%%%%%%%%%%%%%%%%%%%%%%%%%%%%%%%%%%%%%%%%%%%%%%%%%%%%%%%%%%%%%%%%%%%
% HEADER
\documentclass[a4paper,12pt]{article}
\usepackage{eurosym}
\usepackage{vmargin}
\usepackage{amsmath}
\usepackage{graphics}
\usepackage{epsfig}
\usepackage{enumerate}
\usepackage{multicol}
\usepackage{subfigure}
\usepackage{fancyhdr}
\usepackage{listings}
\usepackage{framed}
\usepackage{graphicx}
\usepackage{amsmath}
\usepackage{chngpage}
\usepackage{vmargin}
\setmargins{2.0cm}{2.5cm}{16 cm}{22cm}{0.5cm}{0cm}{1cm}{1cm}
\renewcommand{\baselinestretch}{1.3}
\setcounter{MaxMatrixCols}{10}

\begin{document}
%%%%%%%%%%%%%%%%%%%%%%%%%%%%%%%%%%%%%%%%%%%%%%%%%%%%%%%%%%%%%%%%%%%%%%%%%%%%%%%%%%%%%%%%%%%

8 An insurer has incurred 10,000 claims under a portfolio of home insurance policies.
These claims have a mean size of £2,000 and a standard deviation of £800. One
hundred of these claims have exceeded the excess of loss limit on a reinsurance policy
that the insurer has in place.

\begin{enumerate}[(a)]
\item Using a Lognormal distribution, estimate the excess of loss limit on this
reinsurance policy. State your assumptions and show all your working clearly.
\item Calculate the number of claims that would be expected to be less than £1,000.

It has been proposed by the insurance regulator that statutory solvency calculations
should be based on modelling claims using a suitable Normal distribution.
\item Comment on the appropriateness of using a Normal distribution under various
conditions. (You may use the information about the portfolio of policies above
to illustrate your answer.) 
(iv) Comment briefly on which of the following alternative distributions should be
considered, in addition to the Lognormal distribution, when fitting a suitable
model to these claims, given the histogram showing the claims data in the
figure above, and how you may decide which distribution to include in your
final model.
 gamma
 exponential
 Weibull.
\end{enumerate}
%%%%%%%%%%%%%%%%%%%%%%%%%%%%%%%%%%%%%%%%%%%%%%%%%%%%%%%%%%%%%%%%%%%%%%%%%%%%%%%%%%%%%%%%%%%%%%%%%%%

\newpage

Q8
(i)
mean(y) = exp[mu+0.5*Sigma^2 ] - - (Eqn 1) 
Var(y) = exp[2*mu+ Sigma^2] [exp(Sigma^2)-1] - - (Eqn 2) 
squaring Eqn 1
mean(y)^2 = exp[2*mu+Sigma^2 ] - - (Eqn 3) 
dividing Eqn 2 by Eqn 3
var(y)/ mean(y)^2 = [exp(Sigma^2)-1] 
Hence:
Sigma = (log((sd(y)/mean(y))^2+1))^0.5 = 0.3852 
Mu = log(mean(y))-0.5*sigma^2 = 7.5267 
P(X>x) = 1-psi(ln(x)-mu/sigma)=0.01 gives x = 4550.2 
Using Invpsi(0.99) = 2.33 
(ii)
P(X<1000) = psi(ln(1000)-mu/sigma)= 5.41%
So 541 claims 
(iii)
The probability of very large claims may be significantly underestimated 
leading to potential solvency issues 
This is particularly the case for long, fat-talied distributions (leptokurtic) 
The distribution gives the theoretical possibility of negative claims 
It may be suitable under some conditions 
where the claims distribution is not skewed 
and has thin-tails 
%%%%%%%%%%%%%%%%%%%%%%%%%%%%%%%%%%%%%%%%%%%%%%%%%%%%%%%%%%%%%%%%%%%%%%%%%%%%%%%%%%%%%
It should be left to the insurer’s judgement.  [Marks available 5, maximum 3]
This question was very well answered. In terms of the average mark as a percentage of the available marks, this question was the best answered on the paper. Throughout this session in both A and B papers, candidates have often done best in survival models questions.
In part \item again there is a new presentation of a familiar topic (Kaplan Meier) and once again the best answers showed a clear structure: starting with the 1-lambda terms implied by the survival function, moving to derive lambda and finally considering censoring.
In part \item a large range of comments about censoring were given credit.
\newline
(iv)
Weibull is potential candidate distribution 
and Gamma is potential candidate distribution 
both can model skewed observation data, 
and are non-negative 
Exponential will not be suitable, 
as it is a decreasing function of x.  [Marks available 4, maximum 3]
Decision criteria:
Use AIC/BIC scores 
or calculate the (log) Likelihood 
or carry out a Chi squared test 
or use QQ plots 
May apply Extreme Value Thery to test the tails 
may depend on the model used in previous years 
or on what is typically used by insurers 


\nend{document}