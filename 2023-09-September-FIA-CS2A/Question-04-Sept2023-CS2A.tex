%%%%%%%%%%%%%%%%%%%%%%%%%%%%%%%%%%%%%%%%%%%%%%%%%%%%%%%%%%%%%%%%%%%%%%%%%%%%%%%%%%%%%%%%%%%
% HEADER
\documentclass[a4paper,12pt]{article}
\usepackage{eurosym}
\usepackage{vmargin}
\usepackage{amsmath}
\usepackage{graphics}
\usepackage{epsfig}
\usepackage{enumerate}
\usepackage{multicol}
\usepackage{subfigure}
\usepackage{fancyhdr}
\usepackage{listings}
\usepackage{framed}
\usepackage{graphicx}
\usepackage{amsmath}
\usepackage{chngpage}
\usepackage{vmargin}
\setmargins{2.0cm}{2.5cm}{16 cm}{22cm}{0.5cm}{0cm}{1cm}{1cm}
\renewcommand{\baselinestretch}{1.3}
\setcounter{MaxMatrixCols}{10}

\begin{document}
%%%%%%%%%%%%%%%%%%%%%%%%%%%%%%%%%%%%%%%%%%%%%%%%%%%%%%%%%%%%%%%%%%%%%%%%%%%%%%%%%%%%%%%%%%%


4 A teacher is looking for ways to cluster her pupils into homogenous groups and
decides to set up an experiment as follows.
Each day, every pupil is given a task and scores ‘+2’ if the task is successfully
completed by the end of the day and ‘−2’ otherwise. This experiment runs for n days
(n > 1) and at the end pupils are grouped based on their total score over n days.
Let us consider a pupil in the class who has a 50% chance of successfully completing
the task each day, and we assume that their performance from day to day is
independent. We denote by Zt this pupil’s score at the end of day t, and by At their
total score from day 1 to day t.
\item Determine the expected value and variance of At. 
\item Determine the probability that An = k where k is an integer, writing your
answer as a function of n and k. 
At the end of the experiment, a pupil will be classified as ‘borderline’ if their total
score, An, is zero.
\item Determine Pr(A1 = 2 | An = 0) that is the conditional probability that the pupil
scored +2 on the first day given they finish with a borderline classification.
\end{enumerate}
%%%%%%%%%%%%%%%%%%%%%%%%%%%%%%%%%%%%%%%%%%%%%%%%%%%%%%%%%%%%%%%%%%%%%%%%%%%%%%%%%%%%%%%%%%%%%%%%%%%

\newpage

(i)
A_t = sum(i=1 to t) Z_i 
E[A_t] = sum(i=1 to n) E[Z_i] = 0 
Var(A_t) = sum(i=1 to n) Var(Z_i)
= sum(i=1 to t) { 2^2 x 0.5 + 2^2 x 0.5) = 4t 
(ii)
There are 2^n paths the score may take and each one has equal probability 
In order for A_n = k there must be non-negative integers p and m
where p is the number of ‘+2’ and m the number of ‘-2’ such that
p + m = n, and
This question was generally well answered and is a relatively straightforward application of stochastic processes. The matrix approach, correctly applied, provides the most straightforward route to answering the whole question whereas the combination of individual direct-route probability calculations was more prone to error.
\newline
2(p – m) = k 
that is
p = (2n + k)/4
m = (2n – k)/4 
The number of ways to arrange these p and m results is
􁉀𝑝𝑝+𝑚𝑚𝑝𝑝􁉁 = 􁉀𝑛𝑛(2𝑛𝑛+𝑘𝑘)/4􁉁 
Hence
Pr(A_n = k) = 􁉀𝑛𝑛(2𝑛𝑛+𝑘𝑘)/4􁉁/2𝑛𝑛 
(iii)
Pr(A_1 = 2 | A_n = 0) = Pr(𝐴𝐴𝑛𝑛=0 |𝐴𝐴1=2) 𝑥𝑥 Pr (𝐴𝐴1=2)Pr(𝐴𝐴𝑛𝑛=0) 
now Pr(A_1 = 2) = ½ and 
Pr(A_n = 0) = 􀵬𝑛𝑛𝑛𝑛2􀵰/2𝑛𝑛 
for Pr(A_n = 0 | A_1 = 2) there are 2^(n-1) paths the score may take after A_1 = 2
we need non-negative p’ and m’ where p’ is the number of +2 and m’ the number of -2
then
p’ + m’ = n -1
m’ – p’ = 1
so
m’ = n/2 and p’ = n/2 – 1 
The number of ways to arrange these p’ and m’ results is
􁉀𝑝𝑝′+𝑚𝑚′𝑝𝑝′􁉁 = 􀵬𝑛𝑛−1𝑛𝑛2−1􀵰 
so
Pr(A_n = 0 | A_1 = 2) = 12 𝑥𝑥 􁉆𝑛𝑛−1𝑛𝑛2−1
􁉇/
2
𝑛𝑛−
1
􁉆𝑛𝑛𝑛𝑛2
􁉇/2𝑛𝑛 = 􁉆𝑛𝑛−1𝑛𝑛2−1
􁉇􁉆𝑛𝑛𝑛𝑛2􁉇 
= ½ n /n = 0.5 
[Total 11]
Alternatively full marks are available for a full reasoning out of 0.5 in words: This reasoning would include the following or similar,
• the first outcome is either a pass or a fail.
• the number of routes to a final score of 0 at time n if the first outcome is pass,
• is the same as the number of routes to a final score of 0 at time n if the first is a fail.
• due to symmetry of the success / fail pathways.
• the success / failure probability for the first outcome is ½.
• therefore the required probability must be 0.5.
\newline
