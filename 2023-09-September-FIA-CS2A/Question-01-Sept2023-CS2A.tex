%%%%%%%%%%%%%%%%%%%%%%%%%%%%%%%%%%%%%%%%%%%%%%%%%%%%%%%%%%%%%%%%%%%%%%%%%%%%%%%%%%%%%%%%%%%
% HEADER
\documentclass[a4paper,12pt]{article}
\usepackage{eurosym}
\usepackage{vmargin}
\usepackage{amsmath}
\usepackage{graphics}
\usepackage{epsfig}
\usepackage{enumerate}
\usepackage{multicol}
\usepackage{subfigure}
\usepackage{fancyhdr}
\usepackage{listings}
\usepackage{framed}
\usepackage{graphicx}
\usepackage{amsmath}
\usepackage{chngpage}
\usepackage{vmargin}
\setmargins{2.0cm}{2.5cm}{16 cm}{22cm}{0.5cm}{0cm}{1cm}{1cm}
\renewcommand{\baselinestretch}{1.3}
\setcounter{MaxMatrixCols}{10}

\begin{document}
%%%%%%%%%%%%%%%%%%%%%%%%%%%%%%%%%%%%%%%%%%%%%%%%%%%%%%%%%%%%%%%%%%%%%%%%%%%%%%%%%%%%%%%%%%%

1 A leap year is a calendar year which has 366 days including 29 February. This occurs
once every 4 years and the most recent leap year was 2020. A group of people who
were born on 29 February in different leap years meet for dinner once every 4 years
when it is a leap year to celebrate their unusual birthdays. The number of people at the
dinner on 29 February 2020 was recorded by their year of birth as follows:
Year of birth 1948 1952 1956
Number attending 3 8 18
One of the group decides to estimate the survival of group members using Gompertz
Law with the force of mortality at age x given by:
μ􀯫 􀵌 0.0045 􁈺1.0004􁈻􀯫
\item Comment on the choice of formula for the force of mortality. 
\item Calculate the expected number of people that will be at the next dinner stating
any assumptions you make. [5]
A new member, born on 29 February 1960, asks to join the group.
\item Calculate the expected cost of dinners for this member up to and including
29 February 2040 if each dinner costs £60 per person and ignoring interest
and inflation. [6]
[Total 13]

Q1
(i)
Gompertz Law is a relatively simple expression for the force of mortality 
It has been found to work well at older ages 
In 2020 this group were aged 64 to 72 
A Makeham form with a non-age related element might be slightly better 
The formula contains and exponential element to mortality 
The value of c in this Gompertz expression is relatively small 
Hence mortality increases relatively slowly with age 
[Marks available 3½, maximum 2]
(ii)
assuming that the force of mortality is constant for each year of age 
we can find one year survival probabilities by exp(-mu_x) 
and four year survival probabilities by chaining together for 4 years 
then expected number at the next dinner
= 3 * 4_p_72 + 8 * 4_p_68 + 18 * 4_p_64 
The mu_x and p_x calculations are:
age
mu_x
exp(-mu_x)
64
0.004617
0.995394
65
0.004619
0.9953921
66
0.00462
0.9953903
67
0.004622
0.9953885
68
0.004624
0.9953866
69
0.004626
0.9953848
70
0.004628
0.9953829
71
0.00463
0.9953811
72
0.004631
0.9953793
73
0.004633
0.9953774
74
0.004635
0.9953756
75
0.004637
0.9953737

giving expected number
= 18* 0.981692 + 8 * 0.981663 + 3* 0.981634
= 28.47 people 
Alternative solution using the formula for t_p_x in the Gompertz found in the Core Reading (the formula can be stated and does not need to be derived)
Using the formula on page 32 of the Core Reading:
4p72 = g ^(c^72 ( c^4-1))
where g = exp(-0.0045 / log(1.0004)) = 0.000012978
and c = 1.0004
\newline
So:
4p72 = 0.9816302
Similarly:
4p68 = g ^(c^72 ( c^4-1)) = 0.9816593
4p64 = g ^(c^72 ( c^4-1)) = 0.98168
So expected number is: 0.9816302 * 3 + 0.9816593 * 8 + 0.9816883 * 18
= 28.46855
(full marks are available for other alternative approaches correctly evaluated)
(iii)
the new member is age 60 in 2020
there are dinners in 2024 / 28 / 32 / 36 / 40 
therefore expected cost
= 60 * (4_p_60 + 8_p_60 + 12_p_60 + 16_p_60 + 20_p_60 ) 
again assuming constant force of mortality during each year
and that the same Gompertz formula can be applied to the new entrant 
the mu_x and p_x are given by
age
mu_x
exp(-mu_x)
60
0.004609
0.9954013
61
0.004611
0.9953995
62
0.004613
0.9953977
63
0.004615
0.9953958
64
0.004617
0.995394
65
0.004619
0.9953921
66
0.00462
0.9953903
67
0.004622
0.9953885
68
0.004624
0.9953866
69
0.004626
0.9953848
70
0.004628
0.9953829
71
0.00463
0.9953811
72
0.004631
0.9953793
73
0.004633
0.9953774
74
0.004635
0.9953756
75
0.004637
0.9953737
76
0.004639
0.9953719
77
0.004641
0.99537
78
0.004643
0.9953682
79
0.004644
0.9953663

then
4_p_60 = 0.981721
4_p_64 = 0.981692
\newline
4_p_68 = 0.981663
4_p_72 = 0.981634
4_p_76 = 0.981605
expected cost = 60 * (0.981721 + (0.981721)(0.981692) +
(0.981721)(0.981692)(0.981663) + (0.981721)(0.981692)(0.981663)(0.981634)
+ (0.981721)(0.981692)(0.981663)(0.981634)(0.981605) 
= 60 * 4.731859 = £283.91 
[Total 13]
