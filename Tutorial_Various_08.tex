

%--------------------------------------------------------------------- %
%Question 4 

% Accuracy Precsion and Recall (7 Marks)
% Two Sample Hypothesis test for Sample Proportions (8 Marks)
% Confidence Interval (5 Marks)

%--------------------------------------------------------------------- %
\newpage
\section*{Question 4 [25 marks]}
%\subsection*{Question 4. (20 marks) }

\begin{itemize}
\item[(a)] \textbf{\textit{Binary Classification (6 Marks)}}\\
For following binary classification outcome table, calculate the following appraisal metrics.
\begin{itemize}	
\item 	accuracy;
\item 	recall;
\item 	precision;
\item  	F-measure.
\end{itemize}	

\begin{center}
\begin{tabular}{|c|c|c|}
\hline  & \phantom{spa}Predict Negative\phantom{spa} & \phantom{spa}Predict Positive\phantom{spa} \\ 
\hline\phantom{spa} Observed Negative \phantom{spa}&	9530	&	10	\\ 
\hline \phantom{spa}Observed Positive\phantom{spa} & 	300	&	160	\\ 
\hline 
\end{tabular} 
\end{center}

\begin{itemize}	
\item   Explain why the F-measure is considered a more informative measure of performance than the Accuracy score.

\end{itemize}
\item[(b)] \textbf{\textit{Inference Procedures (10 Marks)}}\\
Two IT training companies, \textit{XtraTech} and \textit{YourSkills}, offer an exam preparation course for a well-known computer industry certification. A study was carried out to compare the results from the most recent group of students from both companies.
\begin{itemize}
\item[$\bullet$]30 students from the \textit{XtraTech} course have completed the test. The average score for these students was 910 marks with a standard deviation of 48 marks.

\item[$\bullet$]25 students from the \textit{YourSkills} course have completed the test. Their average score was 950 marks with a standard deviation of 42 marks.
\end{itemize}

Test the hypothesis that the both sets of students perform equally well on average. You may use a significance level of 5\%. You may assume that both samples are normally distributed and have equal variance.
%\end{itemize}
\begin{itemize}
\item  Formally state the null and alternative hypotheses for this procedure.
\item  Compute the point estimate for the difference in means of the results from both courses.
\item  Compute the appropriate value for standard error for this test. Clearly show your workings.
\item   Compute the test statistic.
\item   What is your conclusion for this procedure?
\end{itemize}
{
\normalsize
\textit{\textbf{Please turn over for the remaining sections of Question 4.}}
}
\newpage
\item[(c)] \textbf{\textit{Inference Procedures (9 Marks)}}\\A study finds that $42\%$ of IT users out of a random sample of 450 in a large
community preferred one web browser to all others. In another large community, $34\%$ of IT users out of a random sample of 350 prefer the same web browser.

\begin{itemize}
\item  Compute the point estimate for the difference in proportions of IT users who prefer this particular web browser.
\item  Compute a 95\% confidence interval for this difference in proportions.
\item  Based on this confidence interval, test the hypothesis that the proportion of IT users using this web browser is the same for both communities. State your null and alternative hypotheses clearly.
\end{itemize}

\end{itemize}

%------------------------------------------------------------------- %
% Question 5
% Huffman Coding
%------------------------------------------------------------------- %
\newpage
\subsection*{Question 5 [25 marks] }
\begin{itemize}
\item[(a)] \textbf{\textit{Huffman Coding (8 Marks)}}\\
A discrete memoryless source $X$ has five symbols $\{x_1,x_2,x_3,x_4,x_5\}$ with probabilities $P(x_1) = 0.45$ , $P(x_2) = 0.20$, $P(x_3) = 0.16$, $P(x_4) = 0.14$ and $P(x_5) = 0.05$.

\begin{itemize}
\item (5 Marks) Construct a Huffman code for X.
\item  Calculate the efficiency of the code.
%\item[(iii)] (1 marks) Calculate the redundancy of the code.
\end{itemize}
\bigskip
\item[(b)] 

{
\normalsize
\textit{\textbf{Please turn over for the remaining sections of Question 5.}}
}
%--------------------------------------------------------%
%  Question 5

%  Entropy
%  Huffman Coding
%--------------------------------------------------------%
\newpage
%\end{itemize}
\begin{itemize}
\item[(c)]


\item[(d)] \textbf{\textit{Communication Channels (4 Marks)}}\\
The input source to a noisy communication channel is a random variable X over the
four symbols $\{a, b, c, d\}$. The output from this channel is a random variable Y over these same
four symbols. \\
\vspace{0.3cm}
\noindent 
The joint distribution of these two random variables is as follows:\\ \bigskip

\begin{center}
\begin{tabular}{|c|c|c|c|c|}
\hline
&x=a& x=b & x=c & x=d \\ \hline
y=a &1/8 &0 &0 & 0 \\ \hline
y=b &0 & 1/4& 1/8& 0 \\ \hline
y=c & 0&1/16 & 1/8 & 0\\ \hline
y=d & 1/16& 0& 0 & 1/4\\ \hline
\end{tabular}
\end{center}

\begin{itemize}
\item  Write down the marginal distribution for $X$ and compute the marginal entropy $H(X)$.
\item  Write down the marginal distribution for $Y$ and compute the marginal entropy $H(Y )$.
%\item[(iii)]  What is the joint entropy $H(X, Y ) $ of the two random variables?
%\item[(iv)] (4 marks) What is the conditional entropy $H(Y|X)$?
%\item[(v)] (3 marks) What is the conditional entropy $H(X|Y)$?
%\item[(vi)] (3 marks) What is the mutual information $I(X;Y)$ between the two random variables?
\end{itemize}

\end{itemize}
\newpage


