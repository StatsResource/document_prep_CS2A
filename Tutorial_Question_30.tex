
\newpage
\section{Tutorial C - Probability}	
\subsection*{Question 1}
Assume that there are three different routes to get to a particular location: $R_1$, $R_2$ and $R_3$. You take $R_1$ 75\% of the time, $R_2$ 20\% of the time and $R_3$ the rest of the time. If you take $R_1$, there is a 90\% chance that you will be on time; if you take $R_2$, there is a 50\% chance that you will be on time and, if you take $R_3$, there is a 70\% chance that you will be on time. \\[0.1cm]
Let $T$ represent on time.\\[-0.2cm]

{\bf(a)} If $T$ represents ``on time'', what notation would we use for ``late''? \quad {\bf(b)} What is the value of $\Pr(R_1 \cap R_2)$? \quad {\bf(c)} Calculate the probability of being on time. \quad {\bf(d)} \emph{Given that} you \emph{are} on time, calculate the probabilities of having used each of the routes. \quad {\bf(e)} Given that you are late, what is the probability that you used $R_1$?






\subsection*{Question 5}
In your favourite RPG game, let's assume that in selecting your character there are 5 character classes and 2 genders. Let's also assume there are 3 levels of difficulty for this game.\\[-0.2cm]

{\bf(a)} How many possible ways can you play this game? \quad {\bf(b)} What if you always choose the ``warrior'' class? \quad {\bf(c)} What if you always choose a female character? \quad {\bf(d)} What if you always play on the highest difficulty setting? \quad {\bf(e)} Let's assume the game has a two-player mode. How many possible ways can you play this game? (hint: you cannot play the game on different difficulty levels). {\bf(f)} What if your friend chooses a different character class to you?



\subsection*{Question 6}
Assume that you are going to an exam and you can only bring 3 items. You have the following items: $\{\text{mobile phone, } \text{pen, } \text{ruler, } \text{calculator, } \text{laptop, } \text{apple} \}$.\\[-0.2cm]

\begin{enumerate}[(i)]
\item In order to make your decision, you first \emph{arrange} these 6 items on your desk. How many possible arrangements are there? \item How many possible groups of three items can you bring with you? \item What if you decide that the pen is essential? \item What if the pen is essential and you also decide that you won't bring an apple or a laptop?
\end{enumerate}


\subsection*{Question 7}
A team of 5 people is required to perform a particular task. We are selecting from a group of 7 women and 3 men.\\[0.2cm]
How many selections are there:\\[-0.2cm]

{\bf(a)} Altogether? \quad {\bf(b)} If one of the men is an expert and must be on the team? \quad {\bf(c)} If two of the individuals do not get along and cannot be on the team together? \quad {\bf(d)} If the group must contain 3 women and 2 men? {\bf(e)} If the group must contain more women than men? \quad {\bf(f)} If the group must contain more men than women?


\section{Tutorial G - Normal Distribution}
\subsection*{Question 1}
Assume that a character in a game is programmed to have an attack power according to $X \sim \text{Normal}(\mu=40,\sigma=3)$.\\[-0.2cm]

{\bf(a)} What is the probability that the attack is greater than 45? \quad {\bf(b)} What is the probability that the attack is between 32 and 42? \quad {\bf(c)} Let $X_1$ and $X_2$ be the first and second attacks. What is the probability that the \emph{sum} of these two attacks is greater than 85 units? \quad \\{\bf(d)} Calculate 99\% limits for the sum of two attacks.  \quad {\bf(e)} What is the probability that the \emph{difference} in attacks is more than 5 units? Note that attack 2 can be 5 units more than attack 1 or attack 1 can be 5 units more than attack 2, i.e., $\Pr(|D|>5)=\Pr(D<-5) + \Pr(D>5)$.



\subsection*{Question 2}

A character in a game deals a standard attack 75\% of the time and a critical attack the rest of the time (call these events $S$ and $S^c$). Given that it is a standard attack, the attack power is $X\,|\,S \sim \text{Normal}(\mu=40,\sigma=3)$. When the character deals a critical attack, a random fluctuation is added to this according to a $\text{Normal}(\mu=5,\sigma=1)$ distribution.\\[-0.2cm]

{\bf(a)} What is the distribution of $X\,|\,S^c$? \quad {\bf(b)} Calculate $\Pr(X<43\,|\,S)$ and $\Pr(X<43\,|\,S^c)$. \quad {\bf(c)} Calculate $\Pr(X<43)$. \,\, (hint: law of total probability) \quad {\bf(d)} If the character deals less than 43 damage points, what is the probability that the attack was a critical attack?



\subsection*{Question 3}
The income of a technician (in thousands) is $X_1 \sim \text{Normal}(\mu=30,\sigma=2)$. The income of an engineer is $X_2 \sim \text{Normal}(\mu=40,\sigma=3.5)$. \\[-0.2cm]

{\bf(a)} Calculate the probability that an engineer earns more than a technician. \quad {\bf(b)} Calculate 90\% limits for the difference in their income. \quad {\bf(c)} For a group of 25 technicians, calculate the probability that the average wage is less than 30500, i.e., $\Pr(\,\overline{\!X}_1 < 30.5)$. \quad {\bf(d)} In a group of 10 engineers, what is the probability that \emph{at least two} of them earn more than 45000? (hint: binomial with $p = \Pr(X_2 > 45)$) \quad {\bf(e)} For a sample of 30 technicians and 35 engineers, calculate the 80\% limits for the difference in their average wages.


\subsection*{Question 4}
Let $X \sim \text{Exponential}(\lambda=0.02)$. Calculate the following:\\[-0.2cm]

{\bf(a)} $\Pr(\,\overline{\!X} > 55)$ in a group of 100. \quad {\bf(b)} $\Pr(\,\overline{\!X} < 53)$ in a group of 40. \quad {\bf(c)} The value of $\bar x$ such that $\Pr(\,\overline{\!X} > \bar x) = 0.1$ when $n=65$. \quad {\bf(c)} The value of $n$ if $\Pr(\,\overline{\!X} < 49) = 0.1$.



\section{KB Tutorial 3}
\subsection*{Question 1}
Assume that there are three different routes to get to a particular location: $R_1$, $R_2$ and $R_3$. You take $R_1$ 75\% of the time, $R_2$ 20\% of the time and $R_3$ the rest of the time. If you take $R_1$, there is a 90\% chance that you will be on time; if you take $R_2$, there is a 50\% chance that you will be on time and, if you take $R_3$, there is a 70\% chance that you will be on time. \\[0.1cm]
Let $T$ represent on time.\\[-0.2cm]

{\bf(a)} If $T$ represents ``on time'', what notation would we use for ``late''? \quad {\bf(b)} What is the value of $\Pr(R_1 \cap R_2)$? \quad {\bf(c)} Calculate the probability of being on time. \quad {\bf(d)} \emph{Given that} you \emph{are} on time, calculate the probabilities of having used each of the routes. \quad {\bf(e)} Given that you are late, what is the probability that you used $R_1$?






\section{KB tutorial 4}

\subsection*{Question 1}
You develop a random number generater which assigns a value to the random variable $X$ according to the following probability distribution:
\begin{center}
	\begin{tabular}{|c|ccccc|}
		\hline
		&&&&&\\[-0.4cm]
		$x$ & 0.0 & 0.5 & 1.0 & 2.0 & 3.0 \\
		\hline
		&&&&&\\[-0.4cm]
		$\Pr(X=x)$ & $0.4$ & $0.2$ & $0.15$ & $0.15$ & $?$ \\[0.1cm]
		\hline
		\multicolumn{6}{c}{}\\
	\end{tabular}
\end{center}

{\bf(a)} What is value the value of $\Pr(X = 3.0)$? \quad {\bf(b)} Calculate $E(X)$ and $Sd(X)$. \quad {\bf(c)} You produce a gambling game where the player wins (in euro) the value of $X$ generated, e.g., if a $2.0$ appears, \euro{2} is won. How much should you charge for a play of this game so that that \emph{you} (the programmer) make a profit of \euro{0.10} on average per game? (i.e., the player \emph{loses} \euro{0.10} on average) \quad {\bf(d)} Using your answer to part (c), determine the probability that you make a profit when somebody plays this game. \quad {\bf(e)} If 10 people play this game, what is the probability that you make a profit 8 times?

\subsection*{Question 2}
You flip three coins. Let $X = $ ``the number of heads'' and $Y = $ ``the number of unique faces''.\\[-0.2cm]

{\bf(a)} What is the sample space for this experiment? \quad {\bf(b)} Construct the \emph{joint distribution} for $X$ and $Y$. \quad {\bf(c)} Based on this joint distribution, construct the \emph{marginal} distribution for $X$ and for $Y$. \quad {\bf(d)} Are $X$ and $Y$ independent? \quad {\bf(e)} Calculate $E(Y)$ and $Sd(Y)$. \quad {\bf(f)} Calculate $\Pr(Y=2\,|\,X=2)$ and interpret its value (compare with $\Pr(Y=2)$).


\subsection*{Question 3}
Let $X =$ ``the attack power of player 1'' and let $Y =$ ``the attack power of player 2''.\\[-0.3cm]

Let the probability distributions for $X$ and $Y$ be:
\begin{center}
	\begin{tabular}{|c|ccc|c|c|ccc|}
		\cline{1-4}\cline{6-9}
		&&&&&&&&\\[-0.4cm]
		$x$ & 0 & 100 & 300 & \qquad\qquad & $y$ & 0 & 80 & 200\\
		\cline{1-4}\cline{6-9}
		&&&&&&&&\\[-0.4cm]
		$\Pr(X=x)$ & $0.2$ & $0.75$ & $0.05$ & & $\Pr(Y=y)$ & $0.1$ & $0.6$ & $0.3$ \\[0.1cm]
		\cline{1-4}\cline{6-9}
		%\multicolumn{9}{c}{}
	\end{tabular}
\end{center}
{\footnotesize(e.g., p1 misses 20\% of the time, deals 100 points of damage 75\% of the time and performs a critical blow 5\% of the time.)}\\[-0.2cm]

{\bf(a)} What is the average attack power of each player? \quad {\bf(b)} If both players have 1000 hit-points, how many attacks does it take for player 1 to defeat player 2 and vice versa? Which player will win on average? \quad {\bf(c)} Let's now assume that player 1 uses his/her \emph{first} turn to cast a spell (and therefore does not attack in this turn). The result of the spell is that player 2 can no longer perform a critical blow, i.e., $\Pr(Y=200) = 0$, \emph{from turn two onwards}. Since setting $p(200) = 0$ leads to $\sum p(y) \ne 1$, assume that the remaining probability ($= 0.3$) is distributed evenly between $p(0)$ and $p(80)$. What is the outcome of the battle now?


\subsection*{Question 4}

You flip a coin 10 times - let $X =$ ``the number of heads''. Using the binomial probability function, calculate the following:\\[-0.2cm]

{\bf(a)} $\Pr(X = 2)$. \quad {\bf(b)} $\Pr(X = 0)$. \quad {\bf(c)}  $\Pr(X > 2)$. \quad {\bf(d)} $\Pr(X \le 3)$. \quad {\bf(e)} $\Pr(5 \le X \le 7)$.  \quad {\bf(f)} $E(X)$ and $Sd(X)$. \quad {\bf(g)} Using the binomial tables, calculate $\Pr(X \le10)$ in the case where the coin is flipped 20 times. \quad {\bf(h)} If the coin is flipped 50 times, what is $E(X)$?

\subsection*{Question 5}

Repeat Question 4 (a) - (e) but now using the binomial tables.



\subsection*{Question 6}

Let's assume that a sequence of bits (binary numbers) is transmitted and, at the other end, decoded; the decoder has a 10\% chance reading a bit incorrectly (i.e., reading a 0 as 1 or vice versa). Let $X$ be the number of errors in the sequence received (i.e., the decoded sequence). Calculate the probability that there are: \\[-0.2cm]

{\bf(a)} \emph{No} errors in a 20-bit string. \quad {\bf(b)} Less than three errors in a 10-bit string. \quad {\bf(c)} More than 10 errors in (i) a 50-bit string and (ii) a 100-bit string (hint: use tables). \quad {\bf(d)} Calculate the average number of errors in a 100-bit string. Calculate the standard deviation also.


\subsection*{Question 7}
We follow on from Question 6 but now consider the case where, to reduce the probability of error, each bit is sent \emph{three} times and then a ``majority vote'' approach is used to determine the value of each received bit. The following example explains the situation:\\[-0.5cm]
\begin{center}
	\begin{tabular}{ccccc}
		\hline
		&&&&\\[-0.3cm]
		\multirow{2}{*}{Sent} & $0$ & $1$ & $1$ & $0$ \\
		& $\overbrace{000}$ & $\overbrace{111}$ & $\overbrace{111}$ & $\overbrace{000}$ \\[0.2cm]
		\hline
		&&&&\\[-0.3cm]
		\multirow{2}{*}{Received} & $\underbrace{001}$ & $\underbrace{111}$ & $\underbrace{010}$ & $\underbrace{000}$ \\
		& $0$ & $1$ & $0$ & $0$ \\[0.2cm]
		\hline
		%\multicolumn{5}{c}{}
	\end{tabular}
\end{center}
$\Rightarrow$ there is one error in decoding the first $000$, but since the majority result is taken, this bit is correctly identified as a $0$. There are two errors in decoding the second $111$, so this bit is misread as a $0$. It is clear that a character is misread if the decoder makes \emph{two or three errors} in these blocks of three replicates.\\[-0.2cm]

{\bf(a)} Show that sending each bit 3 times reduces the error probability from 10\% to 2.8\%. \quad\\ {\bf(b)} Using this reduced value, $p=0.028$, calculate the probability that there are no errors in a 20-bit string. Compare this result to Q6(a). \quad {\bf(c)} Now assume that each bit is sent 5 times and, again, the majority vote approach is used. Calculate the probability that there are no errors in a 20-bit string in this case. %\quad {\bf(d)} Recalculate the two probabilities from part (c) using the Poisson approximation.
