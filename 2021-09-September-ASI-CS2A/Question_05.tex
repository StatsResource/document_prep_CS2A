Q. 5) \item Define an ARCH model and explain what particular properties of the model would
make it appropriate for modelling a time series asset.

Consider the time series
𝑌􀯧=0.1 +0.4Yt-1 +0.9et-1 +et
Where et is white noise process with variance σ2
\item Identify the model as an ARIMA(p,d,q) Process (1)
\item Determine whether 𝑌􀯧 is
a) A stationary process
(1)
IAI CS2A-0921
Page 6 of 8
b) An invertible process (1)
\item Calculate E(𝑌􀯧 ) and find auto co-variance function for 𝑌􀯧 
\item Determine that MA(∞) representation for 𝑌􀯧 
[17]



%%%%%%%%%%%%%%%%%%%%%%%%%%%%%%%%%%%%%%%%%%%%%%%%%%%%%%
Solution 5:
i)
An ARCH(p) model is 𝑋𝑡=𝜇+𝑒𝑡√𝛼0+Σ𝛼𝑘(𝑋𝑡−𝑘−𝜇)2𝑝𝑘=1
Where et are independent (0,1)
2 Marks for writing the equation correctly and explaining the terms
For an asset with price Zt above model is used to to model Ln(Zt/Zt-1 )
 It can be seen that a large departure in Xt-k from μ will result in Xt having a larger variance. This will then result in a large volatility for the asset price.
 This behaviour is observed in actual practice as the volatility of the price of a particular asset is much higher following a significant change in the price of the asset.
 Therefore ARCH model is more appropriate to model such asset class.
(4)
Consider the time series
𝑌𝑡=0.1 +0.4Yt-1 +0.9et-1 +et
Where et is white noise process with variance σ2
ii)
The model is ARIMA(1,0,1) if Yt is stationary
(1)
(ii)
iii)
a) The characteristic polynomial for the AR part is A(z) = 1 − 0.4z the root of which has absolute value greater than 1 so the process is stationary.
b) The characteristic polynomial for the MA part is B(z) = 1 + 0.9z the root of which has absolute value greater than 1 so the process is invertible.
(1)
(1)
(iii)
iv)
Since the process is stationary we know that E(Yt ) is equal to some constant μ independent of t.
Taking expectations on both sides of the equation defining Yt gives
E(Yt ) = 0.1 + 0.4E(Yt−1)
μ = 0.1 +0.4μ
μ = 0.1/( 1 0.4) = 0.1666666
Note that
Cov(Yt , et ) = Cov(0.1 + 0.4Yt-1 + 0.9et-1 + et , et )
=0.4Cov(Yt-1, et ) + 0.9Cov(et-1, et ) + Cov(et , et ) = 0 + 0 + σ2 = σ2
Similarly
IAI CS2A-0921
Page 5 of 8
Cov(Yt ,et-1) =0 + 0.4Cov(Yt-1, et-1) + 0.9Cov(et-1, et-1) + Cov(et ,et-1) = 0.4σ2 + 0.9σ2 + 0 = 1.3σ2
So
γ0 = Cov(Yt , Yt ) = Cov(Yt , 0.1 + 0.4Yt-1 + 0.9et-1 + et )
= 0.4γ1 + 0.9 × 1.3σ2 + σ2 = 0.4γ1 + 2.17σ2 -----------------(1)
And
γ1 = Cov(Yt-1, Yt ) = Cov(Yt-1, 0.1 + 0.4Yt-1 + 0.9et-1 + et )
= 0.4γ0 + 0.9σ2---------------------------------------------------------------------- (2)
Substituting for γ1 in (1) gives
γ0 = 0.4 × 0.4γ0 + 0.4 × 0.9σ2 + 2.17σ2 = 0.16γ0 + 2.53σ2
γ0 = (2.53 /0.84 )σt62 = 3.011905σ2
Substituting into (2) gives
γ1 = 0.4 × 3.011905σ2 + 0.9σ2 = 2.104762σ2
And in general
γs = 0.4γs−1 for s ≥ 2
So γs = 0.4s-1 × 2.104762 σ2.
(6)
(iv)
v)
We have (1−0.4𝐵)𝑌𝑡=0.1+0.9𝑒𝑡−1+𝑒𝑡
Hence 𝑌𝑡= (1−0.4𝐵)−1(0.1+0.9𝑒𝑡−1+𝑒𝑡) = Σ0.4𝑖𝐵𝑖∞𝑖=0(0.1+0.9𝑒𝑡−1+𝑒𝑡)
=0.11−0.4+0.9Σ0.4𝑖𝑒𝑡−𝑖−1+Σ0.4𝑖𝑒𝑡−𝑖∞𝑖=0∞𝑖=0 =0.16667+𝑒𝑡+1.3Σ0.4𝑖−1𝑒𝑡−𝑖∞𝑖=1
(4)
[17 Marks]

