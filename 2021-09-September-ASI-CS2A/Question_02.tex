Q. 2) \item State why it is important to divide data into homogeneous classes when undertaking
mortality investigations. 
\item List down certain reason explaining the importance of graduating the crude mortality
rates before using them in financial projections. 
\item What do you understand by central exposed to risk?

\item Specify the data needed for the exact calculation of a central exposed to risk (waiting
time) depending on age and sex. 
Q. 2)
i)
 The rate obtained using a heterogenous data may not represent the group risk adequately and hence may lead to over or underestimation of the insurance risk and hence the premium calculation
 There would be a risk of adverse selection by the group who are more exposed to risk and the product will look expensive to the less risky population
 Will be difficult to sufficiently understand the exact cause of any deviation of the actual experience against the assumption
(Max 2, 1 mark each)
(2)
ii)
 We assume that mortality rates progress smoothly with age. Therefore a crude estimate at age x carries information about the rates at adjacent ages, and graduation allows us to use this fact to “improve” the estimate at age x by smoothing.
 This reduces the sampling errors at each age.
 It is desirable that financial quantities progress smoothly with age, as irregularities are hard to justify to clients.
(Max 2, 1 mark each)
(2)
iii)
The central exposed to risk at age x, is the waiting time in a multiple-state or Poisson model.
If we let ai be the latest of the time of entry into observation, the date of attaining age label x, or the beginning of the investigation, and let bi be the earliest of death, date of losing age label x or the end of the investigation, then central exposed to risk is = Sum over all lives (bi-ai)
(1)
(1)
[2]
iv)
 record all dates of birth
 record all dates of entry into observation
 record all dates of exit from observation
 Reason for exist (cause of the cessation of observation)
(2)
[8 Marks]
