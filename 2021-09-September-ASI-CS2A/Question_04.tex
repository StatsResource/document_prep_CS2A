Q. 4) An insurance portfolio contains policies for three categories of policyholder: A, B and C.
The number of claims in a year, N, on an individual policy follows a Poisson distribution
with mean \lambda. Individual claim sizes are assumed to be exponentially distributed with mean
4 and are independent from claim to claim. The distribution of \lambda, depending on the
category of the policyholder, is
Category Value of \lambda Proportion of policyholders
A 2 0.2
B 3 0.6
C 4 0.2
If the total claim amount denoted by a policyholder in one year is S
\item Prove that E(S)=E[E(S|\lambda)] 
\item Show that E (S|\lambda)=4 \lambda and Var(S|\lambda) =32 \lambda 
\item Calculate E(S) 
\item Calculate Var(S) 

%%%%%%%%%%%%%%%%%%%%%%%%%%%%%%%%%%%%%%%%%%%%%%%%%%%%

Solution 4:
i)
Let f (s) denote the marginal probability density for S and let f (s/\lambda) denote the conditional probability density for S/ \lambda. Then
E[E(S/ \lambda)]= \Sum𝑝(\lambda𝑖)3
𝑖=1∫sf(s\lambda)ds∞0
=∫\Sum𝑝(\lambda𝑖)3
𝑖=1∞0 sf(s\lambda)ds
But \Sum𝑝(\lambda𝑖)3
𝑖=1f(s\lambda) = f(s) by definition
So
E[E(S/ \lambda)]= ∫𝑠𝑓(𝑠)𝑑𝑠=𝐸(𝑆)∞0
(2)
ii)
Using the results for compound distributions, we have:
E(S/\lambda)= E(N/\lambda)* E(X/\lambda)= E(N/\lambda)* E(X)=4\lambda
Var(S/\lambda)= E(N/\lambda)* Var(X)+ Var(N/\lambda)* E(X)2
=\lambda*16+\lambda*4*4
=32\lambda
(2)
%%%%%%%%%%%%%%%%%%%%%%%%%%%%%%%%%%%%%%%%%%%%%%%%%%%%
iii)
E(S)=E[E(S/\lambda)]= E(4\lambda)=4E(\lambda)=12
(2)
iv)
Note that E(\lambda)= 3 \lambda = and
Var(\lambda)=0.2x22 + 0.6X32 +0.2x42 -9=0.4
Var(S)=Var[E(S/\lambda)] +E[Var(S/\lambda)]
=Var(4\lambda)+E(32\lambda)
=16xVar(\lambda) +32E(\lambda)
=16x0.4+32x3
=102.4
(2)
[8 Marks]
IAI CS2A-0921
Page 4 of 8
