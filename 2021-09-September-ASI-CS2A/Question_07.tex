Q. 7) As per English Life Tables No. 17 (ELT17), number of lives L(x) at ages 30 and 35 is
given below:
Age (x) L(x)
30 98600.4
35 98193.0
Calculate the number of L(32) assuming:
\item Constant force of mortality between exact ages 30 and 35 years 
\item Uniform distribution of deaths between exact ages 30 and 35 years 

\item It is known that the force of mortality from age 35 to 40 years is double of what was
observed during age 30 to 35 years. Using the result from (\item; Calculate the probability
that a life alive at the age of 35, is still alive at age 38 
[8]


Solution 7:
i)
Let the constant force of mortality be μ.
Then we have
5p30 = exp {−∫μdx}50 = e ^ (-5μ)
It is given that 5p30 = L(35) / L(30) = 98193.0 / 98600.4 = 0.995868
Thus, e ^ (-5μ) = 0.995868
 -5 μ = ln(0.995868) = -0.00414
 μ = 0.00083
Using the value of μ calculated, L(32) = L(30) * 2p30 = L(30) * exp {−∫μdx20
= L(30) * exp (-2 * μ)
= 98600.4 * exp (-2 * 0.00083) = 98600.4 * 0.998345
= 98437.24
(4)
ii)
Under Uniform distribution of death, the number of deaths are uniform
L(32) = L(30) – 2/5 * {L(30)- L(35)}
= 98600.4 – 2/5 * (98600.4 - 98193.0)
= 98437.44
(2)
iii)
Force of mortality = 2* 0.00083 = 0.00166
3p35 = exp (-3 * 0.00166) = 0.995044
(2)
[8 Marks]
