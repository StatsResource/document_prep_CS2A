%%%%%%%%%%%%%%%%%%%%%%%%%%%%%%%%%%%%%%%%%%%%%%%%%%%%%%%%%%%%%%%%%%%%%%%%%%%%%%%%%%%%%%%%%%%
% HEADER
\documentclass[a4paper,12pt]{article}
\usepackage{eurosym}
\usepackage{vmargin}
\usepackage{amsmath}
\usepackage{graphics}
\usepackage{epsfig}
\usepackage{enumerate}
\usepackage{multicol}
\usepackage{subfigure}
\usepackage{fancyhdr}
\usepackage{listings}
\usepackage{framed}
\usepackage{graphicx}
\usepackage{amsmath}
\usepackage{chngpage}
\usepackage{vmargin}
\setmargins{2.0cm}{2.5cm}{16 cm}{22cm}{0.5cm}{0cm}{1cm}{1cm}
\renewcommand{\baselinestretch}{1.3}
\setcounter{MaxMatrixCols}{10}

\begin{document}
%%%%%%%%%%%%%%%%%%%%%%%%%%%%%%%%%%%%%%%%%%%%%%%%%%%%%%%%%%%%%%%%%%%%%%%%%%%%%%%%%%%%%%%%%%%


3 The figure below shows three vectors ax, bx and kt, which were obtained by applying
the two-factor Lee–Carter model to mortality data derived from the period 1970 to
2018 for the age range 65 to 90 years.
\item   Discuss what each of these three vectors suggest about the characteristics of
the mortality rates in the data. 

\item   Suggest a socio-economic reason for your conclusion about vector b in part \item  
above. 
\item  Calculate m70, 2018 using the output from the model (reading the values from
the plots in the figure). 
An ARIMA(0,1,0) model (discrete random walk with drift) with Normal errors
ε\sim N(0, σ2) was used to model projected k values. The drift term of the time series
model was estimated as −0.4763, with standard error 0.084, and σ2 was estimated as
0.346.
\item  Calculate a 90% confidence interval for m70, 2019 based on the value of
m70, 2018 calculated in part \item .
On analysing the past data over a larger age range (from 20 to 90 years) it was
determined that there was an increase in rates for 20–30 year olds in the 1970s due
to an illness that almost exclusively affected the younger population. The higher
mortality rates only lasted until 1980 when a cure was found and introduced to the
population.
(v) Explain how this cohort effect may manifest itself in the output from the twofactor
Lee–Carter model, fitted to ages 20–90 and calendar years 1970–2018,
in terms of a, b and k. [5]

%%%%%%%%%%%%%%%%%%%%%%%%%%%%%%%%%%%%%%%%%%%%%%%%%%%%%%%%%%%%%%%%%%%%%%%%%%%%%%
\subsection*{Solutions}

\begin{itemize}
\item
\item
\end{itemize}


Q3
\item  
From the 3 plots in Figure 1
a
The increasing values of a imply that Mortality rates increase with age \item 
Visually the plot appears linear \item 
which implies that the increase is exponential 
k
Given that the b-values are positive \item 
%-------------------------%
\medskip 
the decreasing values of k imply that mortality has been improving over the period \item 
either: Visually the plot appears somewhat linear \item 
which implies that improvement rates have been constant \item 
or: the rate of decrease appears less pronounced at the earliest and latest years
suggesting different rates of improvement over the period 
b
The increasing values of b imply that mortality improvements have been greatest
for younger ages \item 
The values appear constant in ages 65 - 70 and decrease thereafter \item 

% (½ mark for any other reasonable observation about the nature of the graphs)
% [Marks available 6, maximum 4]

\item  
Medical improvements have been greatest for younger ages
Education has had the greatest effect on younger population e.g. related to smoking advice, general health
Or any other reasonable comment 
(Award ½ mark if reasonable comment about older ages instead)
\item 
exp(-3.35 + 0.0484 * (-13)) = 0.0187 accept any 0.0185 - 0.0189 
(Award ½ mark if parameters read correctly or calculation performed correctly but not both)
\item 
projected k, in 1 years: -0.4763 1.64 * (0.0842 + 0.346)1/2 
k = -0.4763 0.9745 
k = (-1.4508, 0.4982) 
Therefore, the lower limit of the required confidence interval is
m = 0.0187 exp(0.0484 * -1.4508) = 0.0174 
The upper limit is m = 0.0187 exp(0.0484 * 0.4982) = 0.0192 
(v)
ax is a measure of the average rate at each age over the investigation period. Values
would therefore be relatively high for the 20-30 year group. A plot of ax may show
an “illness bump" 
For kt there would be a rapid decline in values around 1980, due to the rapid fall off
of deaths in the 20-30 age range following the cure being introduced to the
population. It may lead to a general underestimation of projected mortality rates if a simple linear model is adopted for projecting k. The effect will depend on the
relative weightings of deaths in that age group 
bx will show large numbers in the 20-30 year band; these characteristics will be incorrectly projected into the future (assuming the illness has been eradicated), with improvements at these ages being greatly exaggerated 
In summary projected mortality rates are likely to be too low, with the greatest effect
on the 20-30 year age band 
%-------------------------%
\medskip 
Other reasonable comments on ages 30+ 
\end{document}
% [Marks available 8, maximum 5]
% [Total 16]
% This question was generally poorly answered, particularly parts \item  , \item  and (v).
% In part \item   a large range of sensible points about a, b and k were given credit. Many candidates simply recited definitions of these parameters rather than applying those definitions to the evidence of the graphs in the question. The best answers combined a description of the plots with an understanding of the model parameters.
% In part \item  partial credit was given to a wide range of approaches to calculating a confidence interval. In particular a number of candidates derived a value for m70;2019 and then built a confidence interval for that rather than building the interval around k.
% In part (v), candidates were not given credit for discussion of the advantages and disadvantages of cohort models given one is assumed in the question.
