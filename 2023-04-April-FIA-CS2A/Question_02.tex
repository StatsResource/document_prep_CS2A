%%%%%%%%%%%%%%%%%%%%%%%%%%%%%%%%%%%%%%%%%%%%%%%%%%%%%%%%%%%%%%%%%%%%%%%%%%%%%%%%%%%%%%%%%%%
% HEADER
\documentclass[a4paper,12pt]{article}
\usepackage{eurosym}
\usepackage{vmargin}
\usepackage{amsmath}
\usepackage{graphics}
\usepackage{epsfig}
\usepackage{enumerate}
\usepackage{multicol}
\usepackage{subfigure}
\usepackage{fancyhdr}
\usepackage{listings}
\usepackage{framed}
\usepackage{graphicx}
\usepackage{amsmath}
\usepackage{chngpage}
\usepackage{vmargin}
\setmargins{2.0cm}{2.5cm}{16 cm}{22cm}{0.5cm}{0cm}{1cm}{1cm}
\renewcommand{\baselinestretch}{1.3}
\setcounter{MaxMatrixCols}{10}

\begin{document}
%%%%%%%%%%%%%%%%%%%%%%%%%%%%%%%%%%%%%%%%%%%%%%%%%%%%%%%%%%%%%%%%%%%%%%%%%%%%%%%%%%%%%%%%%%%


2 Country A has recently gone through an economic crisis. As the country makes an
attempt to recover, the Department of Finance is trying to estimate the rate of
recovery in employment. The department has decided to use a two-state
continuous-time Markov model to estimate the rate of return to employment. It has
also decided to use data from one of the previous economic recoveries for the
purpose. The two states are:
The employment data from a previous economic recovery was as follows:
 Waiting time to gain employment in the first year (in person-years): 30,000
 Waiting time to gain employment in the second year (in person-years): 22,000
 Number of people gaining employment in the first year: 5,000
 Number of people gaining employment in the second year: 7,000
It may be assumed that force of gaining employment in any 1 year is constant.
\begin{enumerate}
\item   State the likelihood function of the maximum likelihood estimator of the
transition rate defining all the terms you use. 
\item   Calculate the maximum likelihood estimate of the transition rate from the state
of being unemployed to gaining employment for each of the first 2 years. 
\item  Estimate, by stating the expression, the variance of the second year maximum
likelihood estimator. 
\item  Calculate the probability of not gaining any employment in the next 2 years.
\end{enumerate}
%%%%%%%%%%%%%%%%%%%%%%%%%%%%%%%%%%%%%%%%%%%%%%%%%%%%%%%%%%%%%%%%%%%%%%%%%%%%%%
\subsection*{Solutions}

\begin{itemize}
\item
\item
\end{itemize}

Q2
\item  
The likelihood function for the ith year \item 
L (mu_i;di,vi) = exp(-mu_i *vi)* mu_i ^di 
Where:
mu_i is the transition rate from unemployed to employed in the ith year \item 
%-------------------------%
\medskip 
di is the number of transitions from state “unemployed” to state “employed”
in ith year \item 
vi is the total observed waiting time in State “unemployed” in ith year \item 
\item  
Which results in maximum likelihood estimate of:
mu_i^hat = di / vi
Therefore,
mu_1^hat = 5000 /30000 = 0.16667 
mu_2 ^hat = 7000/22000 = 0.31818 
\item 
The maximum likelihood estimator mu_2^hat has a variance equal to:
Mu2/E[V] 
Where:
Mu2 is the true transition rate in the second year
E[V] is the expected waiting time of being unemployed
Mu2 \sim = mu_2^hat = 0.31818
E[V] \sim =v2 = 22000 
Variance = 0.000014 
\item 
Estimating
P (not getting employed in the neX_{t} 2 years) = 2p0 = exp( - int( o to 2) mu x+s ds) 
= Exp(-mu1)*exp(-mu2) 
= 0.8465* 0.7275 \item 
= 0.61579 \item 
\end{document}
%%%%%%%%%%%%%%%%%%%%%%%%%%%%%%%%%%%%%%%%%%%%%%%%%%%%%%%%%%%%%%%%%%%%%%%%%%%%%%%%%%%%%
% This question was well answered, and the average mark was the highest across the whole paper.
% In parts \item   and \item   credit was given to candidates who only started to differentiate between years 1 and 2 in the second part, and to candidates who expressed all their answers numerically without full notation. The most common mistake was to calculate a blended transition rate across the two years for which partial credit was given if subsequent calculations proceeded correctly with that rate.
