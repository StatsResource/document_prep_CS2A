%%%%%%%%%%%%%%%%%%%%%%%%%%%%%%%%%%%%%%%%%%%%%%%%%%%%%%%%%%%%%%%%%%%%%%%%%%%%%%%%%%%%%%%%%%%
% HEADER
\documentclass[a4paper,12pt]{article}
\usepackage{eurosym}
\usepackage{vmargin}
\usepackage{amsmath}
\usepackage{graphics}
\usepackage{epsfig}
\usepackage{enumerate}
\usepackage{multicol}
\usepackage{subfigure}
\usepackage{fancyhdr}
\usepackage{listings}
\usepackage{framed}
\usepackage{graphicx}
\usepackage{amsmath}
\usepackage{chngpage}
\usepackage{vmargin}
\setmargins{2.0cm}{2.5cm}{16 cm}{22cm}{0.5cm}{0cm}{1cm}{1cm}
\renewcommand{\baselinestretch}{1.3}
\setcounter{MaxMatrixCols}{10}

\begin{document}
%%%%%%%%%%%%%%%%%%%%%%%%%%%%%%%%%%%%%%%%%%%%%%%%%%%%%%%%%%%%%%%%%%%%%%%%%%%%%%%%%%%%%%%%%%%


7 A mountain rescue service is looking to introduce a new training programme for
volunteers who wish to join the service. Each Saturday for 10 weeks trainee rescuers
are asked to join a mountain climb. Only those who successfully complete the climb
are invited back the following week.
The rescue service will recruit those trainees who successfully complete a certain
number of Saturday climbs. To decide on how many successful weeks should be
required for a new recruit, the rescue service conducts a trial with 20 volunteers. The
table below shows how many of these volunteers fail to complete the climb each week
and the number who are eligible but do not arrive for the beginning of each climb.
Week Eligible but do
not arrive
Arrive but fail to
complete the climb
1 0 1
2 0 2
3 1 2
4 0 0
5 0 1
6 4 1
7 0 2
8 0 1
9 0 2
10 0 1
\item   Explain why the Kaplan–Meier estimate is a suitable way to evaluate this
training programme. 
The rescue service would like to recruit 30% of the volunteers who start the
programme.
\item   Calculate the number of successful weeks the service should require trainees
to complete using the Kaplan–Meier estimate. [8]
\item  Discuss what concerns the rescue service should have about using this study to
set the recruitment criteria for all future volunteers. 

%%%%%%%%%%%%%%%%%%%%%%%%%%%%%%%%%%%%%%%%%%%%%%%%%%%%%%%%%%%%%%%%%%%%%%%%%%%%%%
\subsection*{Solutions}

\begin{itemize}
\item
\item
\end{itemize}


Q7
\item 
We have discrete data 
\begin{itemize
\item The hazard depends on duration / time 
\item 
There is [right] censoring 
\item 
There is non-informative censoring 
\item 
The data is suited to a non-parametric approach 
\item 
Other sensible comment on data suited to K-M approach 
\end{itemize} 

\item 
At duration t weeks, let dt be the number who fail the task that week
ct be the number censored that week
(see below for application of censoring to this problem)
nt be the ”risk set” - the number of volunteers still on the program
then ht is the hazard of failing the task in week t where ht = dt / nt

and the Kaplan Meier survival function is S(t) where
S(t) = \item 
The Kaplan Meier estimate assumes that censoring occurs after failure therefore volunteers who do not arrive for week j need to be included in cj-1 not cj 
t
nt
dt
ct
ht
1-ht
S(t)
1
20
1
0
0.05
0.95
0.95
2
19
2
1
0.105263158
0.894736842
0.85
3
16
2
0
0.125
0.875
0.74375
4
14
0
0
0
1
0.74375
5
14
1
4
0.071428571
0.928571429
0.690625
6
9
1
0
0.111111111
0.888888889
0.613888889
7
8
2
0
0.25
0.75
0.460416667
8
6
1
0
0.166666667
0.833333333
0.383680556
9
5
2
0
0.4
0.6
0.230208333
10
3
1
0
0.333333333
0.666666667
0.153472222
\item 
\item 

\item 
\item 

We seek the largest t at which S(t) >= 0.3 \item 
%-------------------------%
\medskip 
The number of weeks required is 8 
\item 
hazard unlikely to be zero at week 4 \item 
large amount of censoring between weeks 5 & 6 - would want to investigate why 
relatively small sample size \item 
would knowledge of the required number of weeks change behaviour? \item 
the right censoring may well be informative \item 
other types of censoring may be present \item 
different training programs may not be uniformly difficult \item 
other sensible comments \item 
(To obtain full marks on this part some discussion of the censoring here is required)
[Marks 4 available, maximum 3]
[Total 13]
This Kaplan Meier Estimate question was reasonably well answered.
The key to a full correct solution in part \item   and then to a strong answer to part \item  is to understand the role of censoring in this scenario. The ordering of failure and censoring is one of the K-M assumptions and well prepared candidates recognised this in their answers.
A wide variety of layouts for the calculations and answers in \item   were given full credit. Candidates are reminded of the importance of defining terms when completing survival model calculations.
Part \item  was less well answered and again asks candidates to apply knowledge of the model to the scenario in the question.
