CS2A A2023–2
1 A process, X_{t}, is created as follows. $n$ black balls and $n$ white balls are initially placed
in two separate boxes, A and B, in such a way that each box contains $n$ balls. An
experiment is performed in which a ball is selected at random from each box at time t,
(t = 1, 2, …), and the two selected balls interchanged. Let 𝑋􀯧 be the number of white
balls in box A just after time t.
\item   Explain whether X_{t} is irreducible. 
\item   Determine the elements of the transition probability matrix of X_{t}. 
\item Assume that all the $n$ balls that were initially in box A were the white balls and all the
$n$ balls that were initially in box B were the black balls.
 Determine the probability that Xn = 0, simplifying your answer where possible.

%%%%%%%%%%%%%%%%%%%%%%%%%%%%%%%%%%%%%%%%%%%%%%%%%%%%%%%%%%%%%%%%%%%%%%%%%%%%%%
\subsection*{Solutions}

\begin{itemize}
\item
\item
\end{itemize}

Q1
\item  
The chain is irreducible because every state can be reached from any other state 
\item  
We are interested in:
If , 
If , 
For




(Candidates do not need to use the i, j notation - other formats are acceptable including an explanation of where there are zero entries without listing them all)
\item 
Starting with all the white balls in A, getting will require that a white ball is
drawn from A and a black ball from B for each 
Thus, the probability


\end{document}
%%%%%%%%%%%%%%%%%%%%%%%%%%%%%%%%%%%%%%%%%%%%%%%%%%%%%%%%%%%%%%%%%%%%%%%%%%%%%%%%%%%%%
% In fact, as a proportion of the marks available, the average mark was lowest for the whole paper. 
% This is a surprise as the question does not require specialist technical knowledge, but rather basic knowledge of transition probabilities and a careful approach to problem-solving. This question is an excellent example of the comment at the beginning of this report about preparation for open-book examinations.
