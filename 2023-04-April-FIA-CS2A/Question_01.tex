%%%%%%%%%%%%%%%%%%%%%%%%%%%%%%%%%%%%%%%%%%%%%%%%%%%%%%%%%%%%%%%%%%%%%%%%%%%%%%%%%%%%%%%%%%%
% HEADER
\documentclass[a4paper,12pt]{article}
\usepackage{eurosym}
\usepackage{vmargin}
\usepackage{amsmath}
\usepackage{graphics}
\usepackage{epsfig}
\usepackage{enumerate}
\usepackage{multicol}
\usepackage{subfigure}
\usepackage{fancyhdr}
\usepackage{listings}
\usepackage{framed}
\usepackage{graphicx}
\usepackage{amsmath}
\usepackage{chngpage}
\usepackage{vmargin}
\setmargins{2.0cm}{2.5cm}{16 cm}{22cm}{0.5cm}{0cm}{1cm}{1cm}
\renewcommand{\baselinestretch}{1.3}
\setcounter{MaxMatrixCols}{10}

\begin{document}
%%%%%%%%%%%%%%%%%%%%%%%%%%%%%%%%%%%%%%%%%%%%%%%%%%%%%%%%%%%%%%%%%%%%%%%%%%%%%%%%%%%%%%%%%%%


%%% CS2A A2023–2
\large 
\noindent A process, $X_{t}$, is created as follows. $n$ black balls and $n$ white balls are initially placed in two separate boxes, A and B, in such a way that each box contains $n$ balls. 

An
experiment is performed in which a ball is selected at random from each box at time t,
($t = 1, 2, \ldots$), and the two selected balls interchanged. Let ���� be the number of white
balls in box A just after time $t$.

\begin{enumerate}[(a)]
\item   Explain whether $X_{t}$ is irreducible. 
\item   Determine the elements of the transition probability matrix of $X_{t}$. 
\item Assume that all the $n$ balls that were initially in box A were the white balls and all the
$n$ balls that were initially in box B were the black balls.
 Determine the probability that $X_{n} = 0$, simplifying your answer where possible.
\end{enumerate}

%%%%%%%%%%%%%%%%%%%%%%%%%%%%%%%%%%%%%%%%%%%%%%%%%%%%%%%%%%%%%%%%%%%%%%%%%%%%%%
\subsection*{Solutions}

\subsection*{Part (a)}
\begin{itemize}
\item  
The chain is irreducible because every state can be reached from any other state 
\end{itemize}

%%----------------------------------------------%%

\subsection*{Part (b)}
\item  
We are interested in:
If , 
If , 
For

\begin{itemize}
\item
\item
\end{itemize}



% (Candidates do not need to use the i, j notation - other formats are acceptable including an explanation of where there are zero entries without listing them all)


\begin{itemize}
\item Starting with all the white balls in A, getting will require that a white ball is
drawn from A and a black ball from B for each 
\item Thus, the probability

\end{itemize}




\end{document}
%%%%%%%%%%%%%%%%%%%%%%%%%%%%%%%%%%%%%%%%%%%%%%%%%%%%%%%%%%%%%%%%%%%%%%%%%%%%%%%%%%%%%
% In fact, as a proportion of the marks available, the average mark was lowest for the whole paper. 
% This is a surprise as the question does not require specialist technical knowledge, but rather basic knowledge of transition probabilities and a careful approach to problem-solving. This question is an excellent example of the comment at the beginning of this report about preparation for open-book examinations.
