5 A sample of size n is taken from a process, X_{t}, which is believed to be an ARMA(1,1)
process of the form
𝑋􀯧 􀵌 $a$𝑋􀯧􀬿􀬵 􀵅 𝑒􀯧 􀵅 $b$𝑒􀯧􀬿􀬵
where |$a$|, |$b$| 􀵏 1. The sample autocorrelations at lag 1 and lag 2 are 0.65 and 0.325,
respectively.
\item   Estimate the parameters a and b by equating the sample autocorrelations to the
theoretical values. 
Fisher’s transformation states that the sample correlation coefficient, r, between two
random variables, Y and Z, is such that 􀰭
􀰮􀭪􀭭􀭥􁉀􀰭􀰶􀳝
􀰭􀰷􀳝􁉁 is approximately Normally distributed
with mean 􀬵
􀬶 log 􁉀􀬵􀬾􀮡
􀬵􀬿􀮡􁉁 and variance 􀬵
􀯡􀬿􀬷, where \rho  is the theoretical correlation
coefficient between Y and Z and n is the sample size.
\item   Determine the minimum value of n necessary to reject the null hypothesis that
b = 0 in favour of the alternative b > 0 at the 95% significance level. You
should assume that a is equal to the value determined in part \item   and use
Fisher’s transformation on the autocorrelation at lag 1. 

%%%%%%%%%%%%%%%%%%%%%%%%%%%%%%%%%%%%%%%%%%%%%%%%%%%%%%%%%%%%%%%%%%%%%%%%%%%%%%
\subsection*{Solutions}

\begin{itemize}
\item
\item
\end{itemize}

Q5
\item  
We have (1 + a * b) * (a + b) / (1 + b^2 + 2 * a * b) = 0.65 (1) 
and (1 + a * b) * (a + b) / (1 + b^2 + 2 * a * b) * a = 0.325 (2) 
Dividing (2) by (1) gives a = 0.5 \item 
Substituting in (1) gives:
(1 + 0.5 * b) * (0.5 + b) = 0.65 * (1 + b^2 + 2 * 0.5 * b) \item 
i.e. 0.5 * b^2 + 1.25 * b + 0.5 = 0.65 * (b^2 + b + 1) \item 
i.e. 0.15 * b^2 - 0.6 * b + 0.15 = 0. \item 
The roots of this quadratic are 0.268 and 3.732 
We require the root less than 1 in magnitude \item 
which is 0.268 \item 
\item  
For autocorrelation at lag 1 the sample size to be used in the formula for Fisher’s transformation is n - 1 
The test statistic ½ * log ((1 + r_1) / (1 - r_1)) is approximately Normally distributed
with mean ½ * log ((1 + rho_1) / (1 - rho_1)) and variance 1 / (n - 4), where r_1 is
the sample autocorrelation at lag 1 and rho_1 is the theoretical autocorrelation at
lag 1 
For a = 0.5 and b = 0, rho_1 = 0.5 \item 
The 95th percentile of the standard Normal distribution is 1.645 \item 
We therefore require the least positive integer n such that
½ * log ((1 + 0.65) / (1 - 0.65)) - ½ * log ((1 + 0.5) / (1 - 0.5)) > 1.645 / sqrt(n - 4), 
i.e. such that
n > 4 + (1.645 / (½ * log ((1 + 0.65) / (1 - 0.65)) - ½ * log ((1 + 0.5) / (1 - 0.5))))^2. 
The least positive integer n satisfying this inequality is 57 
(Full credit was given to candidates who use n-3 instead of n-4. In this case the final numeric answer will be 56)
[Total 12]
%-------------------------%
\medskip 
This question was not well answered and for the third consecutive session candidates have not answered Time Series questions as well as expected. A lot of candidates spent valuable exam time deriving the autocorrelation formulae rather than applying them, which the examiners suspect was due to the derivations being consulted during the open-book exam rather than the application of the formulae having been revised beforehand.
In part \item   the first marks available are for restating the autocorrelation formulae in terms of the ARMA(1,1) model and then proceeding from there.
