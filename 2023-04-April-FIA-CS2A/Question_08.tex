%%%%%%%%%%%%%%%%%%%%%%%%%%%%%%%%%%%%%%%%%%%%%%%%%%%%%%%%%%%%%%%%%%%%%%%%%%%%%%%%%%%%%%%%%%%
% HEADER
\documentclass[a4paper,12pt]{article}
\usepackage{eurosym}
\usepackage{vmargin}
\usepackage{amsmath}
\usepackage{graphics}
\usepackage{epsfig}
\usepackage{enumerate}
\usepackage{multicol}
\usepackage{subfigure}
\usepackage{fancyhdr}
\usepackage{listings}
\usepackage{framed}
\usepackage{graphicx}
\usepackage{amsmath}
\usepackage{chngpage}
\usepackage{vmargin}
\setmargins{2.0cm}{2.5cm}{16 cm}{22cm}{0.5cm}{0cm}{1cm}{1cm}
\renewcommand{\baselinestretch}{1.3}
\setcounter{MaxMatrixCols}{10}

\begin{document}
%%%%%%%%%%%%%%%%%%%%%%%%%%%%%%%%%%%%%%%%%%%%%%%%%%%%%%%%%%%%%%%%%%%%%%%%%%%%%%%%%%%%%%%%%%%



8 Consider the time-series model:
𝑦􀯧 􀵌 $a$ 𝑦􀯧􀬿􀬶 􀵅 𝑒􀯧 􀵅 $b$ 𝑒􀯧􀬿􀬵 (A)
where 𝑒􀯧 is a white noise process with mean 0 and variance σ􀬶.
\begin{enumerate}
\item   Derive the possible values of $a$ and $b$ for which the process 𝑦􀯧 is stationary
and invertible. 
\item   State the values of p and q for which 𝑦􀯧 is an ARMA(p, q) process. 
%---------------------------%
\medskip
If $b$ 􀵌 0 the original model (A) reduces to
𝑦􀯧 􀵌 $a$ 𝑦􀯧􀬿􀬶 􀵅 𝑒􀯧 (B)
\item  Derive the autocorrelation function for this model while stationarity is
assumed to hold. [8]
An actuary attempts to fit the model (A) to some time series data but concludes that
the simpler model (B) is more appropriate.
\item  Discuss how this conclusion could have been reached. 
\end{enumerate}
\medskip 
%%%%%%%%%%%%%%%%%%%%%%%%%%%%%%%%%%%%%%%%%%%%%%%%%%%%%%%%%%%%%%%%%%%%%%%%%%%%%%
\subsection*{Solutions}

\begin{itemize}
\item
\item
\end{itemize}

Q8
\item  
Using the backshift operator one can show that the corresponding polynomials are
1-a B^2 
and
1+bB 
The roots need to be in absolute value less than 1
abs(a)<1 and abs(b)<1 
\item  
ARMA(2,1) 
\item 
The Yule-Walker equations are
gamma_0=a gamma_2+sigma^2 
and
gamma_k=a gamma_{k-2} for k >= 1 
So
%-------------------------%
\medskip 
gamma_1=a gamma_1 
gamma_2=a gamma_0 
These imply that
gamma_1=0, gamma_2=a gamma_0 and in general 
gamma_k =0 for k odd 
gamma_k = a^{k/2} gamma_0 for k even 
therefore
rho_k=0 for k odd \item 
rho_k=a^(k/2) for k even \item 


%%% (There are no marks available for deriving the Yule Walker equations from first principles)
%----------------------------------------------------------%
\item 
Sample acf of the data could have indicated insignificant spikes for odd lags as
for b=0 case those values are zero 
AIC/BIC could have also been used to confirm the statistical preference between the
two models 
In the parameter estimation process for model (1), some low t-values could have been produced, particularly for the parameter b, indicating over-parametrisation. 
other sensible comments contrasting the fit of the two models 
%----------------------------------------------------------%

\end{document}
