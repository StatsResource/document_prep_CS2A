Q. 3)
The time in minutes, until a squash player scoring each point (against a particular player) is exponentially distributed with parameter \lambda, where \lambda takes values 1/5, 1/2 or 1 depending upon whether the player has made no point, one point or more than one point respectively till the last score. The probability of scoring a point is not dependant on whether the player has previously scored or not. Each game is played up to 11 points.
Note: The player who first scores 11 points wins. Assume that the player is going to win without giving a chance for the opponent to score.
\item Explain how you could model this as a Markov process, commenting on any assumptions made.
(1)
\item Calculate the probability that the player will score more than two points in next 6 minutes (Using differential equation).
(6)
\item Calculate the expected time in minutes to complete the game.
(1)
[8]

Solution 3
\item
This is a 12-state Markov jump process. The states are (1) 0 point, (2) 1 points, (3) 2 points,……..(12) 11 points.
The Markov assumption is that the probability of jumping to any particular state depends only on knowing the current state that is occupied and the transition rates between states are constant over time.
[1]
\item
Let N(t) be the number of points scored up to time t. Then the state space N(t) is the set {0,1,2,3…………………..11}
Since the game starts at point 0 we have N(0)=0.
The transition diagram is
1/5 1/2 1 1 1
……………………….
The required probability is
P(N(6)> 2) = 1- P(N(6) =0)- P(N(6) =1)- P(N(6) =2)
First, we have:
P(N(6) =0) = e-6*1/5 = 0.301194,
Calculating P(N(6) =1)
Forward equation for P(N(t) =1) = P01(t) is
0
1
2
3
11

Page 6 of 17
ⅆⅆ𝑡𝑃01(𝑡)=15𝑃00(𝑡) – 12𝑃01(𝑡)
Solving by integrating factor method, by first writing in the form:
𝑑𝑑𝑡𝑃01(𝑡)+ 12𝑃01(𝑡)=15𝑃00(𝑡)
We have 𝑃00(𝑡) = P(N(t) =0) e-t/5
Substituting for 𝑃00(𝑡), the differential equation becomes
ⅆⅆ𝑡𝑃01(𝑡)+ 12𝑃01(𝑡)=15 e-t/5
Multiplying by the integrating factor et/2 gives:
et/2 * ⅆⅆ𝑡𝑃01(𝑡) + et/2 * 12𝑃01(𝑡) = 15 e3t/10
LHS can be written as:
ⅆⅆ𝑡 ( et/2 * 𝑃01(𝑡))
Integrating, we get,
et/2 * 𝑃01(𝑡) = 1/5* 10/3 * e3t/10 +C
et/2 * 𝑃01(𝑡) = 2/3 * e3t/10 +C
When t= 0: 𝑃01(0)=0
Substituting t=0,
0 = 2/3 + C
C= -2/3
et/2 * 𝑃01(𝑡) = 2/3 * e3t/10 -2/3
𝑃01(𝑡) = 2/3 * e-t/5 -2/3 e-t/2
P(N(6) =1) = 𝑃01(6) = 2/3 * ( e-6/5 - e-6/2 ) = 0.167605
Calculating P(N(6) =2)
Forward equation for P(N(t) =2) = P02(t) is
ⅆⅆ𝑡𝑃02(𝑡)=𝑃00(𝑡)\mu 02 +𝑃01(𝑡)\mu 12+𝑃02(𝑡)\mu 22
Using the transition rates \mu 02=0, \mu 12=12,\mu 22=−1
ⅆⅆ𝑡𝑃02(𝑡)= 1/2*𝑃01(𝑡)-𝑃02(𝑡)
𝑑𝑑𝑡𝑃02(𝑡)+ 𝑃02(𝑡)=12 𝑃01(𝑡)
Substituting for 𝑃01(𝑡),

Page 7 of 17
ⅆⅆ𝑡𝑃02(𝑡)+ 𝑃02(𝑡)= 12 * {2/3 * e-t/5 -2/3 e-t/2 }
= 1/3* { e-t/5 - e-t/2 }
Multiplying by the integrating factor et , we get
ⅆⅆ𝑡{𝑃02(𝑡)et} = 1/3* { e4t/5 - et/2 }
Integrating,
𝑃02(𝑡)et = = 1/3* {5/4* e4t/5 - 2*et/2 } + C
Since , 𝑃02(0) =0
0=13∗{54−2} + C
0= -1/4 + C
C= ¼
𝑃02(𝑡)et = = 1/3* {5/4* e4t/5 - 2*et/2 } + ¼
𝑃02(𝑡)= 1/3* {5/4* e-t/5 - 2*e-t/2 } + ¼ * e-t
P(N(6) =2) = 𝑃02(6)= 1/3* {5/4* e-6/5 - 2*e-3 } + ¼ * e-6
=0.092926
Probability of scoring more than two points in next 6 minutes
= 1- P(N(6) =0)- P(N(6) =1)- P(N(6) =2)
= 1-0.301194 -0.167605-0.092926
=0.438275
[6]
\item
For i= 0,1,2,……10 the expected holding time in state i is 1/\lambdai where \lambdai is the total force out of State i.
State i must be followed by state i+1 . So the expected time to complete the game is
10
Σ 1/\lambdai = 5+ 2+ (1*9) = 16 minutes
i=0
[1]
[8 Marks]

