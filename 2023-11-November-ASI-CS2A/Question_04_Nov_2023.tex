Q. 4)
A health insurance policy offers three types of hospital cash benefits options - Silver, Gold and Platinum, having different lumpsum hospital cash benefits payable.
Claims on these portfolios of policies occur according to a Poisson process with a mean rate of 4 claims per day, irrespective of the type of benefit. 
50% of claims are of type Silver, 30% are of type Gold and remaining of type Platinum.
\item Calculate the expected waiting time until the first claim of type Silver.
(1)
\item Calculate the probability that there are at least 9 claims during the first 2 days, given that there were exactly 7 claims during the first day. 
State any assumptions made.
(2)

Page 4 of 7
\item Calculate the probability that there are at least 5 claims of Gold type during the first day and at least 7 claims of amount Gold type during the first 2 days.
(4)
[7]


Solution 4:
\item
Claims of type Silver occur according to a Poisson process with a mean of 50%* 4 = 2 per day.
So the waiting time until the first claim of type Silver has an exp(2) distribution and the expected waiting time is 1/2 days.
[1]
\item
Let N(t) denote the number of claims during the interval [0,t]. Then:
P[N(2)>=9| N(1)=7]
= P[N(2)- N(1)>=2| N(1)- N(0)=7]
= P[N(2)- N(1)>=2]

Page 8 of 17
Since N(0)=0 and assuming number of claims in non-overlapping time intervals are independent.
N(2)-N(1) ˜ Poisson(4), so:
P(N(2)>=9| N(1)=7) =P(Poisson(4)>=2)
=1- P(Poisson(4)<=1)
= 1- exp(-4) (1+4/1!)
=1- exp(-4)*5
= 1- 0.018316*5
= 0.908422

\item
Let NG(t) denote the number of claims of Gold Type in the interval [0,t].
P(NG(1) >=5, NG(2))>=7)
If we have 7 or more claims during the first day, then the second condition is automatically satisfied.
If we have exactly 6 claims on the first day, then we need at least 1 claim on the second day.
If we have exactly 5 claims on the first day, then we need at least 2 claims on the second day.
So the required probability is:
P(NG(1) >=7) + P(NG(1) =6, NG(2)- NG(1)>=1)
+ P(NG(1) =5, NG(2)- NG(1)>=2) ---------------------------------(1)
Now NG(1) and NG(2)- NG(1) are both independent Poisson with mean 0.3*4 = 1.2 .
P(NG(1) >=7) = 1- (NG(1) <=6) = 1- exp(-1.2)* (1+ 1.2/1!+1.2^2/2! + 1.2^3/3!+ 1.2^4/4! + 1.2^5/5!+ 1.2^6/6!)
=1- .999749
=0.000251
P(NG(1) =6, NG(2)- NG(1)>=1) = P(NG(1) =6)* P(NG(2)- NG(1))>=1)
= P(NG(1) =6)* { 1- P(NG(2)- NG(1))= 0)
= {exp(-1.2)* 1.2^6/6! } * {1- exp(-1.2)}
= 0.000873
P(NG(1) =5, NG(2)- NG(1)>=2) = P(NG(1) =5)* { 1- P(NG(2)- NG(1)) <=1}
= {exp(-1.2)* 1.2^5/5! } *
{1- exp(-1.2)(1+1.2)}
=0.006246*0.337373
=0.002107
Substituting in (1),
The required probability = 0.000251+ 0.000873+0.002107
= 0.003231

[7 Marks]
