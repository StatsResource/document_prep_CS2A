Q. 5)
\item Explain rate interval and three possible definition of age, rate interval, \mu  rate of mortality and q probability of death.
(3)
\item How would an Actuary graduate the crude rates of mortality for following classes of lives:
a) patients dealing with Chronic myeloid leukemia (CML).
b) male population of all civil services of entire country.
c) females with life insurance for large developed country.
d) a life insurance product to be sold to lives under age 30.
(2)
\item An investigation took place into the mortality of diabetes patients between exact age 50 and 51, the force of mortality \mu 50 is assumed to be constant.
The investigation began on 1st April 2022 and ended on 1st April 2023. Table below gives the data collected in this investigation for 10 lives.
Life
Date of birth
Date of entry into observation
Date of exit from observation
Exit due to death (Yes-1, No-0)
1
1st May 1972
1st April 2022
1st April 2023
0
2
1st July 1972
1st April 2022
1st April 2023
0
3
1st October 1972
1st July 2022
1st April 2023
0
4
1st January 1972
1st April 2022
1st July 2022
1
5
1st March 1973
1st April 2022
1st April 2023
0
6
1st December 1971
1st July 2022
1st October 2022
0
7
1st August 1972
1st April 2022
1st April 2023
0
8
1st August 1972
1st April 2022
1st September 2022
1
9
1st November 1972
1st October 2022
1st April 2023
0
10
1st June 1972
1st October 2022
1st April 2023
0
a) Estimate the constant force of mortality and probability of death using a two-state model and the data of 10 lives in the table above.
(4)
b) Assuming the maximum likelihood estimate of the constant force using a Poisson model is exactly same as estimate using the two-state model, outline the differences between the two-state model and the Poisson model when used to estimate the force of mortality or otherwise.
(2)
iv) A graduation of mortality experience has been carried out for the given population.

Page 5 of 7
Age last birthday
𝒒̂x using census data
37
0.0072
38
0.0078
39
0.0081
40
0.0104
41
0.0112
42
0.0114
43
0.0120
Perform test of goodness of fit for the graduated 𝑞̂x where Exposed Actual death data to be captured using following information:
Age nearest birthday
Year 1
Year 2
Beginning of Year 1
Beginning of Year 2
Beginning of Year 3
37
6
8
1050
1043
1053
38
6
8
1130
1123
1133
39
8
8
1060
1052
1062
40
11
9
1235
1225
1235
41
14
12
1160
1149
1159
42
13
13
1370
1357
1367
43
14
16
1240
1225
1235
Explicitly state any assumption of rate interval and degrees of freedom.
(4)
[15]



Solution 5:
\item
A rate interval is a period of one year during which a life’s recorded age remains the same.

Page 9 of 17
Definition of x
Rate interval
𝜇̂ estimates
𝑞̂ estimates
Age last birthday
[x,x+1]
\mu x+1/2
qx
Age nearest birthday
[x-1/2,x+1/2]
\mu x
qx-1/2
Age next birthday
[x-1,x]
\mu x-1/2
qx-1

\item
a) Graphical as no possibility of standard table
b) Parametric formula as large experience is available
c) With reference to standard table as the standard table includes experience for females
d) Spline function to be used as it special case of parametric formula and there are different knots for this age group with infant mortality, accident hump, etc.

\item
a)
\mu  = d/v
we observe 2 deaths i.e. d=2
‘v’ is for all 10 lives and computation of v on annual basis
Exposure period in age 50 to 51 under investigate to be determined
Life (\item
Date of birth
Date of entry into observation
Date of exit from observation
vi(years)
1
1st May 1972
1st April 2022
1st April 2023
11/12
2
1st July 1972
1st April 2022
1st April 2023
9/12
3
1st October 1972
1st July 2022
1st April 2023
6/12
4
1st January 1972
1st April 2022
1st July 2022
3/12
5
1st March 1973
1st April 2022
1st April 2023
1/12
6
1st December 1971
1st July 2022
1st October 2022
3/12
7
1st August 1972
1st April 2022
1st April 2023
8/12
8
1st August 1972
1st April 2022
1st September 2022
1/12
9
1st November 1972
1st October 2022
1st April 2023
5/12
10
1st June 1972
1st October 2022
1st April 2023
6/12
V = Σ𝑣𝑖 = 4.4167
\mu  = 2/ 4.4167 = 0.45283
q=1-exp^(-\mu )
q = 1-0.635826
q= 0.364174

b)
• Poisson model is not an exact model, it allows for non-zero probability of more than n deaths in an sample size of n.
• The variance of MLE of two-state model is only available asymptotically whereas for Poisson model is available in term of true \mu 
• Two-state model extents to processes with increments, whereas Poisson model does not.
• The Poisson model is a less satisfactory approximation to the multiple state model when transition rates are high.

iv)
Null Hypothesis: The graduated rates are the true underlying mortality rates for the population
Standardised deviation at each age is computed below using formula
Zx = 𝜃𝑥−𝐸𝑥𝑞𝑥̂√𝐸𝑥𝑞𝑥̂:
To determine initial exposure and observed deaths adjustment is required at each age for last birthday.

Page 10 of 17
Two approaches:
1st approach:
Exposure = (0.5*Year1 beginning + Year2 beginning + 0.5*Year3 beginning)/2
No of death = avg of 2 years
OR
2nd approach:
Exposure = (0.5*Year1 beginning + Year2 beginning + 0.5*Year3 beginning)
No of death = sum of 2 years
1st approach:
The corresponding goodness of fit to be performed.
Age
Initial Exposure to Risk Ex
Observed Deaths θx
𝒒̂x
Expected Deaths
Ex*𝒒̂x
Zx
𝒁𝒙𝟐
37
1047.25
7
0.0072
7.54
-0.19673
0.03870
38
1127.25
7
0.0078
8.79
-0.60452
0.36545
39
1056.50
8
0.0081
8.56
-0.19063
0.03634
40
1230.00
10
0.0104
12.79
-0.78063
0.60939
41
1154.25
13
0.0112
12.93
0.02014
0.00041
42
1362.75
13
0.0114
15.54
-0.64325
0.41377
43
1231.25
15
0.012
14.78
0.05854
0.00343
OR
2nd approach:
Age
Initial Exposure to Risk Ex
Observed Deaths θx
𝐪̂x
Expected Deaths
Ex*𝐪̂x
Zx
𝐙𝐱𝟐
37
2094.50
14
0.0072
15.08
-0.27821
0.07740
38
2254.50
14
0.0078
17.59
-0.85493
0.73090
39
2113.00
16
0.0081
17.12
-0.26959
0.07268
40
2460.00
20
0.0104
25.58
-1.10398
1.21877
41
2308.50
26
0.0112
25.86
0.02848
0.00081
42
2725.50
26
0.0114
31.07
-0.90969
0.82753
43
2462.50
30
0.012
29.55
0.08278
0.00685
1st approach:
Test statistics = Σ𝒁𝒙𝟐 = 1.47
OR
2nd approach:
Σ𝒁𝒙𝟐 = 2.94
Degrees of freedom = 7-1 = 6
At 5% level, the critical value = 12.59
1.47/2.94<12.59 and no evidence to reject null hypothesis

\end{document}