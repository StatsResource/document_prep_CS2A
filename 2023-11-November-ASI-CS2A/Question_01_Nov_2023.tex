
Q. 1)
An investor analyst assesses the rating of insurance and reinsurance companies every month. 
For the purpose of analysis, the ratings are grouped and classified in the order of merit as Marginal (rating symbols below B), Fair (rating symbols B and B-), Good (rating symbols B+ and B++), Excellent (rating symbols A and A-), Superior (rating symbols A+ and A++).
According to the history, the rating of the bonds evolves as a Markov chain with transition probability matrix given below for some parameter ‘\alpha ’ :
M F G E S
𝑴𝑭𝑮 𝑬𝑺 [ \alpha \alpha 1−2\alpha 00\alpha 1−3\alpha −2\alpha 22\alpha 2\alpha 2002\alpha 21−3\alpha −2\alpha 23\alpha 0002\alpha 21−3\alpha −2\alpha 23\alpha 002\alpha 23\alpha 1−3\alpha −2\alpha 2]
Note: The symbols M, F, G, E and S in the above matrix represent the rating groups Marginal, Fair, Good, Excellent and Superior respectively.
\item Draw the transition graph of the chain.
(2)
\item Determine the range of \alpha  for which the matrix is a valid transition matrix.
(2)
\item Explain whether the chain is irreducible and/or aperiodic.
(1)
iv) Using the above transition probability matrix, calculate the long-run probability that an insurance company is at rating status Superior (S). 
Assume \alpha =0.1.
(4)
[9]

Solution 1:
\item
1-3\alpha -2\alpha 2
1-3\alpha -2\alpha 2
2\alpha 2
1−2\alpha  2\alpha  2\alpha 2
3\alpha 
\alpha  2\alpha 2
3\alpha 
1-3\alpha -2\alpha 2 3\alpha 
\alpha  \alpha  2
2\alpha 2
1-3\alpha -2\alpha 2 2\alpha 2
1-3\alpha -2\alpha 2

\item
The transition matrix will be valid if the entries in each row add up to 1 (which is true) and each row entry lies in the range [0,1]
So, 0 \leq  \alpha \leq 1;
0 \leq  1−2\alpha \leq 1 => 0 \leq  2\alpha \leq 1 => 0 \leq  \alpha \leq 12
0 \leq  3\alpha \leq 1 => 0 \leq  \alpha \leq 13
0 \leq  1-3\alpha -2\alpha ^2 \leq 1
Since \alpha ≥ 0,it automatically implies that 1-3\alpha -2 \alpha ^2 \leq 1
We need to find the values of \alpha  for which 0 \leq  1-3\alpha -2\alpha 2
The equation 0 = 1-3\alpha -2 \alpha 2 => 2 \alpha 2+3\alpha  -1 =0
 \alpha =(−3±√17)4 = 0.280776 or -1.78078
 -1.78078\leq  \alpha  \leq  0.280776
Putting these together, we can see that all of the conditions
0 \leq  \alpha \leq 1
0 \leq  \alpha \leq 12
0 \leq  \alpha \leq 13
-1.78078\leq  \alpha  \leq  .280776
So we must have
0 \leq  \alpha  \leq  0.280776

\item
The chain is irreducible since every state can be eventually reached from every other state.
If \alpha < 0.280766 every state has an arrow to itself, so every state is aperiodic.
When \alpha = 0.280766, none of the states has an arrow to itself. However, return to each of these states is possible in 2 or 3 or 4 steps. 
So, each state is aperiodic. So the chain is aperiodic.
[1]
iv)
When \alpha =0.1 the above transition probability matrix becomes
[ 0.10.10.8000.10.680.20.02000.020.680.30000.020.680.3000.020.30.68]
M
F
S
E
G

Page 3 of 17
The stationary distribution is the vector of probabilities $\pi P = \pi$ , where $P$ is the transition matrix above. 
Writing out the equations, we have
0.1𝛑1+0.1𝛑2 = 𝛑1-------------------------(1)
0.1𝛑1+0.68𝛑2+0.02𝛑3 = 𝛑2------------------(2)
0.8𝛑1+0.2𝛑2+0.68𝛑3+0.𝟎𝟐𝛑4+0.02𝛑5 = 𝛑3------------------(3)
0.02𝛑2+0.3𝛑3+0 .𝟔𝟖𝛑4+0.3𝛑5 = 𝛑4------------------(4)
0.𝟑𝛑4+0.68𝛑5 = 𝛑5------------------(5)
We can discard equation (3) and replace with
𝛑1+𝛑2+𝛑3+ 𝛑4+𝛑5 =1
From equation (1), 𝛑2= 𝟗𝛑1 ----------------------(6)
Substituting in (2) ,
0.1𝛑1 + 0.02 𝛑3 = 0.32 * 𝟗𝛑1
0.02 𝛑3=2.78𝛑1
𝛑3= 𝟏𝟑𝟗𝛑1 ------------------------------(7)
First we solve equations (4) and (5) by eliminating 𝛑5.
From (5) we get, 0.𝟑𝛑4 = 0.𝟑𝟐 𝛑5
𝛑5=0.3/0.32𝛑4
Substituting in (4) we get,
0.02𝛑2+0.3𝛑3+0 .𝟔𝟖𝛑4+0.3∗ 0.3/0.32𝛑4 = 𝛑4
0.02𝛑2+0.3𝛑3= 𝛑4∗( 1− 0.68- 0.09/0.32)= 0.03875 𝛑4
Substituting for 𝛑2 𝑎𝑛𝑑 𝛑3 from equations (6) and (7) in this equation we get
.02*9𝛑1 + 0.3* 139 𝛑1 = 0.03875 𝛑4
𝛑4=1080.774𝛑1
From (5) 0.32𝛑5= 0.3𝛑4 𝛑5=0.30.32∗1080.774𝛑1=1013.226𝛑1
Using the relationship 𝛑1+𝛑2+𝛑3+ 𝛑4+𝛑5 =1 we get
(1+9+139+1080.774+1013.226) 𝛑1 =1
𝛑1 = 0.000446
𝛑5= 1013.226* 0.000446 = 0.451728

[9 Marks]
