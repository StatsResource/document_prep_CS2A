

Q. 6)
Claims on a mediclaim policy follow the lognormal distribution with parameters \mu  = 10 and σ = 0.9. 
The insurer effects an individual excess of loss reinsurance treaty with a retention limit of Rs.15,000.
\item Calculate the probability that the claim paid by the reinsurer is more than Rs. 5,000.
(2)
\item Next year the claim amounts on these policies are expected to increase by 10%. 
The reinsurer wants to change the retention limit so that the probability that a claim involves the reinsurer is 5% higher than the previous arrangement. 
Calculate the new retention limit.
(4)
\item With the new retention limit in (\item, calculate the reinsurer’s expected claim payment per claim next year .
(4)




Solution 6:
\item
Let X denote the individual claim amount random variable. Then X ˜ log N(10, 0.81).
The claim amount paid by the reinsurer exceeds Rs.5000 if X > 20,000
Using the CDF of the lognormal distribution we have:
P(X> 20,000) = \int 𝑓(𝑥)𝑑𝑥\infty20000 , where f(x) is p.d.f of log N(10, 0.81)
= \Phi(U0) - \Phi(L0)
U0 = \infty, L0 = (ln(20000)-10)/0.9 = -0.10724
So, the required probability is equal to
= \Phi(\infty) - \Phi(-0.10724) = 1- 0.457299 =0.542701

\item
In the previous arrangement, i.e. with the old retention limit (Rs.15,000), the probability the a claim involves the reinsurer
P(X> 15000) = \int 𝑓(𝑥)𝑑𝑥\infty15000 = \Phi(\infty) - \Phi(L’0)
L’0 = (In(15000)-10)/.9 = -0.42688
Probability that a claim involves the reinsurer (with the old retention limit)
= 1- \Phi(-0.42688)
= 1-0.334732 = 0.665268
Next year claims are expected to increase by 10%.
Let Z be the new retention limit. The claim amount random variable for next year is 1.1X
So, the probability that a claim involves the reinsurer is
P(1.1X > Z) = P(X> Z/1.1) = \int 𝑓(𝑥)𝑑𝑥\infty𝑍1.1
= \Phi(\infty) - \Phi(L*0) =1- \Phi(L*0),
where L*0 = (In(Z/1.1) -10)/ 0.9
Any one of the two possible interpretations of new probability stated below:
1st: 0.665268+.05 = 0.715268
2nd: 0.665268*1.05 = 0.698531
1st approach:
The choice of Z is such that probability that a claim involves the reinsurer (with the new retention limit) is increased by 5%.
So the new probability becomes 0.665268+.05 = 0.715268
So, 1- \Phi(L*0) = 0.715268
\Phi(L*0) = 1- 0.715268 = 0.284732
L*0 = -0.56884


L*0 = (In(Z/1.1) -10)/ 0.9
In(Z/1.1) = 10+0.9* (-0.56884) = 9.48804
Z/1.1 = exp(9.48804) = 13200.95
So, the new retention limit is equal to Z =13200.95*1.1 =14521.04
Or
2nd approach
The choice of Z is such that probability that a claim involves the reinsurer (with the new retention limit) is increased by 5%.
So the new probability becomes 0.665268*1.05 = 0.698531
So, 1- \Phi(L*0) = 0.698531
\Phi(L*0) = 1- 0.698531= 0.301469
L*0 = -0.52018
L*0 = (In(Z/1.1) -10)/ 0.9
In(Z/1.1) = 10+0.9* (-0.52018) = 9.531837
Z/1.1 = exp(9.48804) = 13791.9
So, the new retention limit is equal to Z =13200.95*1.1 =15171.09

\item
Let Z’ denote the reinsurer’s claim payment random variable next year. Then:
Z’ = 0 if 1.1X<= 14521.04 ( ie X<= 14521.04/1.1)
= 1.1X- 14521.04 if 1.1X > 14521.04 (ie. X> 14521.04/1.1)
As 14521.04/1.1= 13200.95,
E(Z’) = \int  (1.1𝑋−14521.04)𝑓(𝑥)𝑑𝑥\infty13200.95
= 1.1 \int  𝑋𝑓(𝑥)𝑑𝑥\infty13200.95 -14521.04\int  𝑓(𝑥)𝑑𝑥\infty13200.95----------(1)
\int  𝑋𝑓(𝑥)𝑑𝑥=exp(10+0.5∗0.81)\infty13200.95 * (\Phi(\infty) - \Phi(L1))
L1 = (In(13200.95) -10)/0.9- 0.9
= -1.46884
\int  𝑋𝑓(𝑥)𝑑𝑥=exp(10+0.5∗0.81)\infty13200.95∗( 1− \Phi(-1.46884))
= 33024.64*(1- 0.070938)
= 30681.65
\int  𝑓(𝑥)𝑑𝑥\infty13200.95= \Phi(\infty) − \Phi(L0)

Page 13 of 17
L0 = (In(13200.95) -10)/.9= -0.56884
\int  𝑓(𝑥)𝑑𝑥\infty13200.95= 1 − \Phi(−0.56884) = 1-0.284732 = 0.715268
Substituting in (1),
Reinsurer’s expected claim payment next year is equal to
= 1.1* 30681.65 – 14521.04* 0.715268
= Rs. 23363.38

[10 Marks]
