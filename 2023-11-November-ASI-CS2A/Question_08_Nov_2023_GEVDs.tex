Q. 8)
A generalised extreme value distribution used to model flood damage in Actuaria town, in particular the shape parameter ‘\gamma’ used is 1.
\item Identify the type of extreme value distribution and state it’s key characteristic.
(1)
\item Individual losses X of flood in the Actuaria town, follow a Pa(5,2) distribution (i.e  in $Pa(\lambda,\alpha)$ notation)

The maximum claim in the sample of 50 claims is defined as XM which follows approximate distribution of GEV(a,b,c) where 
\begin{eqnarray*}
a &=& \lambda \times (n^{1/\alpha}) -\lambda,\\ 
b &=& \lambda \times (n^{1/\alpha}) /\alpha  \\  
c &=& 1/\alpha \\
\end{eqnarray*}

Calculate the approximate and exact probability that the maximum claim is more than Rs. 4 crores and comment on GEV distribution.
(2)
\item State the key advantage that the generalised Pareto distribution has over the generalised extreme value distribution for the extreme event of flood damage.
(1)

\subsection*{Remark}
\begin{itemize}
\item We are given a Pa(5,2) distribution
\item Shape Parameter $\alpha = 5$
\item Scale Parameter $\lambda = 2$
\item Sample of 50 claims $n=50$
\end{itemize}


%%%%%%%%%%%%%%%%%%%%%%%%%%%%%%%%%%%%%%%%%%%%%%%%
\begin{framed}
The Fréchet distribution, also known as inverse Weibull distribution,[2][3] is a special case of the generalized extreme value distribution. 
It has the cumulative distribution function


$${\displaystyle \Pr(X\leq x)=e^{-x^{-\alpha }}{\text{ if }}x>0.}$$
where α > 0 is a shape parameter. 

It can be generalised to include a location parameter m (the minimum) and a scale parameter s > 0 with the cumulative distribution function


$${\displaystyle \Pr(X\leq x)=e^{-\left({\frac {x-m}{s}}\right)^{-\alpha }}{\text{ if }}x>m.}$$

\end{framed}
\newpage
%%%%%%%%%%%%%%%%%%%%%%%%%%%%%%%%%%%%%%%%%%%%%%


Solution 8:
\item
Since \gamma>0, this is a Fréchet-type GEV distribution.
The key characteristic is they are ‘heavy tail’ whose higher moments can be infinite.
[1]
\item
a = 2 * (50^{1/5}) -2=2.37345 → location parameter
b = 2 * (50^{1/5})/5 = 0.87469 →scale parameter
c=1/5=0.2 →shape parameter


P(XM>4) ≈ 1 – F(XM=4) = 1- exp{-[1+c*(4−𝑎𝑏)]^(−1/𝑐)}
P(XM>4) ≈ 1 – exp{-[1+0.2*(4-2.37345)/0.87469]^(1/0.2)} = 1-0.814027 = 0.18597
P(XM>4) = 1- Fx(4) = 1-[1-(\lambda/(\lambda+4))\alpha ]n = 1-[1-(2/(2+4))5]50
=1-[1-(1/3)5]50 = 0.18632
The probabilities are very close, suggesting the GEV is reasonably approximate for this block size.


%%%%%%%%%%%%%%%%%%%%%%%%%%%%%%%%%%%%%%%%%%%%%%%%%%%%%%%%%%%%%%%%%%%%%%%%%%%%%%%
\item
The key advantage is the generalised Pareto distribution makes use of all the data in the tail whereas generalised extreme value distribution might exclude some data values.


\end{document}
