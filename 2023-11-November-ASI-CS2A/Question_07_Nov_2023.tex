
Q. 7)
\item The following data is observed from n = 100 realisations from a time series:
Σ𝑥𝑖100𝑖=1=5954.39 , Σ(𝑥𝑖−𝑥̅)2100𝑖=1=3832.26 , Σ(𝑥𝑖−𝑥̅)((𝑥𝑖 1−𝑥̅)99𝑖=1=3628.34
Estimate using the data above, the parameters \mu , \alpha  and σ from the model:
Xt - \mu = \alpha  (Xt-1 - \mu )   et
where et is a white noise process with variance σ2.
(3)

Page 6 of 7
\item After fitting the model with the parameters found in (\item, it was calculated that the number of turning points of the residual’s series et is 48. 
Explain and perform significance test at 95% confidence interval on turning points whether there is evidence that et is not generated from a white noise process.
(3)
[6]



Solution 7:
\item
\mu  = 5954.39/100= 59.54
Using the known expression of the auto covariance function for AR(1) process 𝜌𝑘= \alpha 𝑘
\alpha 𝑘̂= 𝜌1 = 3628.34/3832.26 = 0.9468
To estimate variance σ
Taking variance on both sides of Xt - \mu = \alpha  (Xt-1 - \mu ) + et
The fact is Var(Xt - \mu )= Var(Xt-1 - \mu )= \gamma0
Thus, \gamma0=\alpha 2\gamma0+ σ2
𝜎̂2= γ0̂ (1−∝2̂)
=(3832.26/100)*(1-0.94682)
=3.9699
𝜎̂=1.99

\item
Null Hypothesis: Estimate of errors et follows white noise process
We will reject the null hypothesis if the 48 falls outside the Confidence interval of 95%.
Mean = 2(N-2)/3 = 2(100-2)/3 = 65.33
Variance = (16N-29)/90=17.45 and thus S.D. =4.177985
The CI is (65.33± 1.96*4.177985) and thus CI is (57.1, 73.5)
Since 48 is outside confidence interval of 95%, et does not follow white noise process

[6 Marks]
