Q. 2)
An Actuary has been asked to build a model to predict whether a particular policy will have a claim or not. Consider a sample dataset given below corresponding to past year.
Location/Zone
Policy Type
Business Type
Age Group
Claim
(X1)
(X2)
(X3)
(X4)
(Y)
East
Individual
Fresh
<45
No
West
Floater
Port
45-55
Yes
South
Individual
Port
>55
Yes
North
Individual
Port
>55
Yes
East
Floater
Fresh
<45
No
North
Floater
Fresh
45-55
No
South
Floater
Fresh
45-55
No
West
Individual
Port
>55
Yes
East
Individual
Port
<45
No
West
Individual
Fresh
<45
No
North
Floater
Port
>55
Yes
South
Floater
Port
45-55
No

Page 3 of 7
You have been asked to use Naïve-Bayes Classifier Model to predict whether a claim will be made or not for a given set of data inputs.
\item What assumption will you make in order to apply Naïve-Bayes Classifier Model?
(1)
\item Show that the conditional probability that a policy with n characteristics gives a claim can be written as
P (Claim = Yes | X1 = x1, X2 = x2,….., Xn=xn) ∝ P(Claim=Yes) \pi 𝑃(𝑋𝑛𝑗=1j = xj | Claim = Yes)
where Xj is random variable denoting the variable j and Xj is the realisation of that random variable.
(3)
\item Calculate the conditional probability P(Xj | Yj) for each xj in X and yj in Y.
(7)
iv) Assume a new policy A of Individual Policy type from North Zone of age group 45-55 years and Port business type is written by the company. 

Predict whether there will be a claim or not on the given policy. Your calculation shall include proportional probabilities also.

%%%%%%%%%%%%%%%%%%%%%%%%%%%%%%%%%%%%%%%%%%%%%%%%%%%%%%%%%%%%%%%%%%%%%%%%%%%%%%%%%%%%%%%%%%%%%%%%%%%%%%%%%%%%%%

Solution 2:
\item
The fundamental Naïve Bayes assumption is that each feature makes an
• independent and
• equal
contribution to the outcome.
[1]
\item
Using the definition of conditional probabilities, i.e., P(X|Y) = P (X, Y)/P(Y)


P(Claim = Yes | X1 = x1, X2 = x2, ….., Xn = xn) = P( Claim = Yes, X1 = x1, X2 = x2, ….., Xn = xn) / P(X1 = x1, X2 = x2, ….., Xn = xn)
Using the same definition in reverse. i.e., P(X,Y) = P(X|Y) P(Y). we can write the numerator as :
P(Claim = Yes, X1 = x1, X2 = x2, ….., Xn = xn) = P (X1 = x1, X2 = x2, ….., Xn = xn | Claim = Yes) P (Claim = Yes)
Under the naïve bayes approach, we assume that Xj are conditionally independent given the claim outcome. 

This means we can write the first part of above equation as:
P(X1 = x1, X2 = x2, ….., Xn = xn | Claim = Yes) = P(X1= x1 | Claim = Yes) * P(X2= x2 | Claim = Yes) * ….. * P(Xn= xn | Claim = Yes)
= \pi 𝑃(𝑋𝑗=𝑥𝑗 |𝐶𝑙𝑎𝑖𝑚=𝑌𝑒𝑠)𝑛𝑗=1
Finally, treating the denominator as a constant of proportionality in original equation, we can write the desired probability as required:
P (Claim = Yes | X1 = x1, X2 = x2,….., Xn=xn) ∝ P(Claim=Yes) * \pi 𝑃(𝑋𝑛𝑗=1j = xj | Claim = Yes)

\item
The conditional Probability P(Xi | Y\item for each xi in X and yj in Y can be written as:
Location/Zone
Yes
No
P(Yes)
P(No)
North
2
1
2/5
1/7
East
0
3
0/5
3/7
West
2
1
2/5
1/7
South
1
2
1/5
2/7
Total
5
7
100%
100%
Policy Type
Yes
No
P(Yes)
P(No)
Individual
3
3
3/5
3/7
Floater
2
4
2/5
4/7
Total
5
7
100%
100%
Business Type
Yes
No
P(Yes)
P(No)
Fresh
0
5
0/5
5/7
Port
5
2
5/5
2/7
Total
5
7
100%
100%
Age Group
Yes
No
P(Yes)
P(No)
<45
0
4
0/5
4/7
45-55
1
3
1/5
3/7
>55
4
0
4/5
0/7
Total
5
7
100%
100%
Claim
Claim
P(Yes)/P(No)
Yes
5
5/12
No
7
7/12
Total
12
100%
[7]
iv)
Let Policy A = (North Zone, Port, Individual, 45-55)
P(Claim = Yes | Policy A) = P( North Zone | Claim = Yes) P( Port | Claim = Yes) P( Individual | Claim = Yes) P( 45-55 | Claim = Yes) / P(Policy A)

Page 5 of 17
P(Claim = No | Policy A) = P( North Zone | Claim = No) P( Port | Claim = No) P( Individual | Claim = No) P( 45-55 | Claim = No) / P(Policy A)
Since P(Policy A) is common in both probabilities, we can ignore P(Policy A) and find proportional probabilities as:
P(Claim = Yes | Policy A) ∝ 2/5 * 5/5 * 3/5 * 1/5 ≈ 0.048
P(Claim = No | Policy A) ∝ 1/7 * 2/7 * 3/7 * 3/7 ≈ 0.0075
Now, since
P(Claim = Yes | Policy A) + P(Claim = No | Policy A) = 1
These numbers can be converted into a probability by making the sum equal to 1 (normalization):
P(Claim = Yes | Policy A) = 0.0480.048+0.0075 = 0.865
P(Claim = No | Policy A) = 0.00750.048+0.0075 = 0.135
Since, P(Claim = Yes | Policy A) > P(Claim = No | Policy A)
So, prediction that Claim will be there on the given policy is Yes.

[15 Marks]
