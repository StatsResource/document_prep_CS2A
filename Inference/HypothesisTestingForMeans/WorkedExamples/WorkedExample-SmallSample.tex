
\documentclass[a4paper,12pt]{article}
%%%%%%%%%%%%%%%%%%%%%%%%%%%%%%%%%%%%%%%%%%%%%%%%%%%%%%%%%%%%%%%%%%%%%%%%%%%%%%%%%%%%%%%%%%%%%%%%%%%%%%%%%%%%%%%%%%%%%%%%%%%%%%%%%%%%%%%%%%%%%%%%%%%%%%%%%%%%%%%%%%%%%%%%%%%%%%%%%%%%%%%%%%%%%%%%%%%%%%%%%%%%%%%%%%%%%%%%%%%%%%%%%%%%%%%%%%%%%%%%%%%%%%%%%%%%
\usepackage{eurosym}
\usepackage{vmargin}
\usepackage{amsmath}
\usepackage{graphics}
\usepackage{epsfig}
\usepackage{subfigure}
\usepackage{enumerate}
\usepackage{fancyhdr}
\usepackage{framed}

\setcounter{MaxMatrixCols}{10}
%TCIDATA{OutputFilter=LATEX.DLL}
%TCIDATA{Version=5.00.0.2570}
%TCIDATA{<META NAME="SaveForMode"CONTENT="1">}
%TCIDATA{LastRevised=Wednesday, February 23, 201113:24:34}
%TCIDATA{<META NAME="GraphicsSave" CONTENT="32">}
%TCIDATA{Language=American English}

\pagestyle{fancy}
\setmarginsrb{20mm}{0mm}{20mm}{25mm}{12mm}{11mm}{0mm}{11mm}
\lhead{StatsResource} \rhead{Worked Examples} \chead{Inference Procedures} %\input{tcilatex}

\begin{document}

\large 

%---------------------------------------------------------------------------------
\begin{itemize}
\item The standard deviation of the life for a particular brand of ultraviolet tube is known to be $s = 500$ hr,
\item Also it is assumed, but not known, that the operating life of the tubes is normally distributed. \item The manufacturer claims that average tube life
is 9,000hr. \item Test this claim at the 5 percent level of significance against the alternative hypothesis
that the mean life is not 9,000 hr, and given that for a sample of $n = 10$ tubes the mean operating
life was $\bar{x} = 8,800$ hr.
\item (\textit{Intuitively this would suggest a one-tailed test that the mean is less than 9000 hours})
\end{itemize}



\bigskip
\noindent \textbf{Step 1: Hypotheses }
\smallskip
\begin{itemize}
\item $H_0 \mbox{ : } $ $\mu = 9000$ (Average life span is 9000 hours.)
\item $H_1 \mbox{ : } $ $\mu \neq 9000$ (Average life span is not 9000 hours.)
\end{itemize}
\bigskip
\begin{itemize}
\item The observed difference is -200 hours. (i.e. 8,800 - 9,000 hours)
\item The standard error is determined from formulae.
\[ S.E. (\bar{x}) = {s \over \sqrt{n}} = {500 \over \sqrt{10}}  = 158.1139 \]
\end{itemize}

%--------------------------------------------------------------------------------------%

%--------------------------------------------------------------------------------------%
\newpage
\noindent \textbf{Steps 2 and 3: Test Statistic and Critical Value }
\smallskip
\begin{itemize}
\item The test statistic is $(-200 /  158.11) = -1.265$
\item The CV is determined with $\alpha = 0.05$ and $k = 2$ ( column = $\alpha/k=0.025$).
\item The sample is small n = 10 $df = n-1 = 9$ (i.e. row =9).
\item Therefore $CV = 2.262$

\item (Remark: If the sample was large, we could use $CV = 1.96$).

\end{itemize}


%--------------------------------------------------------------------------------------%
\newpage
\noindent \textbf{Step 4: Decision Rule}
\smallskip
\begin{itemize}
\item \textbf{Decision:}\\ Is $|TS| >CV$? \\Is $1.265 > 2.262$?
\item No. We fail to reject the null hypothesis. \item There is not enough evidence to say that the mean lifespan is not 9000 hours.
\end{itemize}



%%%%%%%%%%%%%%%%%%%%%%%%%%%%%%%%%%%%%%%%%5
\newpage
BLANKS
\end{document}

