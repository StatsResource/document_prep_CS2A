
An environmental group states that fewer than 60\% of industrial plants comply with air pollution standards. An independent researcher takes a sample of 400 plants and finds that 270 are complying with air pollution standards. 
\begin{enumerate}[(a)]
\item  Carry out a hypothesis test to investigate the claim made by the environmental group. Clearly state your null and alternative hypotheses and your conclusion.
\item Compute the 95\% confidence interval.
\item Compute the Test Statistic.
\item[(ii)] By interpreting this confidence interval, state your conclusion about the environmental group's claim? Explain how you made this decision.
\end{enumerate}

\item The following questions are concerned with important topic in hypothesis testing
\begin{itemize}
\item[i.] In the context of hypothesis testing, explain what a p-value is, and how it is used. Support your answer with a simple example.
\item[ii.]What is meant by Type I error and Type II error?
\end{itemize}

\item 
The standard deviations of data sets \texttt{X} and \texttt{Y} are 10 and 9 respectively. An inference procedure was carried out to assess whether or not \texttt{X} and \texttt{Y} can be assumed to have equal variance.
\begin{itemize}
\item[i.] Formally state the null and alternative hypothesis.
\item[ii.] The Test Statistic has been omitted from the computer code output. Compute the value of the Test Statistic.
\item[iii.] What is your conclusion for this procedure? Justify your answer.
\item[iv.] Explain how a conclusion for this procedure can be based on the $95\%$ confidence interval.
\end{itemize}

%---- R code for Variance Test ----%
%---- Dummy Code Included                   ----%
\begin{framed}
\begin{verbatim}
        F test to compare two variances
data:  X and Y
F = ......, num df = 13, denom df = 11, p-value = 0.7349
alternative hypothesis: true ratio of variances is not equal to 1
95 percent confidence interval:
 0.3639938 3.9475262
sample estimates:
ratio of variances
          .......
\end{verbatim}
\end{framed}
