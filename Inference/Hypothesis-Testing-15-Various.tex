

\item 

%(b) 
A researcher was investigating computer usage among students at a particular university. Three hundred undergraduates and one hundred postgraduates were chosen at random and asked if they owned a laptop. It was found that 150 of the undergraduates and 80 of the postgraduates owned a laptop. 
\begin{enumerate}[(a)]
\item Find a 95\% confidence interval for the difference in the proportion of undergraduates and postgraduates who own a laptop. 
\item On the basis of this interval, do you believe that postgraduates and undergraduates are equally likely to own a laptop? 
\end{enumerate}


\item 

The weight of 10 students was observed before commencement of their studies and after graduation (in kgs).

\begin{center}
\begin{tabular}{|c||c|c|c|c|c|c|c|c|c|c|} \hline 
Student &	1&	2	&3	&4	&5	&6	&7	&8	&9	&10\\ \hline 
Weight before &	68	& 74	& 59	& 65	& 82&67&57&90&74&77\\ \hline 
Weight after&71&73&61&67&85&66&61&89&77&83\\ \hline 
\end{tabular}
\end{center}


\begin{enumerate}[(i)]
\item Compute the mean and the standard deviation of the case-wise differences.
	\item  Calculate a 95\% confidence interval for the amount of weight that students put on during their studies. 
	\item   Test the hypothesis that the mean weight of students increases during their studies at a significance level of  5\%. 
\end{enumerate}
% Using this confidence interval, test the hypotheses that on average 
% i) students put on 3 kilos during their studies
% ii) students lose 3 kilos during their studies.

% c) What assumption was made in order to both carry out the test and calculate the confidence interval?

%=====================================================%
\item A microbiologist measures the total growth in 24 hours of two strains of a germ culture  in the same petri dish. Nine identical specimens are prepared. The growth rate for the eight samples for each strain are tabulated below:

\begin{center}
	\begin{tabular}{|c|c|c|} \hline 
		Specimen &	Strain 1	&	Strain 2	\\ \hline \hline
		1 & 212 & 224 \\ \hline
		2 & 234 & 231 \\ \hline
		3 & 214 & 209 \\ \hline
		4 & 236 & 243 \\ \hline
		5 & 221 & 231 \\ \hline 
		6 & 212 & 216 \\ \hline
		7 & 202 & 213 \\ \hline 
		8 & 210 & 216 \\ \hline
		9 & 248 & 242 \\ \hline
	\end{tabular} 
\end{center}
\noindent At a significance level of 5\%, is there sufficient evidence to state that there is any difference in growth rates between the two strains.


% State your hypotheses clearly. What is the significance level of this test?


\begin{itemize}
\item[(i)] Formally state the null and alternative hypotheses.
\item[(ii)]  Compute the mean and standard deviation of the case-wise differences.
\item[(iii)] Compute the test statistic.
\item[(iv)] State the appropriate critical value for this hypothesis test.
\item[(v)] Discuss your conclusion to this test, supporting your statement with reference to appropriate values.
\end{itemize}


