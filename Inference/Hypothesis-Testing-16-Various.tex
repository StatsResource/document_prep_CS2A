\documentclass[]{article}

\voffset=-1.5cm
\oddsidemargin=0.0cm
\textwidth = 480pt

\usepackage{framed}
\usepackage{subfiles}
\usepackage{graphics}
\usepackage{newlfont}
\usepackage{eurosym}
\usepackage{amsmath,amsthm,amsfonts}
\usepackage{amsmath}
\usepackage{enumerate}
\usepackage{framed}
\usepackage{color}
\usepackage{multicol}
\usepackage{amssymb}
\usepackage{multicol}
\usepackage[dvipsnames]{xcolor}
\usepackage{graphicx}
\begin{document}
\section*{Hypothesis Testing - Tutorial Sheet B  }
\begin{enumerate}




\item 


Six athletes run 400 metres both at sea level and 1000m above sea level. The times they take are given below. 

\begin{center}
\begin{tabular}{|c|c|c|c|c|c|c|} \hline
Runner & 	1&	2 & 	3	& 4	& 5&  	6\\ \hline
Time at sea level&	45.8&	47.1&	45.4&	46.1 & 	48.4& 	44.9\\ \hline
Time at altitude&	44.8&	45.1&	45.4&	45.1 &	45.4& 	45.9\\ \hline
\end{tabular}
\end{center}
\begin{enumerate}[(a)]
    \item Is there any evidence that their speed depends on altitude? 
\item Calculate a 95\% confidence interval for the mean difference between the time at sea level and the time at altitude. Using this confidence interval, test the null hypothesis that the average time at altitude is 3 seconds faster than the average time at sea level (state the appropriate significance level).
\item What assumption is used in both parts a) and b)? 
\end{enumerate}
%===================================%
\item  A claim has been made that the mean body temperature of healthy adults is equal to 98.6 degrees Fahrenheit. A sample of 106 people who are taking medication for a chronic illness has produced a mean body temperature of 100.3 degrees Fahrenheit and a standard deviation of 0.65 degrees. Test the claim that this population of people taking medication has a different mean body temperature to the general population.  Clearly state your null and alternative hypotheses and your conclusion. Use a 5\% level of significance.




%================================================%
\item
A new process has been developed to reduce the level of corrosion of car bodies. Experiments were carried out on 11 cars using the new process and 11 cars using the old process. The average level of corrosion using the new process was 3.4 with a standard deviation of 0.5. The average level of corrosion using the old process was 4.2 with a standard deviation of 0.8. 
\begin{enumerate}[(a)]
\item Test the hypothesis that the mean of the level of corrosion does not depend on the process used. \\ \textit{For the time being, you may assume that assumption of equal variance is valid. }
\item Is there any evidence that the new process is better at a significance level of 5\%?
\item Calculate a 95\% confidence interval for the difference between the mean levels of corrosion under the two processes. Can it be stated that the mean level of corrosion is reduced by 1.5 at a significance level of 5\%? 
%\item What assumptions were used in ii) and iii)? 
\end{enumerate}

%================================================%

\item 
 A researcher was investigating computer usage among students at a particular university. Three hundred undergraduates and one hundred postgraduates were chosen at random and asked if they owned a laptop. It was found that 150 of the undergraduates and 80 of the postgraduates owned a laptop. 
\begin{itemize}
    \item[(a)]  Find a 95\% confidence interval for the difference in the proportion of undergraduates and postgraduates who own a laptop. 
    \item[(b)] On the basis of this interval, do you believe that postgraduates and undergraduates are equally likely to own a laptop? 
\end{itemize}
%================================================%

 

