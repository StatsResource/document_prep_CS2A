Q. 3)
In an actuarial student program, the number of years students stay in the program is distributed as follows:
1 year 0.85
2 years 0.60
3 years 0.55
4 years 0.45
The distribution of the amount of time in the program after 4 years has probability density function f(t) = \mu*e(-\mu*t), with \mu selected to match the 0.45 probability of staying in the program 4 years.
Determine the average number of full years that students stay in the program.
[4]

%%%%%%%%%%%%%%%%%%%%%%%%%%%%%%%%%%%%%%%%%%%%%%%%%%%%%%%%%%%%%%%%%%%%%%%%%%

Solution 3:
The life expectancy at entry is e0 = \Sum𝑘𝑝0∞𝑘=1 
After four years, the survival rate is 0.45, and
S0(4) = \int𝑓(𝑡)ⅆ𝑡∞4=\int\mu𝑒−\mu𝑡ⅆ𝑡∞4=ⅇ−4\mu
Therefore ⅇ−4\mu=0.45
\mu = -0.25* ln 0.45 = 0.2

For k >= 4,
kp0 = \int\mu𝑒−\mu𝑡ⅆ𝑡∞𝑘=ⅇ−𝑘\mu = 0.45(k/4)

So e0 = 0.85 + 0.60 + 0.55 + \Sum0.45𝑘∕4∞𝑘=4

= 2 + 0.451−0.45(14⁄) (From sum of a geometric series = a/(1-r))

= 2 + 2.4867
= 4.4867 years 

\begin{itemize}
\item 
\item 
\item 
\end{itemize}
%%%%%%%%%%%%%%%%%%%%%%%%%%%%%%%%%%%%%%%%%%%%%%%%%%%%%%%%%%%%%%%%%%%
\end{document}