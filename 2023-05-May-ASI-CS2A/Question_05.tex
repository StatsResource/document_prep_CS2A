[20]
Q. 5)
\item  Show that: \intexp(−5𝑦13)𝑑𝑦∞𝑥 = 6/125 P{Y > 10 x1/3}
where Y is a chi-square random variable with six degrees of freedom.
(4)
\item   Using (\item , deduce an expression involving a chi-square probability for the mean residual life for Weibull(5, 1/3) distribution.


\item   By calculating the values of mean residual life function when x=1 , x=8 and x=27, determine whether the mean residual life of the Weibull (5,1/3) distribution is an increasing or decreasing function of x.

[8]


%%%%%%%%%%%%%%%%%%%%%%%%%%%%%%%%%%%%%%%%%%%%%%%%%%%%%%%%%%%%%%%%%%%%%%%%%%%%%%%%%%%%%%%%%%%%%%%

Solution 5:
\item 
Substituting u= 5y (1/3) in the given integral expression:
y =(u/5)3
➔ dy =1/125* 3 u2 *du 
The integral becomes: 3/125\intu^2∗exp(−𝑢)ⅆ𝑢∞5𝑥^(13)

= 6/125\int13∗u3−1∗exp(−𝑢)(2!)ⅆ𝑢∞5𝑥^(13)
= 6/125* P[U> 5*x(1/3)] , where U is Gamma (3,1) 
=6125∗𝑃{𝑌>10∗𝑥^(13)}, where Y is Chi (6)
[1.5 marks for final expression and the justification]

\item  
Mean residual life of Weibull (5,1/3) distribution is :
e(x)= \int{1−𝐹(𝑦)}ⅆ𝑦∞𝑥/ (1-F(x))

= \intⅆ𝑦∞ exp( −5y^(1/3))𝑥/ exp( -5x^(1/3))

= 6/125* P[Y> 10*x^(1/3)]/ exp(-5x^(1/3)), where Y is Chi (6)
[1 mark for correct expression and mention of Y’s distribution]

\item  
When x =1, we have e(x) = 6/125* P(chi square (6) >10)/ e-5
= 6/125* (1-.875348)/ exp (-5)=.888

When x =8, we have e(x) = 6/125* P(chi square (6) >20)* e-10
= 6/125* (1 - 0.997231)/ e-10 = 2.928

When x =27, we have e(x) = 6/125* P(chi square (6) >30)* e-15
= 6/125* (1 - 0.999961)/ e-15 = 6.168

%%%%%%%%%%%%%%%%%%%%%%%
Page 6 of 14
The mean residual life is an increasing function of x as the value of the function is increasing with increasing x.


[8 Marks]
