%%%%%%%%%%%%%%%%%%%%%%%%%%%%%%%%%%%%%%%%%%%%%%%%%%%%%%%%%%%%%%%%%%%%%%%%%%%%%%%%%%%%%%%%%%%
% HEADER
\documentclass[a4paper,12pt]{article}
\usepackage{eurosym}
\usepackage{vmargin}
\usepackage{amsmath}
\usepackage{graphics}
\usepackage{epsfig}
\usepackage{enumerate}
\usepackage{multicol}
\usepackage{subfigure}
\usepackage{fancyhdr}
\usepackage{listings}
\usepackage{framed}
\usepackage{graphicx}
\usepackage{amsmath}
\usepackage{chngpage}
\usepackage{vmargin}
\setmargins{2.0cm}{2.5cm}{16 cm}{22cm}{0.5cm}{0cm}{1cm}{1cm}
\renewcommand{\baselinestretch}{1.3}
\setcounter{MaxMatrixCols}{10}

\begin{document}
%%%%%%%%%%%%%%%%%%%%%%%%%%%%%%%%%%%%%%%%%%%%%%%%%%%%%%%%%%%%%%%%%%%%%%%%%%%%%%%%%%%%%%%%%%%


Q. 9)
\item  Let Xt be an autoregressive process of order 1 i.e. AR(1) given by:
Xt = \mu + \alpha (Xt-1 - \mu) + et
where {et : t = 1, 2,….} is white noise process.
Prove that:
Xt = \mu + \alphat * (X0 - \mu) + \Sum(𝑡−1𝑗=0 j * et– j)
(2.5)
\item   Prove that:
var(Xt ) = \Sum2 (1- \alpha2t ) / (1- \alpha 2) + \alpha 2t * var (X0)
where \Sum2 denotes the common variance of the white noise terms {et : t = 1, 2,….}, and ‘var’ denotes the variance.
(2.5)
%%%%%%%%%%%%%%%%%%%%%%%
Page 5 of 6
\item   \lambda =2 is a root of the characteristic equation of the process:
6 Xt - 13 Xt -1 + 9 Xt -2 - 2 Xt -3 +et
Calculate the other roots and classify the process as I(d).
(4)
iv) Consider the stationary AR process defined by the equation:
Xt = 712 Xt -1 – 112 Xt -2 + et
Determine the values of the ACFs ρ1 and ρ2 and the PACFs φ1 and φ2.
(4)
[13]

%%%%%%%%%%%%%%%%%%%%%%%%%%%%%%%%%%%%%%%%%%%%%%%%%%%%%%%%%%%%%%%%%%%%%%%%%%%%%%%%%%%%%%%%%%%%%%%%%%%%%%%%%%%

Solution 9:
\item 
Xt = \mu + \alpha (Xt-1 \mu) + et
Xt-1 = \mu + \alpha (Xt-2 - \mu) + et-1
Substituting for Xt-1 , equation (1) becomes
Xt = \mu + \alpha (\mu + \alpha (Xt-2 - \mu) + et-1 - \mu) + et
= \mu + \alpha2 * (Xt-2 - \mu) + \alphaet-1 + et

Repeating this process for one step
Xt = \mu + \alpha3 * (Xt-3 - \mu) + \alpha2et-2 + \alphaet-1 + et

In next two steps:
Xt = \mu + \alpha4* (Xt-4 - \mu) + + \alpha3 et-3 + \alpha2et-2 + \alphaet-1 + et
After t steps the equation iteratively becomes:
Xt = \mu + \alpha t * (X0 - \mu)+ \Sum(𝑡−1𝑗=0j * et– j)
[1 mark for showing the iteration sufficiently]
[2.5]
\item  
Xt = \mu + \alpha t * (X0 - \mu)+ \Sum(𝑡−1𝑗=0j * et– j)
Taking variance,
Var (Xt ) = var [ \mu + \alphat * (X0 - \mu) + \Sum(𝑡−1𝑗=0j * et– j)

= \alpha2t * var(X0 - \mu) + \Sum(𝑡−1𝑗=02j * var(et– j))

= \alpha2t * var(X0) + \Sum2 *\Sum(𝑡−1𝑗=02j)

Now using the formula a+ ar+ ar2+……..ar(n-1) = a(1- rn) / (1- r) for the n terms of Geometric progression:
%%%%%%%%%%%%%%%%%%%%%%%
Page 8 of 14
var (Xt ) = \alpha2t * var (X0) + \Sum2 ( 1- 2t) / (1- \alpha 2 )

[2.5]
\item  
The process can be written as
-6 Xt + 13 Xt -1 - 9 Xt -2 + 2 Xt -3 = et
The characteristic equation is
-6 + 13 z – 9 z2 + 2 z3 =0
Since z =2 is a root, we can write this equation as
-6 + 13 z – 9 z2 + 2 z3 = (z-2)( az2+ bz +c), where a,b and c are constants.

One way to solve this equation is comparing the coefficients on both sides.
2z3 = az3
a=2
Comparing the constant term,
-6= -2c
c= 3

Comparing the coefficient of z,
13= c -2b
13=3- 2b
10= -2b
b= -5 
So the equation can be written as
-6 + 13 z – 9 z2 + 2 z3 = (z-2)( 2z2-5z +3)
= (z-2)( 2z-3) (z-1)
So the other roots of the equation are \lambda =1 and \lambda = 3/2
The process is not stationary since one root is strictly equal to 1 (for the process to be stationary all the roots should be strictly greater than 1). 
-6 Xt + 13 Xt -1 - 9 Xt -2 + 2 Xt -3 = et
➔ -6 Xt + 6 Xt -1 + 7 Xt -1 – 7 Xt -2 - 2 Xt -2 + 2 Xt -3 = et
➔ -6 ∇Xt + 7 ∇Xt-1 -2 ∇Xt-2 = et
Characteristic equation then becomes:
➔ -6 + 7z` -2z`2 =0
The roots of this equation are: z` = 2, 3/2
So differencing the process eliminates the root of 1. The two remaining roots (i.e. 2 and 3/2) are strictly greater than 1 in magnitude.
So the differenced process is stationary. Hence the process is I(1).

[4]
iv)
Autocovariance at lag 1, γ1= cov(Xt, Xt-1)
= cov(712Xt -1 – 112Xt -2 + et, Xt-1 )
Since, cov(et, Xt -k) = 0 for all k>=1
i.e., γ1 = 712 γ0 - 112 γ1
Rearranging we get,
1312 γ1= 712γ0
γ1= 713γ0
%%%%%%%%%%%%%%%%%%%%%%%
Page 9 of 14
Dividing by γ0
ρ1=713
Autocovariance at lag 2, γ2= cov(Xt, Xt-2)
= cov(712 Xt -1 – 112 Xt -2 + et, Xt-2)
ie. γ2 = 712 γ1 - 112 γ0
Dividing by γ0
ρ2= 712 ρ1 – 112
ρ2 = 712 *713 – 112 = 49156 – 112 = 36156 = 313
φ1= ρ1 = 713
φ2= (ρ2 - ρ12)/ (1- ρ12)
= (3/13 –( 7/13)^2) / (1- 49/169)
= (-10)/169 / (120/169)
= - 1/12
[4]
[13 Marks]
