

Page 2 of 6
Q. 1)
In a historic city in India, an industrialist wants to make money by offering the denizens a first-of-its-kind experience of viewing the historic places in the city from a helicopter. A brand-new 3-seater helicopter is employed for the ride. As it’s a 3-seater helicopter, a group of 3 passengers are gathered at the boarding center, before they are allowed to board it. Hence, the helicopter would not fly until all the 3 seats are full. Passengers arrive at the center according to a Poisson process with the rate of \lambda = 1/15 per minute.
\item  Calculate the expected waiting time until the first helicopter takes off.

\item   What is the probability that helicopter does not take off in the first two hours, assuming that there are no other hindrances?

\item   Today, it has been informed by the authorities that the weather conditions will not be conducive after three hours. However, the operator wants to fly at least 3 rides today before weather condition becomes poor. What is the probability that the operator completes at least 3 rides in three hours?
(3)


%%%%%%%%%%%%%%%%%%%%%%%%%%%%%%%%%%%%%%%%%%%%%%%%%%%%%%%%%%%%%%%%%%%%%%%%%%%%%%

Solution 1:
\item 
Let Zn= time between arrival of the nth and (n – 1)th clients.
Then Zn's are i.i.d. exponential random variables with mean 1 / \lambda i.e. E[Zn]= 1𝜆
Let Tn = arrival time of the nth passenger = \Sum𝑧𝑛𝑛𝑖=1 𝑬[𝑻𝒏]=𝑬[\Sum𝒛𝒏𝒏𝒊=𝟏]=\Sum𝑬[𝒛𝒏]𝒏𝒊=𝟏= 𝒏𝝀
The expected waiting time until the first chopper takes off is
𝑬[𝑻𝟑]= 𝟑[𝟏/𝟏𝟓] = 45 minutes

\item   Let X(t) be the Poisson process with mean \lambda*t. Note that P(X(t)) = k =(e– \lambdat *(\lambdat)k)/k!,
For k= 1, 2, 3….
We have
P = P[No helicopter takes off in the first 2 hours]
=P[At most 2 passengers in first 120 mins}]
=P[{X(t) <= over (0,120)}]
=P[{X(120) <= 2]
=P[X(120) = 0] + P[X(120) = 1] + P[X(120) = 2]
= ⅇ−12015+(12015)ⅇ−12015+ 12(12015)2ⅇ−(12015)
= 0.01375
= 1.375%

\item  
In order to ensure that the operator flies at least 3 trips in next 3 hours before the weather conditions worsen, there should be at least 9 passengers.
(1)
Hence, the probability that at least 9 clients would arrive is
P = P[{At least 9 passengers arrive in 180 mins}]
= 1 – P[{At most 8 clients arrive in 180 mins }]
= 1 – P[{X(180) <= 8]
= 1 - \Sum𝑃𝑘=8𝑘=0[𝑋(180)=𝑘]
= 1 – P[X(180) = 0] – P[X(180) = 1] – P[X(180) = 2] – P[X(180) = 3] – P[X(180) = 4] – P[X(180) = 5] – P[X(180) = 6] – P[X(180) = 7] – P[X(180) = 8]
= 1 - ⅇ̇−18015[\Sum(18015)𝑘80∗1𝑘!]
(1)
= 1 – 0.15503
= 84.5% (1)

\begin{itemize}
\item 
\item 
\item 
\end{itemize}
%%%%%%%%%%%%%%%%%%%%%%%%%%%%%%%%%%%%%%%%%%%%%%%%%%%%%%%%%%%%%%%%%%%
\end{document}