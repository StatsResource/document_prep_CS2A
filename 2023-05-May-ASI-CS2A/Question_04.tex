%%%%%%%%%%%%%%%%%%%%%%%%%%%%%%%%%%%%%%%%%%%%%%%%%%%%%%%%%%%%%%%%%%%%%%%%%%%%%%%%%%%%%%%%%%%
% HEADER
\documentclass[a4paper,12pt]{article}
\usepackage{eurosym}
\usepackage{vmargin}
\usepackage{amsmath}
\usepackage{graphics}
\usepackage{epsfig}
\usepackage{enumerate}
\usepackage{multicol}
\usepackage{subfigure}
\usepackage{fancyhdr}
\usepackage{listings}
\usepackage{framed}
\usepackage{graphicx}
\usepackage{amsmath}
\usepackage{chngpage}
\usepackage{vmargin}
\setmargins{2.0cm}{2.5cm}{16 cm}{22cm}{0.5cm}{0cm}{1cm}{1cm}
\renewcommand{\baselinestretch}{1.3}
\setcounter{MaxMatrixCols}{10}

\begin{document}
%%%%%%%%%%%%%%%%%%%%%%%%%%%%%%%%%%%%%%%%%%%%%%%%%%%%%%%%%%%%%%%%%%%%%%%%%%%%%%%%%%%%%%%%%%%


Q. 4)
With the increasing number of G3M2 virus cases in the country, two vaccine manufacturers in the country have quickly developed two different single-vial vaccines. These vaccine manufacturers have their own distribution channel, which are well connected through-out the country. These vaccines need to be stored in a temperature-controlled case and an agency has been entrusted with the responsibility of administering the vaccines to the interested public. Considering that the vaccines have been developed within a very short time, there are very few takers for the vaccine worrying the potential side-effects. Due to limited availability, an agency of a particular vaccine manufacturer is permitted to store only four vaccines.
The number of vials administered during a day by an agency is a random variable with the following discrete distribution:
%%%%%%%%%%%%%%%%%%%%%%%
Page 3 of 6
No. of vials potentially administered in a day
Probability
0
50%
1
25%
2
25%
The probability of administration of greater than one vial in a day remains 50%. If the agency has no vaccines in stock at the end of a day, the agency contacts its supplier to order four more vaccines. The vaccines are delivered the following morning, before the agency opens. The vaccine supplier makes a charge of C for the delivery.
\item  Write down the transition matrix for the number of vaccines in stock when the agency opens in a morning, given the number of vaccines when the agency opened the previous day.

\item   Calculate the stationary distribution for the number of vaccines in stock when the agency opens, using your transition matrix in part (\item .

\item   Calculate the expected long term average number of restocking orders placed by the agency per day.

iv) Calculate the expected long-term number of orders lost per day.

The agency is unhappy about losing these sales as there is a profit of P on each sale. It therefore considers changing its restocking approach to place an order before it has run out of vaccines. The charge for the delivery remains at C irrespective of how many vaccines are delivered.
v) Evaluate the expected number of restocking orders, and number of lost sales per trading day, if the agency decides to restock if there are fewer than two vaccines remaining in stock at the end of the day.

\item  Explain why restocking when two or more vaccines remain in stock cannot optimize the agency’s profits.

The agency wishes to maximize the profit it makes on the vaccines.
\item   Derive a condition in terms of C and P under which the agency should change from only restocking where there are no vaccines in stock, to restocking when there are fewer than two vaccines in stock.


%%%%%%%%%%%%%%%%%%%%%%%%%%%%%%%%%%%%%%%%%%%%%%%%%%%%%%%%%%%%%%%%%%%%%%%%%%%%%%%%%%%%%%%%%%%%%%%%%
Solution 4:
\item 
Start previous Start morning
Day 1 2 3 4
1 0.5 0 0 0.5
2 0.25 0.5 0 0.25
3 0.25 0.25 0.5 0
4 0 0.25 0.25 0.5

\item  
If stationary distribution is π = (\pi_{1} \pi_{2} \pi_{3} \pi_{4})
Then π A = π where A is the matrix in (\item 

0.5 \pi_{1} + 0.25 \pi_{2} + 0.25 \pi_{3} = \pi_{1} (\item  
0.5 \pi_{2} + 0.25 \pi_{3} + 0.25 \pi_{4} = \pi_{2} (\item   
0.5 \pi_{3} + 0.25 \pi_{4} = \pi_{3} (\item  

%%%%%%%%%%%%%%%%%%%%%%%
Page 4 of 14
0.5 \pi_{1} + 0.25 \pi_{2} + 0.5 \pi_{4} = \pi_{4} (IV)

From (\item  , \pi_{3} = 0.5 \pi_{4}

From (\item  , \pi_{2} = 0.75 \pi_{4}

From (\item , \pi_{1} = 0.625 \pi_{4}

\pi_{1} + \pi_{2} + \pi_{3} + \pi_{4} =1 = (0.625 + 0.75 + 0.5 +1) \pi_{4}

Solving the above equation,
\pi_{1} = 0.21739 , \pi_{2} = 0.26087 , \pi_{3} = 0.17391, \pi_{4} = 0.34783 
[Max 5]
\item  
Probability of restocking is 0.5 if in \pi_{1} and 0.25 if in \pi_{2}
So long term rate = 0.5 * 0.21739 + 0.25 * 0.26087 
= 0.108695 + 0.06522 = 0.17391 per trading day 

iv)
Probability of losing a sale is 0.25 if in \pi_{1} 
So expected long term lost sales per day = 0.25 * 0.21739 = 0.05435 

v)
Start previous Start morning
Day 2 3 4
2 0.5 0 0.5
3 0.25 0.5 0.25
4 0.25 0.25 0.5

Let the stationary distribution be expressed as \lambda
Then \lambda M = \lambda where M is the matrix above
\lambda2 = 0.5 \lambda2 + 0.25 \lambda3 + 0.25 \lambda4 (A) 
\lambda3 = 0.5 \lambda3 + 0.25 \lambda4 (B) 
\lambda4 = 0.5 \lambda2 + 0.25 \lambda3 + 0.5 \lambda4 (C) 
From (B), \lambda3 = 0.5 \lambda4
From (A), \lambda2 = 0.75 \lambda4
Solving the equation \lambda2 + \lambda3 + \lambda4 = 1, we get
\lambda2 = 0.33333 or 1/3 
\lambda3 = 0.22222 or 2/9 
\lambda4 = 0.22222 / 0.5 = 0.44444 or 4/9 
As no more than two vaccines sell per day, there are no lost sales.
Probability of restocking 0.5 if in \lambda2 and 0.25 in \lambda3 = 0.5*0.33333 + 0.25*0.22222 = 0.22222 
[5]
v\item 
Restocking at two or more vaccines would not result in fewer lost sales than restocking at 1, because the probability of selling more than 2 vaccines is zero.

%%%%%%%%%%%%%%%%%%%%%%%
Page 5 of 14
It would, however, result in more restocking charges than restocking at 1.
Therefore, it must result in lower profits than restocking at 1 so is not optimal. 

v\item  
Costs if restock at zero vaccines: 0.17391C + 0.05435P 
Costs if restock at one vaccine: 0.22222C

So should change restocking approach if 0.22222C < 0.17391C + 0.05435P
i.e. C < 1.1255P


[20 Marks]
