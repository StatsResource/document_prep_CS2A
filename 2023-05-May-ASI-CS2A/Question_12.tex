%%%%%%%%%%%%%%%%%%%%%%%%%%%%%%%%%%%%%%%%%%%%%%%%%%%%%%%%%%%%%%%%%%%%%%%%%%%%%%%%%%%%%%%%%%%
% HEADER
\documentclass[a4paper,12pt]{article}
\usepackage{eurosym}
\usepackage{vmargin}
\usepackage{amsmath}
\usepackage{graphics}
\usepackage{epsfig}
\usepackage{enumerate}
\usepackage{multicol}
\usepackage{subfigure}
\usepackage{fancyhdr}
\usepackage{listings}
\usepackage{framed}
\usepackage{graphicx}
\usepackage{amsmath}
\usepackage{chngpage}
\usepackage{vmargin}
\setmargins{2.0cm}{2.5cm}{16 cm}{22cm}{0.5cm}{0cm}{1cm}{1cm}
\renewcommand{\baselinestretch}{1.3}
\setcounter{MaxMatrixCols}{10}

\begin{document}
%%%%%%%%%%%%%%%%%%%%%%%%%%%%%%%%%%%%%%%%%%%%%%%%%%%%%%%%%%%%%%%%%%%%%%%%%%%%%%%%%%%%%%%%%%%

Q. 12)
\item  Consider the multivariate time series represented by the following pair of equations:
Xt = 0.7 Xt-1+ 0.4 Yt-1 + etX
Yt = 0.1 Xt-1+ 0.2 Yt-1 + etY
where etX and etY are independent white-noise processes. Using eigenvalues, state whether the process is stationary or not.

\item   Consider univariate process:
Xn= 0.8 Xn-1- 0.4 Xn-2+en
where en is a white noise process.
Calculate values of autocorrelation function at lag 1, 2 and expression for autocorrelation function at lag k (k>2).
(4)
\item   Two time series X and Y are defined by the equations:
Xt= 0.55 Xt-1+ 0.45 Yt-1 + etX
Yt= 0.45 Xt-1+ 0.55 Yt-1 + etY
where etX and etY are independent white noise processes.
Show that X and Y are cointegrated, with cointegrating factor (1,-1).

%%%%%%%%%%%%%%%%%%%%%%%%%%%%%%%%%%%%%%%%%%%%%%%%%%%%%%%%%%%%%%%%%%%%%%%%%%%%%
Solution 12:
\item 
Calculate the eigenvalues of the matrix:
(0.70.40.10.2)

i.e. we need to determine the value of \lambda for which ⅆⅇ𝑡(0.7− \lambda0.40.10.2− \lambda)=0

➔ ( 0.7- \lambda) * (0.2- \lambda)- 0.04 =0
➔ 0.10- 0.9 \lambda + \lambda2 =0
\lambda = 0.77 or 0.129

Since both the eigenvalues are strictly less than 1 in magnitude, the process is stationary.


\item  
The autocovariance at lag 1 is :
γ1 = cov (Xn, Xn-1)
= cov (0.8 Xn-1- 0.4 Xn-2+en, Xn-1)
= 0.8cov (Xn-1, Xn-1) - 0.4cov (Xn-2, Xn-1) + cov (en, Xn-1)

γ1 = 0.8 γ0 - 0.4 γ1
ie. 1.4 γ1 = 0.8 γ0
1 = γ1/ γ0 = 8/14

The autocovariance at lag 2 is:
γ2= cov (Xn , Xn-2)
γ2 = 0.8 γ1- 0.4 γ0

2 = γ2 / γ0

So, 2 = 0.8 1 - 0.4
2 = 8/10 * 8/14 – 4/10
= 4/5 *4/ 7- 2/5
= 16/35 -2/5
= 2/35

In general, for k>2,
γk= cov (Xn , Xn-k)
γk = 0.8 γk-1- 0.4 γk-2

k = γk/ γ0

%%%%%%%%%%%%%%%%%%%%%%%
Page 14 of 14
Therefore,
k= 0.8 k-1 - 0.4k-2

[4]
\item  
To show that the processes X and Y are I(1):
From the first equation:
Yt-1= 1/ 0.45* [ Xt – 0.55 Xt-1 - etX ]

Using this in the second equation,
1/ 0.45* [ Xt+1 – 0.55 Xt - et+1X ]= 0.45 Xt-1 + . 0.55/ .45 * [ Xt – 0.55 Xt-1 - etX ] + etY

This simplifies to:
Xt+1 = 1.1 Xt - - 0.1 Xt-1 + et+1X - .55 etX + 0.45 etY
Equivalently,
Xt = 1.1 Xt-1 - 0.1 Xt-2+ etX - .55 e-1X + 0.45 et-1Y

The characteristic polynomial of the AR part of this equation is:
1- 1- 1.1 \lambda + 0.1 \lambda2 = 0
2- ➔ \lambda = 10 or 1
3- The roots of this equation are 10 and 1. Since one root is equal to 1, X is not stationary.

4- Differencing once will eliminate the root of 1. Since the only root is strictly greater than 1 in magnitude X is I (1).

5- The process of Y is may be expressed as a linear form of I(1) process X and etY, therefore, Y is also I (1).

Now to verify if: X-Y is stationary:
Xt- Yt = 0.1 Xt-1 – 0.1 Yt-1 + etX – etY
Setting Wt = Xt – Yt,
The process is AR (1) since the root of its characteristic equation is 10 and this is greater than 1 in magnitude.

Therefore, Wt is stationary, Xt – Yt is also thus stationary and X and Y are cointegrated, with cointegrating factor (1,-1).

\begin{itemize}
\item 
\item 
\item 
\end{itemize}
%%%%%%%%%%%%%%%%%%%%%%%%%%%%%%%%%%%%%%%%%%%%%%%%%%%%%%%%%%%%%%%%%%%
\end{document}