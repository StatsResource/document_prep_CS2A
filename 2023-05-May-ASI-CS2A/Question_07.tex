%%%%%%%%%%%%%%%%%%%%%%%%%%%%%%%%%%%%%%%%%%%%%%%%%%%%%%%%%%%%%%%%%%%%%%%%%%%%%%%%%%%%%%%%%%%
% HEADER
\documentclass[a4paper,12pt]{article}
\usepackage{eurosym}
\usepackage{vmargin}
\usepackage{amsmath}
\usepackage{graphics}
\usepackage{epsfig}
\usepackage{enumerate}
\usepackage{multicol}
\usepackage{subfigure}
\usepackage{fancyhdr}
\usepackage{listings}
\usepackage{framed}
\usepackage{graphicx}
\usepackage{amsmath}
\usepackage{chngpage}
\usepackage{vmargin}
\setmargins{2.0cm}{2.5cm}{16 cm}{22cm}{0.5cm}{0cm}{1cm}{1cm}
\renewcommand{\baselinestretch}{1.3}
\setcounter{MaxMatrixCols}{10}

\begin{document}
%%%%%%%%%%%%%%%%%%%%%%%%%%%%%%%%%%%%%%%%%%%%%%%%%%%%%%%%%%%%%%%%%%%%%%%%%%%%%%%%%%%%%%%%%%%


Q. 7)
Q. 7)
A single policy follows a Poi (0.30)
Individual claim amounts follow a Pareto (\alpha = 3, \lambda = 900)
Claim investigation and processing expense (independent of claim) follow Gamma (\alpha = 100, β = 5)
Premium = 150
Portfolio comprises of n policies (independent).
Find number of policies required to be profitable within 95% confidence interval.
[6]
Q. 8)
\item  For Gumbel copula, determine whether the generator function:
𝜓(t) = (-lnt) \alpha
is a strict generator function for \alpha >=1, and determine the inverse generator function for Gumbel copula.

\item   Derive the coefficient of lower tail dependence for the Gumbel copula for \alpha≥1.
(4)
[6]


%%%%%%%%%%%%%%%%%%%%%%%%%%%%%%%%%%%%%%%%%%%%%%%%%%%%%%%%%%%%%
\section*{Solution}
\begin{itemize}
\item
Solution 7:
X ~ Individual claim
Y ~ Commission
N ~ Poi(0.3n)
E[X] = 900/2 = 450
E[X2] = 3X9002/(22x1) + 4502 = 810000

E[S] = 0.3nE[X+Y]
Var(S) = 0.3n[E(X2)+2E(X) E(Y) + E(Y2)]

E[Y] = 100/5 = 20
E[Y2] = Γ(102) / { Γ(100) X 52}
= 10! / {90! X 25}
= 404
Let ‘n’ be the number of policies.
E[S] = 0.3n(450+20)
= 141n
Var(S) = 0.3(810000+2*450*20+404)
= 2428521.2n

S ~ N(141n, 248521.2n) (approximation)
Profit → P(S<=150n)
➔ P(S<150n) ~P[N(0,1)] < (150n-141n)/sqrt(248521.2n)

➔ P[N(0,1)] < 0.018sqrt(n)
➔ 0.018sqrt(n) >= 1.64485
➔ n > 8350 
[6 Marks]

\begin{itemize}
\item 
\item 
\item 
\end{itemize}
%%%%%%%%%%%%%%%%%%%%%%%%%%%%%%%%%%%%%%%%%%%%%%%%%%%%%%%%%%%%%%%%%%%
\end{document}