Q. 6)
Light bulbs have the following distribution for the amount of time until burning out:
Time t in hours
F(t)
0 – 4800
0
4800 – 6000
(t – 4800)/1200
6000
1
Each bulb uses 0.015 kilowatt-hours of electricity per hour.
Calculate the expected number of kilowatt-hours used by 50 bulbs in their first 5000 hours.


%%%%%%%%%%%%%%%%%%%%%%%%%%%%%%%%%%%%%%%%%%%%%%%%%%%%%%%%%%%%%%%%%%%%%%%%%%%%
Solution 6:
E[min(T,5000) = \int𝑠0(𝑡)ⅆ𝑡50000=\int1ⅆ𝑡48000+\int(1−𝑡−48001200)ⅆ𝑡50004800 
= 4800 + 5t - 𝑡22400
= 4800 + 183.333333
= 4983.33333 
The number of kilowatt-hours used by 50 bulbs is 50 * 0.015 * 4983.333 = 3737.5 KWH



\begin{itemize}
\item 
\item 
\item 
\end{itemize}
%%%%%%%%%%%%%%%%%%%%%%%%%%%%%%%%%%%%%%%%%%%%%%%%%%%%%%%%%%%%%%%%%%%
\end{document}