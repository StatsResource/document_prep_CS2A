%%%%%%%%%%%%%%%%%%%%%%%%%%%%%%%%%%%%%%%%%%%%%%%%%%%%%%%%%%%%%%%%%%%%%%%%%%%%%%%%%%%%%%%%%%%
% HEADER
\documentclass[a4paper,12pt]{article}
\usepackage{eurosym}
\usepackage{vmargin}
\usepackage{amsmath}
\usepackage{graphics}
\usepackage{epsfig}
\usepackage{enumerate}
\usepackage{multicol}
\usepackage{subfigure}
\usepackage{fancyhdr}
\usepackage{listings}
\usepackage{framed}
\usepackage{graphicx}
\usepackage{amsmath}
\usepackage{chngpage}
\usepackage{vmargin}
\setmargins{2.0cm}{2.5cm}{16 cm}{22cm}{0.5cm}{0cm}{1cm}{1cm}
\renewcommand{\baselinestretch}{1.3}
\setcounter{MaxMatrixCols}{10}

\begin{document}
%%%%%%%%%%%%%%%%%%%%%%%%%%%%%%%%%%%%%%%%%%%%%%%%%%%%%%%%%%%%%%%%%%%%%%%%%%%%%%%%%%%%%%%%%%%


Solution 8:
\item 
When t=0, 𝜓(0) = lim𝑡→0(−ln𝑡)\alpha =∞
Therefore, it is a strict generator function.

The inverse function is found by rearranging the equation:
x= 𝜓(t) = (-ln t)
-ln t= x^( 1/\alpha)
t= exp(-x ^(1/ \alpha))

%%%%%%%%%%%%%%%%%%%%%%%
Page 7 of 14

\item  
C[u,v] = 𝜓-1[𝜓(𝑢)+𝜓(𝑣)]
= 𝜓-1[(−ln𝑢)\alpha+(−ln𝑣)\alpha]
= exp {- ( (−ln𝑢)\alpha+(−ln𝑣)\alpha ) ^(1\alpha)} for \alpha≥1

The coefficient of lower tail dependence is given by:
\lambdaL = = lim𝑢→0+𝑐[𝑢,𝑢]𝑢

= lim𝑢→0+ [exp { − ( (−ln𝑢)\alpha+(−ln𝑢)\alpha ) ^(1\alpha)}/u]

=lim𝑢→0+ [exp { − (2^(1\alpha) (−In u) }/u]

= lim𝑢→0+[exp { (2^(1\alpha) (In u) }/u
 =lim𝑢→0+(u21\alpha) /u
= lim𝑢→0+u21\alpha−1
= 0 

\begin{itemize}
\item 
\item 
\item 
\end{itemize}
%%%%%%%%%%%%%%%%%%%%%%%%%%%%%%%%%%%%%%%%%%%%%%%%%%%%%%%%%%%%%%%%%%%
\end{document}