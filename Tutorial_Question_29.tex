\documentclass[12pt]{article}
\usepackage{amsmath}
%\usepackage[paperwidth=21cm, paperheight=29.8cm]{geometry}
\usepackage[angle=0,scale=1,color=black,hshift=-0.4cm,vshift=15cm]{background}
\usepackage{multirow}
\usepackage{enumerate}
\usepackage[gen]{eurosym}

%\SetBgScale{1}
%\SetBgAngle{0}
%\SetBgColor{black}
%\SetBgContents{\rule{1pt}{30cm}}
%\SetBgHshift{-8.4cm}
%
%\backgroundsetup{contents={
%\begin{tabular}{c|c}
%\hspace{2cm} & \\[0.7cm]
%& {\bf Statistics for Computing ------ Lecture 1 ------ Solutions} \\[0.3cm]
%%\hline
%\hspace{2cm} & \hspace{18.5cm} \\ [28cm]
%\end{tabular}}}

\backgroundsetup{contents={
		{\bf \centering Statistics for Computing ------------------ Tutorial 5 --------------------------- Questions} }}


\setlength{\voffset}{-3cm}
\setlength{\hoffset}{-2cm}
\setlength{\parindent}{0cm}
\setlength{\textheight}{27cm}
\setlength{\textwidth}{17cm}

\pagestyle{empty}



\begin{document}

\textbf{MODULE CODE: MA4104 SEMESTER: Spring 2014/15
	
	MODULE TITLE: Business Statistics DURATION: 2.5 hours
	
	LECTURER: Dr. Helen Purtill
	%===============================================================%
	
\newpage
Q5
A manufacturing technology engineer wanted to establish if there was a relationship between the pull strength of injection moulded parts and the dwell time in the mould. A random sample of 7 different times and their corresponding pounds per square inch were recorded as follows:
X	Y
Time (in hours)	Pull Strength
1	2.8	7.8
2	2.9	8.1
3	3.1	8
4	3.3	8.4
5	3.7	8.6
6	3.9	8.8
7	4.1	9
%% \sumxy =	200.89	 \sumx2 = 	82.46	 \sumx = 23.8	\sumy = 58.7
(a) 	You are required to 
i.	Draw a scattergram and comment on its features
ii.	Find the regression equation and plot the regression equation on scattergram
(8 marks)
The data was entered into Minitab and the following outputs were generated
Predictor	Coef		SE Coef	T		P
Constant			0.2880		19.07		0.000
Hours				0.08391	10.14		0.000
S= 0.1041	R-Sq = 95.4%		R-Sq(adj) = 94.4%
(b) 	You are requested to explain how the T-value of 10.14 was calculated and to 
interpret the corresponding P value of 0.000
(4 marks)
PTO
(c)	Fill in the blanks from the following tables and explain the relationship between F value of 102.76 and the T-value of 10.14 in section (b)
(8 marks)
Analysis of Variance
Source			DF		SS		MS		F		P
Regression		x		1.1144		1.1144		102.76		0.000
Residual Error		x		xx		xxx		
Total			x		1.1686
Observation		Time		Pull Strength		Fitted value	Residual
2		
\newpage
	%-----------------------------------------------------------------------------------%
	
	Q5
	A wood scientist wanted to establish if there was a relationship between the adhesive strength of laminated wood and the dwell time in press machine. A random sample of 7 different times and their corresponding adhesive strengths in pounds per square inch were recorded as follows:
	
	
	X	Y
	Time (in minutes)	Pull Strength
	1	4.3	2.8
	2	4.4	3.1
	3	4.6	3
	4	4.8	3.4
	5	5.2	3.6
	6	5.4	3.8
	7	5.5	4
	
%%	\sumxy =117.04	 \sumx2 = 168.05	 \sumx = 34.2	\sumy = 23.7      \sumy2 = 81.41
	%------------------------------------------------------------%
	
	Q5 continued
	
	(a) 	You are required to 
	i.	Draw a scattergram and comment on its features
	ii.	Find the regression equation and plot the regression equation on scattergram
	iii.	Use a 5% level of significance to test the hypothesis that R correlation coefficient equals Zero. Interpret your answer.
	(8 marks)
	
	The data was entered into Minitab and the following outputs were generated
	
	Predictor	Coef		SE Coef	T		P
	Constant			0.4597		-2.06		0.095
	Hours				0.09369	 9.46		0.000
	
	S= 0.1112	R-Sq = 94.7%		R-Sq(adj) = 93.7%
	
	
	(b) 	You are requested to explain how the T-value of 9.46 was calculated and to 
	interpret the corresponding P value of 0.000
	(4 marks)
	
	
	
	(c)	Fill in the blanks from the following tables and explain the relationship between F value of 89.51 and the T-value of 9.46 in section (b)
	(8 marks)
	Analysis of Variance
	Source			DF		SS		MS		F		P
	Regression		x		1.1067		1.1067		89.51		0.000
	Residual Error		x		xx		xxx		
	Total			x		1.1686
	
	
	
	Observation		Time		Pull Strength		Fitted value	Residual
	2			4.4		3.1			x		x
	
	%-----------------------------------------------------------------------------------%
	
\end{document}