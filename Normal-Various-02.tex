
%======================================================== 
\section{Worked Example 11 - with Solutions}
%% Repeat 2005 (question 2)(b)	
The compressive strength of concrete for fresh-water exhibition tanks has mean 5000 psi and standard deviation 240 psi. Assuming that the compressive strength is normally distributed, calculate the probability that the compressive strength of a sample of concrete is less than 4900 psi.




	

%======================================================== %
\subsubsection{Question 12 Solution}

Question 1(a)

Upper Limit ;  $U = \mu + 3 \sigma $
Lower Limit ;  $L = \mu - 3 \sigma $

Standardisation
Apply the standardisation formula	$Z=\frac{x-\mu}{\sigma} $	to both limits

\[ Z_U = \frac{U-\mu}{\sigma} =  \frac{(\mu + 3 \sigma)-\mu}{\sigma} = 3\]

Similary

$Z_l=-3$ 

\noindent \textbf{Probability of point being above Upper Limit}

From Murdoch Barnes Tables (page 13)  $P(Z \geq 3)=0.00135$

Probability of point being below Lower Limit


To find   we use the “Property of Symmetry”

“Property of Symmetry” -   for any value A

Therefore 

Conclusion: 
Probability of point being outside the 3 Sigma limits is

+ =0.00270 	(i.e. 0.27%)
















Question 1(b)

Upper Limit ;  
LowerLimit ;  

Standardisation
Apply the standardisation formula	 	to both limits


Similary


\begin{itemize}
\item Probability of point being above Upper Limit

From Murdoch Barnes Tables (page 13)  

\item Probability of point being below Lower Limit


To find   we use the “Property of Symmetry”

\item “Property of Symmetry” -   for any value A

Therefore 

Conclusion: 
\item Probability of point being outside the 3 Sigma limits is

+ =0.04550 	(i.e. 4.55%) Question 1(b)
\end{itemize}















Question 2C

Upper Limit ; 80.64		Mean		 	
Lower Limit ; 75.36		Standard Deviation	 

Standardisation
Apply the standardisation formula	 	to both limits


Similary



\begin{itemize}
\item Probability of being above Upper Limit

\item From Murdoch Barnes Tables (page 13)  

Probability of being below Lower Limit


\item \item To find   we use the “Property of Symmetry”

“Property of Symmetry” -   for any value A

Therefore 

Conclusion: 
\item Probability of point being outside the specification limits 

+ is equal to

+ =0.2584 	(i.e. 26%)
\end{itemize}













Question 3A

Upper Limit ; 90		Mean		 	
Lower Limit ; 50		Standard Deviation	 

Standardisation
Apply the standardisation formula	 	to both limits


Similary


\subsubsection{Upper Limit 1.5}
Probability of being above Upper Limit

From Murdoch Barnes Tables (page 13)  

Probability of being below Lower Limit


To find   we use the “Property of Symmetry”

“Property of Symmetry” -   for any value A

Therefore 

Conclusion: 
Probability of point being outside the specification limits is

+ is equal to


+ =0.01478  	(i.e. 1.5%)






%-----------------------------------------------%
Question 3B

Upper Limit ; 90		Mean		 	
Lower Limit ; 50		Standard Deviation	 

Standardisation
Apply the standardisation formula	 	to both limits


Similary



Probability of being above Upper Limit

From Murdoch Barnes Tables (page 13)  

Probability of being below Lower Limit


To find   we use the “Property of Symmetry”

“Property of Symmetry” -   for any value A

Therefore 

Conclusion: 
Probability of point being outside the specification limits is

+ is equal to

+ =0.01099  	(i.e. 1.1%)







\subsection{Solutions 1}

\begin{enumerate}
	
	\item Assume that the number of weekly study hours for students at a certain university
	is approximately normally distributed with a mean of 22 and a standard deviation
	of 6.
	\begin{enumerate}
		\item Find the probability that a randomly chosen student studies less than 12
		hours.
		\item Estimate the percentage of students that study more than 37 hours.
	\end{enumerate}
	
	
	$X \sim \mathcal(22,6^2)$\\
	$P(X \leq 12)$\\
	$P(X \geq 37)$\\
	$Z_1 = \frac{12 - 22}{6} = \frac{-10}{6} = -1.66 $\\
	$Z_2 = \frac{37 - 22}{6} = \frac{15}{6} = 2.5 $
	
	
\end{enumerate}







%----------------------------------------------------------------------------------------------%








		




