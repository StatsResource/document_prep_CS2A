\documentclass[]{report}

\voffset=-1.5cm
\oddsidemargin=0.0cm
\textwidth = 480pt

\usepackage{framed}
\usepackage{subfiles}
\usepackage{graphics}
\usepackage{newlfont}
\usepackage{eurosym}
\usepackage{amsmath,amsthm,amsfonts}
\usepackage{amsmath}
\usepackage{color}
\usepackage{amssymb}
\usepackage{multicol}
\usepackage[dvipsnames]{xcolor}
\usepackage{graphicx}
\begin{document}
\chapter{Normal Distribution - Worked Examples}
\section{Calculations}

\begin{framed}
	\begin{itemize}
		\item The Complement Rule
		\begin{equation}
		P(Z \leq z) = 1 - P(Z \geq z)
		\end{equation}
		\item The Symmetry Rule
		\begin{equation}
		P(Z \leq -A) = P(Z \geq A)
		\end{equation}
		\item The Interval Rule.
		Where $L$ and $U$ are the lower and upper bounds of an interval.
		\begin{equation}
		P(L \leq Z \leq U) = P(Z \geq L) -  P(Z \geq U)
		\end{equation}
		
	\end{itemize}
\end{framed}

\section{Summary of Normal Distribution}

\begin{enumerate}
	\item Working with Tables
	
	\[P(Z \geq Zo)\]
	
	\item The Standardisation Formula
	
	\[P(X \leq Xo) = P(Z \leq Zo)	  \]  
	
	when   \[Zo=\frac{Xo- \mu}{\sigma}\]
	
	\item Complement Rule
	
	\[P(Z\geq Z_0) = 1 - P(Z \leq Z_0)\]
	
	\item  Symmetry Rule
	
	
	\[P(Z \leq -Z_0) = P(Z \geq Z_0)\]
	
\end{enumerate}


%-------------------------------------------------%



\section{Normal - example}

In an examination the scores of students who attend schools of type A are
normally distributed about a mean of 55 with a standard deviation of 6. The
scores of students who attend type B schools are normally distributed about a
mean of 60 with a standard deviation of 5.

Which type of school would have a higher proportion of students with marks above 70?

\begin{itemize}
	\item $\mu_A$ = 55
	\item $\sigma_A$ = 8
	\item $\mu_B$ = 60
	\item $\sigma_B$ = 5
\end{itemize}

We have to fins $P(X_A \geq 70)$
and $P(X_B \geq 70)$.


using the standardisation formula
$Z_A = \frac{70 - 55}{6} = \frac{15}{6} = 2.5 $

$Z_B = \frac{70 - 60}{5} = \frac{10}{5} = 2 $



\section{Normal Distribution : Arab Horses Worked Example}
The mass of Arab horses is normally distributed with mean 900 lbs and standard deviation of 50lbs.
\begin{itemize}
	
	\item  Calculate the probability that an Arab horse weighs more than 940 lbs.
	\item Calculate the probability than an Arab horse weighs between 880 lbs and 960 lbs.
\end{itemize}

\noindent \textbf{Solution}\\

\begin{itemize}
	\item Let X be mass of Arab horses.
	
	\item We have to find $P(X\leq940)$.            (Remark "equality component" is included as a formality, but it is not important)
	
	
	\item	Find the Z value that corresponds to 940 
	
	\[Zo=\frac{Xo-\mu}{\sigma}= \frac{940 -900}{50}= 0.8\]
	
	\[P(X \leq 940) = P(Z \leq 0.8) \]
	
	
	\item		From Murdoch Barnes tables 3, we find that $P(Z \leq 0.8) = 0.2119$
\end{itemize}





Solution 

\[P ( 880 \leq X \leq 960).\]


What proportion of horses are between 880 lbs and 960 lbs?
\begin{itemize}
	\item Find out the probability of the complement event.
	\item The complement event is the combination of being too high  or too low for this interval.
	
	\item Inside interval $P ( 880 \leq X \leq 960).$
	
	\item Outside interval $P (X \leq 880) + P(X \geq 960)$
	
	\item Complement Rule $P ( 880\leq X \leq 960)  = 1 - [P (X\leq 880) +P(X\geq 960)]$
	
\end{itemize}



Find the probability of being too high?

\[Zo=\frac{Xo-\mu}{\sigma}= \frac{960 -900}{50}= 1.2\]

\[P(X \leq 960) = P(Z \leq 1.2) = 0.1151\]


Find the probability of being too low?
\[Zo= \frac{Xo-\mu}{\sigma}= \frac{880 -900}{50}= -0.4 \]
\[P(X \leq 880) = P(Z \leq -0.4)  \]

%------------------------------------------------------------- %
How to compute $P(Z \leq -0.4)$

Symmetry: 	$P(Z \leq -0.4)$ = $P(Z \geq 0.4)$ = 0.3446


\begin{itemize}
	\item Outside Interval = 0.4596        (0.3446 +  0.1151)
	\item Inside Interval = 0.5404
\end{itemize}



%------------------------------------------------------------- %
What weight is exceeded by 97.5\% of Arab horses?

Find Xo  such that P(XXo) = 0.975

\begin{itemize}
	\item 	P(Z1.96) = 0.025     [From Tables] 
	
	\item	P(Z-1.96) = 0.025  [Symmetry]
	
	\item	P(Z-1.96) = 0.975         
	
	\item	\[-1.96 = \frac{Xo- 900}{50} \]
	
	
	\item	Xo= 802 lbs  [Answer]
\end{itemize}	





%------------------------------------------------------------- %



%----------------------------------------------------------------%
\section{Worked Example 2 - with Solutions}
IQ scores are assumed to have a normal distribution with mean 100 and standard deviation 15.

\begin{itemize}
	\item What IQ would you have if you were in the 80th percentile?
	\item Estimate the threshold for the top 10 percent?
	\item What is the probability of having an IQ above 142?
	\item What is the probability of having an IQ below 97?
\end{itemize}


%----------------------------------------------------%
	
	\section{Worked Examples : Repeat 2006}
	Suppose an oil exploration company purchases drill bits that have a life span that is approximately normally distributed, with a mean equal to 80 hours and a standard deviation equal to 10 hours.
	
	\begin{itemize}
		\item[(i)]	What is the probability that a drill bit will fail before 60 hours of use?
		
		\item[(ii)]	What is the probability that a drill bit will last between 70 hours and 90 hours?
		
		\item[(iii)]	The life span of 95\% of drill bits is below what value?
		
	\end{itemize}
	
	
	\section{Worked Examples : MA4104 Business Statistics SPRING 2008}
	
	A tyre manufacturer claims that under normal driving conditions, the tread life of a certain tyre follows a normal distribution with mean 50,000 miles and standard deviation 5000 miles. 
	
	\begin{itemize}
		\item[(i)] If your tyres wear out at 45,000 miles, would you consider this unusual? Support your answer with an appropriate probability calculation using the normal curve. [ 10 marks ] 
		
		\item[(ii)] If the manufacturer sells 100,000 of these tyres and warrants them to last at least 40,000 miles, about how many tyres will wear out before the warranty expires? [ 10 marks ] 
	\end{itemize}
\section{Worked Example 3 - with Solutions}
Assume that the number of weekly study hours for students at a certain university
is approximately normally distributed with a mean of 22 and a standard deviation
of 6.
\begin{enumerate}
	\item Find the probability that a randomly chosen student studies less than 12
	hours.
	\item Estimate the percentage of students that study more than 37 hours.
\end{enumerate}

\textbf{solution}
$X \sim \mathcal(22,6^2)$  ( in form $X \sim \mathcal(\mu,\sigma^2)$\\

Part 1: $P(X \leq 12)$\\


$Z_1 = \frac{12 - 22}{6} = \frac{-10}{6} = -1.66 $\\

Part 2: $P(X \geq 37)$\\

$Z_2 = \frac{37 - 22}{6} = \frac{15}{6} = 2.5 $\\


\section{Worked Example 4 - with Solutions}
The mean is 550kg, with standard deviation 150kg, and we are interested in the area that is greater than 600kg.

\begin{equation}
Z = \frac{ X - \mu }{ \sigma }
\end{equation}

Here X = 600kg,
$\mu$ , the mean = 550kg
$\sigma$, the standard deviation = 150kg
\begin{itemize}
	\item $z = ( 600 - 550 ) / 150$
	\item $z = 50 / 150$
	\item $z = 0.33$
\end{itemize}

Look in the table down the left hand column for z = 0.3, and across under 0.03.
The number in the table is the tail area for z=0.33 which is 0.3707.
This is the probability that the weight will exceed 600kg.



%----------------------------------------------------%

\section{Worked Example 8 - with Solutions (Tyres)}
% \emph{Taken from MA4104 Business Statistics Examination paper, Spring 2008}\\
% Q1. (a) 
A tyre manufacturer claims that under normal driving conditions, the tread life of a certain tyre follows a normal distribution with mean 50,000 miles and standard deviation 5000 miles.

\begin{itemize}
	\item[(i)] If your tyres wear out at 45,000 miles, would you consider this unusual? Support your answer with an appropriate probability calculation using the normal curve. [ 10 marks ]
	\item[(ii)] If the manufacturer sells 100,000 of these tyres and warrants them to last at least 40,000 miles, about how many tyres will wear out before the warranty expires? [ 10 marks ]
\end{itemize}


Part (i) Solution

\begin{itemize}
	\item Test Value ; 45,000km			
	\item Mean		 km	
	\item Standard Deviation	 km
\end{itemize}


Find  

Standardisation
Apply the standardisation formula	 	to test value


i.e.  = 

To find   we use the “Property of Symmetry”

“Property of Symmetry” -   for any value A

From Murdoch Barnes Tables (page 13)  

Therefore   = 0.1587 

“Complement Rule”		 =1-  for any given value A

=1-  = 0.8413

Conclusion

15.87\% of Tyres are expected to last less than 45,000km

\begin{itemize}
\item 84.13\% of Tyres are expected to last longer than 45,000km




Part (i) Solution

Lower Limit ; 40,000km			Mean		 km	
Standard Deviation	 km

Find  

\item Standardisation
Apply the standardisation formula	 	to limit


i.e.  = 

\item To find   we use the “Property of Symmetry”

“Property of Symmetry” -   for any value A


\item From Murdoch Barnes Tables (page 13)  

Therefore   = 0.02275 

\item \textbf{Conclusion}
2.275\% of Tyres are expected to last less than 40,000km

Of a Batch of 100,000 tyres,  2270 tyres will wear out before the warranty expires.
\end{itemize}
%==================================================%


\section{Worked Example 10 - with Solutions}
%- Spring 2006 Q3.    (a)	

The breaking strength of a certain type of plastic block is normally distributed with a
mean of 1500kg and standard deviation of 50kg. 

\begin{itemize}
	\item (iv)	What is the probability that a block with have a breaking strength greater than 1570kg?
	\item (v)	What is the probability that a block with have a breaking strength measuring between 1482kg and 1518kg?
	\item (iii)	Determine the maximum load such that no more than 5`\% of the blocks break?
\end{itemize}
