\documentclass[a4paper,12pt]{article}
%%%%%%%%%%%%%%%%%%%%%%%%%%%%%%%%%%%%%%%%%%%%%%%%%%%%%%%%%%%%%%%%%%%%%%%%%%%%%%%%%%%%%%%%%%%%%%%%%%%%%%%%%%%%%%%%%%%%%%%%%%%%%%%%%%%%%%%%%%%%%%%%%%%%%%%%%%%%%%%%%%%%%%%%%%%%%%%%%%%%%%%%%%%%%%%%%%%%%%%%%%%%%%%%%%%%%%%%%%%%%%%%%%%%%%%%%%%%%%%%%%%%%%%%%%%%
\usepackage{eurosym}
\usepackage{vmargin}
\usepackage{framed}
\usepackage{amsmath}
\usepackage{graphics}
\usepackage{epsfig}
\usepackage{subfigure}
\usepackage{enumerate}
\usepackage{fancyhdr}

\setcounter{MaxMatrixCols}{10}
%TCIDATA{OutputFilter=LATEX.DLL}
%TCIDATA{Version=5.00.0.2570}
%TCIDATA{<META NAME="SaveForMode"CONTENT="1">}
%TCIDATA{LastRevised=Wednesday, February 23, 201113:24:34}
%TCIDATA{<META NAME="GraphicsSave" CONTENT="32">}
%TCIDATA{Language=American English}

%\pagestyle{fancy}
\setmarginsrb{20mm}{0mm}{20mm}{25mm}{12mm}{11mm}{0mm}{11mm}
%\lhead{StatsResource } \rhead{Kevin O'Brien} \chead{Confidence Intervals} %\input{tcilatex}

\begin{document}

Q. 4) \item  In a city during rainy season the days were observed to be raining (R) or not raining
(S) on any particular day. The data was recorded for a month of Jun2020 as below:
Week 1: RSRRSSS
Week 2: SRRSRSS
Week 3: RSRSRRS
Week 4: SSRRSSR
Week 5: RR
It was assumed that the rain prediction is dependent only on previous day and hence
decided to fit Markov chain.
a) Calculate transition probabilities for the Markov chain. 
b) Determine the probability that it will rain on 3rd July 2020. (3)
\item   A life insurance agent sells on an average three life insurance policies per week. Use
Poisson's law to calculate the probability that in a given week he will sell:
IAI CS2A-1120
Page 7 of 8
a) Some policies. 
b) 2 or more policies but less than 5 policies. (2.5)
c) Assuming that there are 5 working days per week, what is the probability that
in a given day he will sell one policy? (2.5)



Solution 4:
i)
a) Data : RSRRSSSSRRSRSSRSRSRRSSSRRSSRRR
prr = 6/14=3/7
prs = 8/14=4/7
psr = 8/15
pss = 7/15
[2]
b)
30th June 1st Jul 2nd Jul 3rd Jul
R S S R
prs = 8/14 pss = 7/15 psr = 8/15 0.142222
R S R R
prs = 8/14 psr = 8/15 prr = 6/14 0.130612
R R S R
prr = 6/14 prs = 8/14 psr = 8/15 0.130612
R R R R
prr = 6/14 prr = 6/14 prr = 6/14 0.078717
IAI CS2A-1120
Page 5 of 9
Total = 0.482164 [3]
ii)
a)
Here, 
“Some policies “ means “1 or more policies” i.e 1 minus the “zero policies” probability:
𝑃(𝑋 > 0) = 1 − 𝑃(𝑥􀬴)
Now, 𝑃(𝑋) =
􀯘􀰷􀴋􀰓􀳣
􀯫!
So, 𝑃(𝑥􀬴) =
􀯘􀰷􀰯􀬷􀰬
􀬴!
= 4.9787 x 10-2 [1]
Therefore the probability of 1 or more policies is given by:
Probability = 𝑃(𝑋 ≥ 0)
= 1 − 𝑃(𝑥􀬴)
= 1- 4.9787 x 10-2
= 0.95021 [1]
[2]
b)
The probability of selling 2 or more, but less than 5 policies is:
𝑃(2 ≤ 𝑋 < 5) = 𝑃(𝑥􀬶) + 𝑃(𝑥􀬷) + 𝑃(𝑥􀬸 ) [1]
=
􀯘􀰷􀰯􀬷􀰮
􀬶!
+
􀯘􀰷􀰯􀬷􀰯
􀬷!
+
􀯘􀰷􀰯􀬷􀰰
􀬸!
= 0.61611 [1.5]
[2.5]
c)
Average number of policies sold per day: 􀬷
􀬹
= 0.6 [1]
So on a given day, 𝑃(𝑋) =
􀯘􀰷􀰬.􀰲􀬴.􀬺􀰭
􀬵!
= 0.32929 [1.5]
[2.5]
[12 Marks]

