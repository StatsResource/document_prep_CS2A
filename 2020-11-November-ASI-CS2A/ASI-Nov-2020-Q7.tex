\documentclass[a4paper,12pt]{article}
%%%%%%%%%%%%%%%%%%%%%%%%%%%%%%%%%%%%%%%%%%%%%%%%%%%%%%%%%%%%%%%%%%%%%%%%%%%%%%%%%%%%%%%%%%%%%%%%%%%%%%%%%%%%%%%%%%%%%%%%%%%%%%%%%%%%%%%%%%%%%%%%%%%%%%%%%%%%%%%%%%%%%%%%%%%%%%%%%%%%%%%%%%%%%%%%%%%%%%%%%%%%%%%%%%%%%%%%%%%%%%%%%%%%%%%%%%%%%%%%%%%%%%%%%%%%
\usepackage{eurosym}
\usepackage{vmargin}
\usepackage{framed}
\usepackage{amsmath}
\usepackage{graphics}
\usepackage{epsfig}
\usepackage{subfigure}
\usepackage{enumerate}
\usepackage{fancyhdr}

\setcounter{MaxMatrixCols}{10}
%TCIDATA{OutputFilter=LATEX.DLL}
%TCIDATA{Version=5.00.0.2570}
%TCIDATA{<META NAME="SaveForMode"CONTENT="1">}
%TCIDATA{LastRevised=Wednesday, February 23, 201113:24:34}
%TCIDATA{<META NAME="GraphicsSave" CONTENT="32">}
%TCIDATA{Language=American English}

%\pagestyle{fancy}
\setmarginsrb{20mm}{0mm}{20mm}{25mm}{12mm}{11mm}{0mm}{11mm}
%\lhead{StatsResource } \rhead{Kevin O'Brien} \chead{Confidence Intervals} %\input{tcilatex}

\begin{document}


[8]
Q. 7) In country of Originia with dense population in metro cities and having 30 geographical
states, the study was undertaken by using the recent data collected for the latest pandemic
in the world. The data available is only for one full year from 1st infection was observed
till vaccine is prepared.

%==============================================%
\begin{enumerate}[(a)]
\item  Explain why lives might be lost to the investigation if we are carrying out study of
multistate model (states such as normal, infected, recovered, death). 
\item   Explain the above in context of informative and non-informative censoring. 
\item    Any other type of censoring is demonstrated by this data. 
In order to perform survival analysis, you have been asked to investigate the impact of
covariates.
\item  List down possible covariates to perform survival analysis. (3)
\item  Explain possible lifetime distributions for proportional hazard function. Discuss or
debate its advantages or disadvantages for performing survival analysis for
population of Originia. 
\end{enumerate}

%==============================================%




Solution 7:
i)
Lost to the investigation for each state of the model.
 Infected may or may not get tested if testing is not compulsory. Thus, may be considered in “Normal”
state. Thus lost from “infected” state and in turn from “recovered” or “death” state.
 Infected may recover even before the testing is suggested by doctor. Thus, may be considered in
“Normal” state. Thus lost from “infected” state and in turn from “recovered” or “death” state.
 Any other cause of death than this infection will be considered transited in “Normal” to “death” state.
However, if it is not tested it will be lost from “infected” and then to “death” state.
 There will not be any lost to investigation for “Normal” state.

%==============================================%
ii)
It is non-informative censoring for lifetime investigation.
 Infected may or may not get tested if testing is not compulsory. Thus, may be considered in “Normal”
state. Thus lost from “infected” state and in turn from “recovered” or “death” state. – Non-informative
censoring as information of lifetime to perform survival/ mortality analysis is not captured under
appropriate state.
 Infected may recover even before the testing is suggested by doctor. Thus, may be considered in
“Normal” state. Thus lost from “infected” state and in turn from “recovered” or “death” state. – Noninformative
censoring as information of lifetime to perform survival/ mortality analysis is not captured
under appropriate state.
 Any other cause of death than this infection will be considered transited in “Normal” to “death” state.
However, if it is not tested it will be lost from “infected” and then to “death” state. – Non-informative
censoring as information of lifetime to perform survival/ mortality analysis is not captured under
appropriate state.
 For “Normal”state – Informative censoring as the survival expectation of this group is higher than the
“infected” or “recovered” individuals.
%==============================================%
iii)
Other types of censoring as below:
 Right censoring – since till vaccine is prepared and not till end of infection. 
 Interval censoring – since we may only know the age to the nearest birthday and not exact date of birth
as we have started capturing the information from certain time point. 
[1]
%==============================================%
iv)
List of covariates to perform survival analysis:
a) Age
b) Gender
c) Severity of symptoms
d) Number of days infection lasted
e) Co-morbid conditions such as BP, diabetes
f) Physically Active/ non-active
g) Level of fitness –no. of hours of weakly exercise/ activities
h) Profession
i) Socio – economic group / Income range
j) States
k) Metro/Non-metro city
l) Density of population
%==============================================%
v)
The lifetime distributions for proportional hazard functions are as follows:
a) The Exponential distribution 
The hazard rate under this lifetime distribution function is constant. 
This constant hazard model could reflect the hazard for an individual who remains in good health.
Here, the level of hazard would reflect the risk of death from unnatural causes. 
b) The Weibull distribution 
The hazard rate under this lifetime distribution function is monotonically decreasing or increasing.

The decreasing hazard model could reflect hazard for patients recovering from major surgery. The
level of hazard is expected to fall as the time since the operation increases. 
%==============================================%
c) The Gompertz-Makeham formula 
The lifetime distribution function has exponential hazard rate. 
The exponential hazard model could reflect the hazard for leukaemia sufferers who are not
responding to treatment. The severity of condition and the level of hazard increase with the survival
time. OR Over longer period exponential hazard rate could reflect the increasing chance of death
from natural causes as age increases. 
%==============================================%
d) The log-logistic distribution (humped hazard) 
The lifetime distribution function has humped hazard rate. 
The humped hazard model could reflect the hazard for patients with disease that is most likely to
cause death during the early stages. As the initial condition becomes more severe, the level of
hazard increases. But once the patient survives the period of highest risk, the level of hazard
decreases. 

%==============================================%
\newpage 

To perform survival analysis of population of Originia:
a) The Exponential distribution can be used for the population which continues to be in “normal”
state as they are expected to follow the natural and normal survival/ death expectations.
However, it may not be applicable for all age group.

b) The Weibull distribution can be used for the population which recovers from the being infected.
Their survival probability is expected to get better and better as they recover from being
infected. The pandemic is fairly new and it will not be very clear whether the recovered
population may or may not have any future implications/ re-occurrence/ other side effects due
to getting infected once.
c) The exponential hazard function could be appropriate for age group which is most impacted by
this infection. The severity of condition and the level of hazard may increase with the survival
for short span of time. Also, they may be no time to observe this trend in case the infection
results into deaths in really very short span of time. It does not look appropriate to apply this
hazard function to any of the state of the analysis.
d) The log-logistic distribution (humped hazard) is likely to reflect the hazard infected population
with most impacted age group as it is most likely to cause death during the early stages. As the
initial condition becomes more severe, the level of hazard increases. But once the patient
survives the period of highest risk, the level of hazard decreases.

%==============================================%
\end{document}