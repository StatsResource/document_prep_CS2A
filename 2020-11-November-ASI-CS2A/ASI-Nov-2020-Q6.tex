\documentclass[a4paper,12pt]{article}
%%%%%%%%%%%%%%%%%%%%%%%%%%%%%%%%%%%%%%%%%%%%%%%%%%%%%%%%%%%%%%%%%%%%%%%%%%%%%%%%%%%%%%%%%%%%%%%%%%%%%%%%%%%%%%%%%%%%%%%%%%%%%%%%%%%%%%%%%%%%%%%%%%%%%%%%%%%%%%%%%%%%%%%%%%%%%%%%%%%%%%%%%%%%%%%%%%%%%%%%%%%%%%%%%%%%%%%%%%%%%%%%%%%%%%%%%%%%%%%%%%%%%%%%%%%%
\usepackage{eurosym}
\usepackage{vmargin}
\usepackage{framed}
\usepackage{amsmath}
\usepackage{graphics}
\usepackage{epsfig}
\usepackage{subfigure}
\usepackage{enumerate}
\usepackage{fancyhdr}

\setcounter{MaxMatrixCols}{10}
%TCIDATA{OutputFilter=LATEX.DLL}
%TCIDATA{Version=5.00.0.2570}
%TCIDATA{<META NAME="SaveForMode"CONTENT="1">}
%TCIDATA{LastRevised=Wednesday, February 23, 201113:24:34}
%TCIDATA{<META NAME="GraphicsSave" CONTENT="32">}
%TCIDATA{Language=American English}

%\pagestyle{fancy}
\setmarginsrb{20mm}{0mm}{20mm}{25mm}{12mm}{11mm}{0mm}{11mm}
%\lhead{StatsResource } \rhead{Kevin O'Brien} \chead{Confidence Intervals} %\input{tcilatex}

\begin{document}

Q. 6) Individual claims under a certain type of insurance policy are for either 1 or 2. The insurer
is considering entering into an excess of loss reinsurance arrangement with retention
1 + k (where $k < 1$). 

Let $X_i$ denote the amount paid by the insurer (net of reinsurance) on the i th claim.

%==============================================%
\begin{enumerate}
\item  Calculate and simplify expressions for the mean and variance of Xi. 



The number of claims in a year follows a Poisson distribution with mean 500 (assume
α = 0.2). The insurer wishes to set the retention so that the probability that aggregate claims
in a year will exceed 700 is less than 1%.
\item   Show that setting k = 0.334 gives the desired result for the insurer. 
\end{enumerate}
%==============================================%


[8 Marks]
Solution 6:
i)
E(Xt )= α + (1+ k)(1− α)
= 1+ k(1− α)
[1]
Var( Xt )= E(Xi
2 ) − E(Xi)2
= α + (1+ k)2 (1−α) − (1+ k(1−α))2
= α + (1−α) + 2k(1−α) + k2 (1−α) −1− 2k(1−α) – k2(1−α)2
= k2(1−α)(1− (1−α))
= k2 α(1−α) [2]
[3]
ii)
Let Y denote the aggregate claims in a year. Then Y has a compound Poisson distribution,
E(Y)=500 X E(Xi ) = 500 + 500k(1−α) = 500 + 400k =633.60
[1]
And
Var(Y ) = 500× E(Xi
2 )
= 500(α + (1+ k)2 (1−α))
= 500(α + (1−α) + 2k(1−α) + k2 (1−α))
IAI CS2A-1120
Page 7 of 9
= 500 + (1000k + 500k2 )(1−α)
= 500 + 800k + 400k2 =811.82 [1.5]
Using a normal approximation
P(Y>700)=P(Z> [700-E(Y)]/Sq rt(Var(Y)) [1]
By putting, K=0.334, we get the probability 1% (full marks who derived the 1% probability).
[1.5]
[5]
[8 Marks]
\end{document}