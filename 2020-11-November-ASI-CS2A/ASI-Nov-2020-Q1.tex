\documentclass[a4paper,12pt]{article}
%%%%%%%%%%%%%%%%%%%%%%%%%%%%%%%%%%%%%%%%%%%%%%%%%%%%%%%%%%%%%%%%%%%%%%%%%%%%%%%%%%%%%%%%%%%%%%%%%%%%%%%%%%%%%%%%%%%%%%%%%%%%%%%%%%%%%%%%%%%%%%%%%%%%%%%%%%%%%%%%%%%%%%%%%%%%%%%%%%%%%%%%%%%%%%%%%%%%%%%%%%%%%%%%%%%%%%%%%%%%%%%%%%%%%%%%%%%%%%%%%%%%%%%%%%%%
\usepackage{eurosym}
\usepackage{vmargin}
\usepackage{framed}
\usepackage{amsmath}
\usepackage{graphics}
\usepackage{epsfig}
\usepackage{subfigure}
\usepackage{enumerate}
\usepackage{fancyhdr}

\setcounter{MaxMatrixCols}{10}
%TCIDATA{OutputFilter=LATEX.DLL}
%TCIDATA{Version=5.00.0.2570}
%TCIDATA{<META NAME="SaveForMode"CONTENT="1">}
%TCIDATA{LastRevised=Wednesday, February 23, 201113:24:34}
%TCIDATA{<META NAME="GraphicsSave" CONTENT="32">}
%TCIDATA{Language=American English}

%\pagestyle{fancy}
\setmarginsrb{20mm}{0mm}{20mm}{25mm}{12mm}{11mm}{0mm}{11mm}
%\lhead{StatsResource } \rhead{Kevin O'Brien} \chead{Confidence Intervals} %\input{tcilatex}

\begin{document}


Q. 1) Find out the correct option for the below questions.
\item  The process Xt = 2+ et – 5 et-1 +6 et-2 is
a) Invertible
b) Non-invertible
c) Both Invertible and non invertible
d) Cannot be determined
e) None of the above 

\item   The process 12Xt =10Xt-1 – 2 Xt-2 +12et -11et-1 +2et-2 is
a) Stationary and Non-invertible
b) Stationary and Invertible
c) Non-Stationary and Non-invertible
d) Non-Stationary and Invertible
e) Stationary and cannot be determined Invertibility 

\item    The probability that a random loss exceeds the mean loss amount if the loss
distribution is Pareto with α=3, and λ=2000
a) 0.2396
b) 0.2963
c) 0.2369
d) 0.2936
e) None of the above 

%==============================================%

\item  Claims from a particular portfolio have a generalized Pareto distribution with α=6,
λ=2000 and k=4. A proportional reinsurance is in force with a retained proportion of
80%. The mean and variance of the amount paid by the insurer and reinsurer is
a) Insurer(64,9216), Reinsurer(32,576)
b) Insurer(64,9216), Reinsurer(64,576)
c) Insurer(128,9126), Reinsurer(64,576)
d) Insurer(128,9216), Reinsurer(32,576)
e) None of the above

%==============================================%

\item  An insurer believes that claims from a particular portfolio would follow a Pareto
distribution with parameters α=2, and λ=900. If the insurer does not want to pay for
20% claims from that portfolio what would be the policy deductible
a) 176.09
b) 200.43
c) 124.57
d) 106.23
e) None of the above

%==============================================%
\item   Which is not the required criteria for an insurable risk
a) An occurrence probability should be very high
b) Individual risk events should be independent
c) There should be an ultimate liability on insurer
d) Moral hazards should be eliminated as far as possible 

%==============================================%
\item    An insurer believes that the claims from its particular portfolio in coming year will
be log normally distributed with a mean size of INR 5000 and a standard deviation
of INR 7500. What percentage of claim that will be above INR 25000 in next year.
a) 1.3%
b) 1.8%
c) 2.1%
d) 3.5%
e) None of the above (3)

%==============================================%
\item     A motor insurer operates a no claims discount system with the following levels of
discount {0%, 10%, 20%, 35%, 60%}. The rules governing a policyholder’s discount
level, based upon the number of claims made in the previous year, are as follows:
 Following a year with no claims, the policyholder moves up one discount level,
or remains at the 60% level (maximum level).
 Following a year with one claim, the policyholder moves down one discount
level, or remains at 0% level (minimum level).
 Following a year with two or more claims, the policyholder moves down two
discount levels (subject to a limit of the 0% discount level).
The number of claims made by a policyholder in a year is assumed to follow probability
distribution as follows:
0 claim during a year = 0.75
1 claim during the year = 0.24
2 or more claim during the year = 0.01
What is the transition matrix for the no claims discount system.
a)
0.25 0.75 0 0 0
0.25 0 0.75 0 0
0.01 0.24 0 0.75 0
0 0 0.25 0 0.75
0 0 0 0.25 0.75
b)
0.24 0.75 0.01 0 0
0.24 0.01 0.75 0 0
0.01 0.24 0 0.75 0
0 0.01 0.24 0 0.75
0 0 0.01 0.24 0.75
IAI CS2A-1120
Page 4 of 8
c)
0.25 0.75 0 0 0
0.25 0 0.75 0 0
0.01 0.24 0 0.75 0
0 0.01 0.24 0 0.75
0 0 0.01 0.24 0.75
d)
0.25 0.75 0 0 0
0.25 0 0.75 0 0
0 0.25 0 0.75 0
0 0 0.25 0 0.75
0 0 0 0.25 0.75
(3)
ix) For transition probabilities derived in (\item     above, find out the stationary distribution:
a) (0.0422, 0.1264, 0.3782, 0.1134, 0.3411)
b) (0.0112, 0.0302, 0.0818, 0.2192, 0.6576)
c) (0.2287, 0.2160, 0.1850, 0.0926, 0.2777)
d) (1, 0, 0, 0, 0)
e) None of the above
(5)
x) Select the correct forward differential equation using first principles for 􀰡􀳟􀯣􀳣
􀰮􀰰
􀰡􀯧
for
following multiple state model in which S(t), the state occupied at time t by a life
initially aged x, is assumed to follow a continuous- time Markov process.
Let 𝜇􀯫􀬾􀯧
􀯜􀯝 denote the force of transition at age x+t (t >= 0) from state i to state j, and let
𝑡𝑝
􀯫
􀯜􀯝
= 𝑃 (𝑆(𝑡) = 𝑗 | 𝑆(0) = 𝑖)
a) 􀰡􀳟􀯣􀳣
􀰮􀰰
􀰡􀯧
= 𝑡𝑝
􀯫
􀬶􀬵
𝜇􀯫􀬾􀯧
􀬵􀬸 + 𝑡𝑝
􀯫
􀬶􀬷
𝜇􀯫􀬾􀯧
􀬷􀬸 + 𝑡𝑝
􀯫
􀬶􀬸
𝜇􀯫􀬾􀯧
􀬸􀬸 − 𝑡𝑝
􀯫
􀬶􀬶
𝜇􀯫􀬾􀯧
􀬶􀬸
b) 􀰡􀳟􀯣􀳣
􀰮􀰰
􀰡􀯧
= 𝑡𝑝
􀯫
􀬶􀬷
𝜇􀯫􀬾􀯧
􀬷􀬸 + 𝑡𝑝
􀯫
􀬶􀬵
𝜇􀯫􀬾􀯧
􀬵􀬸 + 𝑡𝑝
􀯫
􀬶􀬸
𝜇􀯫􀬾􀯧
􀬸􀬶 − 𝑡𝑝
􀯫
􀬶􀬶
𝜇􀯫􀬾􀯧
􀬶􀬸
State 1 State 2 State 3
State 4
IAI CS2A-1120
Page 5 of 8
c) 􀰡􀳟􀯣􀳣
􀰮􀰰
􀰡􀯧
= 𝑡𝑝
􀯫
􀬶􀬷
𝜇􀯫􀬾􀯧
􀬷􀬸 + 𝑡𝑝
􀯫
􀬶􀬵
𝜇􀯫􀬾􀯧
􀬵􀬸 + 𝑡𝑝
􀯫
􀬶􀬵
𝜇􀯫􀬾􀯧
􀬵􀬶 − 𝑡𝑝
􀯫
􀬶􀬸
𝜇􀯫􀬾􀯧
􀬸􀬶
d) 􀰡􀳟􀯣􀳣
􀰮􀰰
􀰡􀯧
= 𝑡𝑝
􀯫
􀬶􀬵
𝜇􀯫􀬾􀯧
􀬵􀬸 + 𝑡𝑝
􀯫
􀬶􀬷
𝜇􀯫􀬾􀯧
􀬷􀬸 + 𝑡𝑝
􀯫
􀬶􀬶
𝜇􀯫􀬾􀯧
􀬶􀬸 − 𝑡𝑝
􀯫
􀬶􀬸
𝜇􀯫􀬾􀯧
􀬸􀬶
e) None of the above
(5)
x\item  Match the below examples with the correct stochastic process.
Example Stochastic process
(\item  Wrestlers weight every year on 1st Jan (A) Discrete time discrete space
(\item   Monthly car accidents in Mumbai city (B) Continuous time discrete space
(\item    Temperature of incubator while hatching
chicken eggs
(C) Discrete time continuous space
(\item  Trend of Air pollution index on meter (D) Continuous time continuous space

\section*{Solution}
Solution 1:
i) Answer – (b) [2]
ii) Answer – (b) [2]
iii) Answer – (b) [2]
iv) Answer – (a) [2.5]
v) Answer – (d) [2.5]
vi) Answer – (a) [1]
vii) Answer – (c) [3]
viii) Answer – (c) [3]
ix) Answer – (b) [5]
x) Answer – (d) [5]
xi) Answer – (i)-(C), (ii)-(A), (iii)-(D), (iv)-(B) [2]
[30 Mark]

%==============================================%
\end{document}