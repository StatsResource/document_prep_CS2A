\documentclass[a4paper,12pt]{article}
%%%%%%%%%%%%%%%%%%%%%%%%%%%%%%%%%%%%%%%%%%%%%%%%%%%%%%%%%%%%%%%%%%%%%%%%%%%%%%%%%%%%%%%%%%%%%%%%%%%%%%%%%%%%%%%%%%%%%%%%%%%%%%%%%%%%%%%%%%%%%%%%%%%%%%%%%%%%%%%%%%%%%%%%%%%%%%%%%%%%%%%%%%%%%%%%%%%%%%%%%%%%%%%%%%%%%%%%%%%%%%%%%%%%%%%%%%%%%%%%%%%%%%%%%%%%
\usepackage{eurosym}
\usepackage{vmargin}
\usepackage{framed}
\usepackage{amsmath}
\usepackage{graphics}
\usepackage{epsfig}
\usepackage{subfigure}
\usepackage{enumerate}
\usepackage{fancyhdr}

\setcounter{MaxMatrixCols}{10}
%TCIDATA{OutputFilter=LATEX.DLL}
%TCIDATA{Version=5.00.0.2570}
%TCIDATA{<META NAME="SaveForMode"CONTENT="1">}
%TCIDATA{LastRevised=Wednesday, February 23, 201113:24:34}
%TCIDATA{<META NAME="GraphicsSave" CONTENT="32">}
%TCIDATA{Language=American English}

%\pagestyle{fancy}
\setmarginsrb{20mm}{0mm}{20mm}{25mm}{12mm}{11mm}{0mm}{11mm}
%\lhead{StatsResource } \rhead{Kevin O'Brien} \chead{Confidence Intervals} %\input{tcilatex}

\begin{document}

[12]
Q. 5) In the country of Indiana the testing kits for pandemic Covid-19 are limited in numbers.
So the Government has decided to test only those citizens who qualify certain criteria.
Citizens of the country are given a questionnaire and asked for having physical symptoms
of Covid-19. The questionnaire asked some simple questions like international travel
history, coming in contact with any positive tested person, any co-morbid conditions. The
symptom checked were fever, cough, difficulty in breathing, and any other respiratory
problem persisting since last 3 days. An algorithm is then applied to estimate the number
of infected persons in the country to augment medical services

In the test, both the questionnaire and the symptoms test are given to 2000 patients whose
disease status is known. Some results from the test are shown in the table below
True Status
Prediction from Symptom Prediction from questions
Has
disease
Does not
have disease
Has disease
Does not have
disease
Has disease 900 100 800 200
Does not have disease 300 700 200 800

%==============================================%
\item  Find out the correct option for (symptoms, questionnaire) test
a) Precision: (0.75,0.80) ; Recall: (0.90,0.80), F1 Score: (0.818, 0.80)
b) Precision: (0.90,0.80) ; Recall: (0.75,0.80), F1 Score: (0.90, 0.80)
c) Precision: (0.75,0.90) ; Recall: (0.75,0.80), F1 Score: (0.80, 0.90)
d) Precision: (0.75,0.90) ; Recall: (0.90,0.75), F1 Score: (0.818, 0.90)
e) None of the above
(3)
\item   Show that the F1 score expresses the true positives as a proportion of the true
positives plus the average of those incorrectly classified. 
\item    Comment on your result from (\item  and on the usefulness of the test. 

%==============================================%

Solution 5:
i) Answer – (a) [3]
ii)
The F1 score may be written as
2*[TP/(TP+FP)]*[TP/(TP+FN)]
[TP/(TP+FP)]+[TP/(TP+FN)]
IAI CS2A-1120
Page 6 of 9
Solving this, we get F1 = TP/(TP+0.5FN+0.5FP)
This expresses the true positives as a proportion of the true positives plus the average of those
incorrectly classified.
[2]
iii)
Comments -
 Compared with questionnaire, the symptom test is better at identifying the true positives
 But it is not so precise as it classifies as positive a higher proportion of those who do not have the
disease.
 Whether recall or precision are chosen as measures will depend on whether it is most important to
identify all the persons who have the disease, or not to unduly worry and treat people who are diseasefree
 As the disease is serious it would perhaps be best to maximise the true positives and minimise the
false negatives and so the clinical procedure would be preferred.
 In real life, a very large proportion of those tested will not have the disease, so testing equal numbers
of patients who do and do not have the disease may not be so useful.
 The F1 score, however, is reasonably robust to the situation where most people do not have the
disease, as its calculation does not involve the true negatives
 As the sample size is relatively small, the test should be re-performed on a larger population before
drawing any firm conclusions.
 The questionnaire is likely to be easier/cheaper to administer and therefore may be a good short-term
substitute until the clinical procedure can be established in areas that currently have no screening in
place.
[3]
