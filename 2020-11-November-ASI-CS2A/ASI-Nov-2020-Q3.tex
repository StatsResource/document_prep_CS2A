\documentclass[a4paper,12pt]{article}
%%%%%%%%%%%%%%%%%%%%%%%%%%%%%%%%%%%%%%%%%%%%%%%%%%%%%%%%%%%%%%%%%%%%%%%%%%%%%%%%%%%%%%%%%%%%%%%%%%%%%%%%%%%%%%%%%%%%%%%%%%%%%%%%%%%%%%%%%%%%%%%%%%%%%%%%%%%%%%%%%%%%%%%%%%%%%%%%%%%%%%%%%%%%%%%%%%%%%%%%%%%%%%%%%%%%%%%%%%%%%%%%%%%%%%%%%%%%%%%%%%%%%%%%%%%%
\usepackage{eurosym}
\usepackage{vmargin}
\usepackage{framed}
\usepackage{amsmath}
\usepackage{graphics}
\usepackage{epsfig}
\usepackage{subfigure}
\usepackage{enumerate}
\usepackage{fancyhdr}

\setcounter{MaxMatrixCols}{10}
%TCIDATA{OutputFilter=LATEX.DLL}
%TCIDATA{Version=5.00.0.2570}
%TCIDATA{<META NAME="SaveForMode"CONTENT="1">}
%TCIDATA{LastRevised=Wednesday, February 23, 201113:24:34}
%TCIDATA{<META NAME="GraphicsSave" CONTENT="32">}
%TCIDATA{Language=American English}

%\pagestyle{fancy}
\setmarginsrb{20mm}{0mm}{20mm}{25mm}{12mm}{11mm}{0mm}{11mm}
%\lhead{StatsResource } \rhead{Kevin O'Brien} \chead{Confidence Intervals} %\input{tcilatex}

\begin{document}

Q. 3) The life insurance company performing the analysis current experience (5 years) with
population for its maximum selling age group of 30-39 years
\item  Based on below data please test the hypothesis that the company’s experience is in
line with population.
Age
Exposed to
risk
Actual
Deaths
Population
mortality
30 2,921 5 0.001605
31 4,040 6 0.001652
32 3,855 7 0.001712
33 4,640 7 0.001787
34 4,822 9 0.001875
35 5,583 10 0.001980
36 6,013 13 0.002104
37 5,911 12 0.002247
38 5,657 13 0.002412
39 5,219 14 0.002603
a) Calculate the test statistic (4)
b) State degrees of freedom (0.5)
c) State upper tail value of the Chi-square distribution with calculated degrees of
freedom at 95% (0.5)
d) Conclusion on the hypothesis tested 
\item   Perform signs test by stating n, p and x. State the p-value and conclusion whether the
Null hypothesis of positive deviations is having binomial distribution. (4)


%==============================================%
Solution 3:
i) The Null Hypothesis is that the company’s experience is in line with population
a)
Age Observed (O) Expected ( E) = Population mortality* Exposed to risk (O-E)^2 (O-E)^2/E
30 5 4.688205 0.097216 0.020736
31 6 6.67408 0.454384 0.068082
32 7 6.59976 0.160192 0.024272
33 7 8.29168 1.668437 0.201218
34 9 9.04125 0.001702 0.000188
35 10 11.05434 1.111633 0.100561
36 13 12.65135 0.121555 0.009608
37 12 13.28202 1.643568 0.123744
IAI CS2A-1120
Page 4 of 9
38 13 13.64468 0.415617 0.03046
39 14 13.58506 0.172178 0.012674
Total 96 99.51243 5.846482 0.591544
Test statistic = 0.59 [4]
b) Degrees of freedom = 10 
c) upper tail value of the Chi-square distribution with 10 degrees of freedom at 95% is 18.31 
d) The company’s experience is in line with population as there is not sufficient evidence to reject the
null hypothesis. 0.59 < 18.31 [1]
ii)
No. of positive = 4
N = total no. of groups analysed = 10
P follows Binomial (10,1/2)
2*P(P<= 4) = 2*0.3770
[2]
Page 187 of tables by looking up n=10, p=0.5, x=4
p-value = 0.3770 [1]
Since p-value is greater than 5%, it is not significant. [1]
[4]
[10 Marks]

\end{document}