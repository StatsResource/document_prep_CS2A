
\section{KB Tutorial 2}

\subsection*{Question 4}
Consider a RAID (redundant array of inexpensive disks) system where multiple hard disks are used simultaneously.\\[0.2cm]
Let's assume that we have two hard disks that work \emph{independently} of each other. Define the events $H_1 =$ ``hard disk one works'' and $H_2 =$ ``hard disk two works'' and also assume that $\Pr(H_1) = \Pr(H_2) = 0.9$.\\ \smallskip
\begin{itemize}
	\item RAID-0 is a system which increases performance but only works if \emph{both} hard disks work.
	\item RAID-1 is a system which does not increase performance but still works with only one working hard disk.
\end{itemize}

{\bf(a)} Calculate $\Pr(\text{RAID-0 works})$ and $\Pr(\text{RAID-0 fails})$. \\[0.3cm] 
 \item  Calculate $\Pr(\text{RAID-1 works})$ and $\Pr(\text{RAID-1 fails})$. 
 \item  Calculate $\Pr(H_1^c)$ and $\Pr(H_2^c)$. \\[0.3cm]

 \item  Cheap hard disks exist with $\Pr(H) = 0.6$. Consider a RAID-1 system with 3 of these hard disks - calculate $\Pr(\text{RAID-1 fails})$ in this case. \\[0.3cm] 
 \item  In part (a) we found that $\Pr(\text{RAID-1 fails}) = 0.01$. How many cheap disks would be required to match this level of reliability?




%-----------------------------------------------------------------------------------------------------------%
\newpage
\section*{MA4603 and MA4505 Tutorial 2 (Week 3)}
Remark : No SPSS related questions this week 



\section{Tutorial C - Probability}	
\subsection*{Question 1}
Assume that there are three different routes to get to a particular location: $R_1$, $R_2$ and $R_3$. You take $R_1$ 75\% of the time, $R_2$ 20\% of the time and $R_3$ the rest of the time. If you take $R_1$, there is a 90\% chance that you will be on time; if you take $R_2$, there is a 50\% chance that you will be on time and, if you take $R_3$, there is a 70\% chance that you will be on time. \\[0.1cm]
Let $T$ represent on time.\\[-0.2cm]

{\bf(a)} If $T$ represents ``on time'', what notation would we use for ``late''? 
 \item  What is the value of $\Pr(R_1 \cap R_2)$? 
 \item  Calculate the probability of being on time. 
 \item  \emph{Given that} you \emph{are} on time, calculate the probabilities of having used each of the routes. 
 \item  Given that you are late, what is the probability that you used $R_1$?









\subsection*{Question 4}
Let $X \sim \text{Exponential}(\lambda=0.02)$. Calculate the following:\\[-0.2cm]

{\bf(a)} $\Pr(\,\overline{\!X} > 55)$ in a group of 100. 
 \item  $\Pr(\,\overline{\!X} < 53)$ in a group of 40. 
 \item  The value of $\bar x$ such that $\Pr(\,\overline{\!X} > \bar x) = 0.1$ when $n=65$. 
 \item  The value of $n$ if $\Pr(\,\overline{\!X} < 49) = 0.1$.



\section{KB Tutorial 3}
\subsection*{Question 1}
Assume that there are three different routes to get to a particular location: $R_1$, $R_2$ and $R_3$. You take $R_1$ 75\% of the time, $R_2$ 20\% of the time and $R_3$ the rest of the time. If you take $R_1$, there is a 90\% chance that you will be on time; if you take $R_2$, there is a 50\% chance that you will be on time and, if you take $R_3$, there is a 70\% chance that you will be on time. \\[0.1cm]
Let $T$ represent on time.\\[-0.2cm]

{\bf(a)} If $T$ represents ``on time'', what notation would we use for ``late''? 
 \item  What is the value of $\Pr(R_1 \cap R_2)$? 
 \item  Calculate the probability of being on time. 
 \item  \emph{Given that} you \emph{are} on time, calculate the probabilities of having used each of the routes. 
 \item  Given that you are late, what is the probability that you used $R_1$?






